%
% $XORP: xorp/docs/user_manual/ospf.tex,v 1.17 2008/07/23 02:24:55 atanu Exp $
% 

\chapter{OSPFv2 and OSPFv3}
\label{ospf}
\section{OSPF Terminology and Concepts}

OSPF is the Open Shortest Path First protocol, it is one of two
principle Interior Gateway Protocols (IGP) the other being IS-IS. RIP
is also an IGP however OSPF and IS-IS have better scaling properties.

The initial OSPF specification is RFC 2328 OSPF Version 2 or
OSPFv2. This specification is specific to IPv4. There is a later
specification draft-ietf-ospf-ospfv3-update-23.txt OSPF for IPv6,
which happens to be Version 3 or OSPFv3.

The specifications for OSPFv2 and OSPFv3 are fairly similar the
obvious difference is the handing of IPv4 and IPv6 addresses.

The XORP implementation of OSPF supports both OSPFv2 and OSPFv3.

For consistency with our other protocols OSPFv2 is {\tt ospf4} and
OPSFv3 is {\tt ospf6} in the configuration files, the {\tt 4} in {\tt
ospf4} refers to the IPv4 address family and {\tt 6} in {\tt ospf6}
refers to the IPv6 address family.

\subsection{Key OSPF Concepts}

As an Interior Gateway Protocol OSPF runs within a single Autonomous
System. One way that OSPF achieves good scaling properties is to allow
an AS to be split into distinct regions that OSPF calls areas. The areas
are structured in a two level hierarchy, area {\tt 0.0.0.0} is special
and is called the {\tt BACKBONE} area. All other areas must be
connected to that {\tt BACKBONE} either directly or through virtual links.

A fundamental quantity in OSPF that describes topology and routing
information is the Link State Advertisement (LSA). Every OSPF router
within an area should have exactly the same LSAs in its
database. There are different type of LSAs the base specification
describes Router-LSAs, Network-LSAs, Summary-LSAs and AS-external-LSAs.

The OSPF protocol has explicit support for introducing routes from
other protocols, these routes are introduced via AS-external-LSAs. For
example, routes from a RIP cloud can be introduced into OSPF and will
appear as AS-external-LSAs.

Areas in OSPF can be one of three different types {\tt normal}, {\tt
stub} and {\tt Not-So-Stubby}. The {\tt BACKBONE} is always {\tt
normal}. All LSAs in OSPF are flooded only within an area, the
exception is the AS-external-LSA that can be flooded between all {\tt
normal} areas.

OSPF routers are categorised into four overlapping categories, an
internal router whose interfaces are all in one area, an area border
router ({\tt ABR}) that has interfaces in more than one area, a backbone router
that has an interface to the {\tt BACKBONE} and an AS boundary router
that introduces routes from other protocols.

An AS boundary router can be configured in {\tt normal} and {\tt
Not-So-Stubby} areas. In a {\tt normal} area the AS boundary router
generates an AS-external-LSA that is flooded to all other {\tt normal}
areas. In a {\tt Not-So-Stubby} an AS boundary router generates a
Type-7 LSA that {\bf may} be translated at a area border router to an
AS-external-LSA, which will be flooded to all {\tt normal} areas. An
AS-external-LSA is never flooded into a {\tt stub} or {\tt
Not-So-Stubby} area. The different types of areas exist to limit the
number of LSAs in a particular area, for example a {\tt stub} area may
have a small number of internal routes and default routes to the
{\tt ABR}s.

\section{Standards}

XORP OSPF complies with the following standards:
\begin{description}
\item{\bf RFC 2328}: OSPF Version 2
\item{\bf draft-ietf-ospf-ospfv3-update-23.txt} OSPF for IPv6
\item{\bf RFC 3101}: The OSPF Not-So-Stubby Area (NSSA) Option
\end{description}

\newpage
\section{Configuring OSPF}

\subsection{Configuration Syntax}

The configuration syntax for XORP OSPFv2 is given below.

\vspace{0.1in}
\noindent\framebox[\textwidth][l]{\scriptsize
\begin{minipage}{6in}
\begin{alltt}
\begin{tabbing}
xx\=xx\=xx\=xx\=xx\=xx\=xx\=xx\=xx\=xx\=xx\=xx\=\kill
protocols \{\\
\>ospf4 \{\\
\>\>targetname: {\it text}\\
\>\>router-id: {\it IPv4}\\
\>\>rfc1583-compatibility: {\it bool};\\
\>\>ip-router-alert: {\it bool}\\
\\
\>\>traceoptions \{\\
\>\>\>flag \{\\
\>\>\>\>all \{\\
\>\>\>\>\>disable: {\it bool}\\
\>\>\>\>\}\\
\>\>\>\}\\
\>\>\}\\
\\
\>\>area {\it IPv4} \{\\
\>\>\>area-type: {\it text}\\
\\
\>\>\>default-lsa \{\\
\>\>\>\>disable: {\it bool}\\
\>\>\>\>metric: {\it uint(0..0xffffff)}\\
\>\>\>\}\\
\\
\>\>\>summaries \{\\
\>\>\>\>disable: {\it bool}\\
\>\>\>\}\\
\\
\>\>\>area-range {\it IPv4Net} \{\\
\>\>\>\>advertise: {\it bool}\\
\>\>\>\}\\
\\
\>\>\>virtual-link {\it IPv4} \{\\
\>\>\>\>transit-area: {\it ipv4}\\
\>\>\>\>hello-interval: {\it uint(1..65535)}\\
\>\>\>\>router-dead-interval: {\it uint(1..65535)}\\
\>\>\>\>retransmit-interval: {\it uint(1..65535)}\\
\>\>\>\>transit-delay: {\it uint(0..3600)}\\
\>\>\>\>authentication \{\\
\>\>\>\>\>simple-password: {\it text}\\
\\
\>\>\>\>\>md5 {\it uint(0..255)} \{\\
\>\>\>\>\>\>password: {\it text}\\
\>\>\>\>\>\>start-time: {\it text(``YYYY-MM-DD.HH:MM'')}\\
\>\>\>\>\>\>end-time: {\it text(``YYYY-MM-DD.HH:MM'')}\\
\>\>\>\>\>\>max-time-drift: {\it uint(0..65535)}\\
\>\>\>\>\>\}\\
\>\>\>\>\}\\
\>\>\>\}\\
\\
{\rm continued overleaf....}
\end{tabbing}
\end{alltt}
\end{minipage}
}
\vspace{0.1in}

\newpage
\vspace{0.1in}
\noindent\framebox[\textwidth][l]{\scriptsize
\begin{minipage}{6in}
\begin{alltt}
\begin{tabbing}
xx\=xx\=xx\=xx\=xx\=xx\=xx\=xx\=xx\=xx\=xx\=xx\=\kill
\>\>\>interface {\it text} \{\\
\>\>\>\>link-type: {\it text}\\
\\
\>\>\>\>vif {\it text} \{\\
\\
\>\>\>\>\>address {\it IPv4} \{\\
\>\>\>\>\>\>priority: {\it uint(0..255)}\\
\>\>\>\>\>\>hello-interval: {\it uint(1..65535)}\\
\>\>\>\>\>\>router-dead-interval: {\it uint(1..4294967295)}\\
\>\>\>\>\>\>interface-cost: {\it uint(1..65535)}\\
\>\>\>\>\>\>retransmit-interval: {\it uint(1..65535)}\\
\>\>\>\>\>\>transit-delay: {\it uint(0..3600)}\\
\>\>\>\>\>\>authentication \{\\
\>\>\>\>\>\>\>simple-password: {\it text}\\
\\
\>\>\>\>\>\>\>md5 {\it uint(0..255)} \{\\
\>\>\>\>\>\>\>\>password: {\it text}\\
\>\>\>\>\>\>\>\>start-time: {\it text(``YYYY-MM-DD.HH:MM'')}\\
\>\>\>\>\>\>\>\>end-time: {\it text(``YYYY-MM-DD.HH:MM'')}\\
\>\>\>\>\>\>\>\>max-time-drift: {\it uint(0..65535)}\\
\>\>\>\>\>\>\>\}\\
\>\>\>\>\>\>\}\\
\\
\>\>\>\>\>\>passive \{\\
\>\>\>\>\>\>\>disable: {\it bool}\\
\>\>\>\>\>\>\>host: {\it bool}\\
\>\>\>\>\>\>\}\\
\\
\>\>\>\>\>\>neighbor  {\it IPv4}\{\\
\>\>\>\>\>\>\>router-id: {\it IPv4}\\
\>\>\>\>\>\>\}\\
\\
\>\>\>\>\>\>disable: {\it bool}\\
\>\>\>\>\>\}\\
\>\>\>\>\}\\
\>\>\>\}\\
\>\>\}\\
\\
\>\>import: {\it text} \\
\>\>export: {\it text}\\
\>\}\\
\}
\end{tabbing}
\end{alltt}
\end{minipage}
}
\vspace{0.1in}

\newpage
The configuration syntax for XORP OSPFv3 is given below.

\vspace{0.1in}
\noindent\framebox[\textwidth][l]{\scriptsize
\begin{minipage}{6in}
\begin{alltt}
\begin{tabbing}
xx\=xx\=xx\=xx\=xx\=xx\=xx\=xx\=xx\=xx\=xx\=xx\=\kill
protocols \{\\
\>ospf6 \{\\
\>\>targetname: {\it text}\\
\>\>router-id: {\it IPv4}\\
\>\>ip-router-alert: {\it bool}\\
\\
\>\>traceoptions \{\\
\>\>\>flag \{\\
\>\>\>\>all \{\\
\>\>\>\>\>disable: {\it bool}\\
\>\>\>\>\}\\
\>\>\>\}\\
\>\>\}\\
\\
\>\>area {\it IPv4} \{\\
\>\>\>area-type: {\it text}\\
\\
\>\>\>default-lsa \{\\
\>\>\>\>disable: {\it bool}\\
\>\>\>\>metric: {\it uint(0..0xffffff)}\\
\>\>\>\}\\
\\
\>\>\>summaries \{\\
\>\>\>\>disable: {\it bool}\\
\>\>\>\}\\
\>\>\>area-range {\it IPv6Net} \{\\
\>\>\>\>advertise: {\it bool}\\
\>\>\>\}\\
\>\>\>virtual-link {\it IPv4} \{\\
\>\>\>\>transit-area: {\it ipv4}\\
\>\>\>\>hello-interval: {\it uint(1..65535)}\\
\>\>\>\>router-dead-interval: {\it uint(1..65535)}\\
\>\>\>\>retransmit-interval: {\it uint(1..65535)}\\
\>\>\>\>transit-delay: {\it uint(0..3600)}\\
\>\>\>\}\\
\\
%{\rm continued overleaf....}
%\end{tabbing}
%\end{alltt}
%\end{minipage}
%}
%\vspace{0.1in}

%\newpage
%\vspace{0.1in}
%\noindent\framebox[\textwidth][l]{\scriptsize
%\begin{minipage}{6in}
%\begin{alltt}
%\begin{tabbing}
%xx\=xx\=xx\=xx\=xx\=xx\=xx\=xx\=xx\=xx\=xx\=xx\=\kill
\>\>\>interface {\it text} \{\\
\>\>\>\>link-type: {\it text}\\
\\
\>\>\>\>vif {\it text} \{\\
\\
\>\>\>\>\>address {\it IPv6} \{\\
\>\>\>\>\>\>disable: {\it bool}\\
\>\>\>\>\>\}\\
\>\>\>\>\>priority: {\it uint(0..255)}\\
\>\>\>\>\>hello-interval: {\it uint(1..65535)}\\
\>\>\>\>\>router-dead-interval: {\it uint(1..4294967295)}\\
\>\>\>\>\>interface-cost: {\it uint(1..65535)}\\
\>\>\>\>\>retransmit-interval: {\it uint(1..65535)}\\
\>\>\>\>\>transit-delay: {\it uint(0..3600)}\\
\>\>\>\>\>passive: {\it text}\\
\>\>\>\>\>neighbor  {\it IPv4}\{\\
\>\>\>\>\>\>router-id: {\it IPv4}\\
\>\>\>\>\>\}\\
\>\>\>\>\>disable: {\it bool}\\
\>\>\>\>\>\}\\
\>\>\>\>\}\\
\>\>\>\}\\
\>\>\}\\
\\
\>\>import: {\it text} \\
\>\>export: {\it text}\\
\>\}\\
\}
\end{tabbing}
\end{alltt}
\end{minipage}
}
\vspace{0.1in}

\noindent
The OSPFv2 and OSPFv3 configurations are practically equivalent with
the following exceptions:
\begin{itemize}
\item OSPFv3 does not support authentication in the protocol itself.
\item OSPFv2 supports a single address per interface/vif, therefore all
parameters are set below the address node.
\item OSPFv2 before release 1.5 the passive keyword was a bool, it is
now a directive, previously this would set the interface into lookback
and advertise a host route. Since release 1.5 the default behaviour is
to advertise the network not the host route.
\item OSPFv2 supports a configuration option {\tt
rfc1583-compatibility}, which changes the preference rules when
choosing among multiple AS-external-LSAs advertising the same destination.
\item OSPFv3 supports multiple addresses per interface/vif, therefore
parameters are set below the vif node.
\item OSPFv3 does not require any addresses to be configured in which
case it will advertise all global addresses configured on the
interface/vif. If any addresses are configured in OSPFv3 then only
those addresses will be advertised. If it is required that no
addresses should be advertised then configuring an address and
disabling it, will stop any global addresses being advertised.
\item OSPFv3 in release 1.4 required that a link local address is
configured on any interface/vif on which it is configured to run.
\end{itemize}

The configuration parameters are used as follows:
\begin{description}
\item{\tt protocols}: This delimits the configuration for all routing
  protocols in the XORP router configuration.  It is mandatory that
  OSPF configuration is under the {\stt protocols} node in the
  configuration.
\item{\tt ospf4}: This delimits the OSPFv2 configuration part of the XORP
  router configuration.
\item{\tt ospf6}: This delimits the OSPFv3 configuration part of the XORP
  router configuration.
\item{\tt targetname}: This is the name for this instance of OSPF.  It
  defaults to ``{\stt ospfv2}'', and it is not recommended that this
  default is overridden under normal usage scenarios.
\item{\tt router-id}: This is a unique four octet ID within the
Autonomous System. The numerically smallest IP address of an interface
belonging to the router is a good choice. The required format of the
{\stt router-id} is a dotted-decimal IPv4 address.
\item{\tt ip-router-alert}: This takes the value {\stt true} or {\stt
false}. The default state is {\stt false}, if set to {\stt true} the
IP router alert option will be placed in all transmitted packets.
\item{\tt traceoptions}: This directive if present will enable all tracing.
\item{\tt area}: This delimits an area in which multiple virtual links
and interfaces can be configured. The {\stt area} directive take an
area identifier parameter, which by convention is specified as a
dotted-decimal IPv4 address.
\begin{description}

\item{\tt area-type}: This is the type of the area, {\stt normal},
{\stt stub} or {\stt nssa}.

\item{\tt default-lsa}: This directive if present is affective for
{\stt stub} or {\stt nssa} areas only. If the router is an Area Border
Router, then a default-lsa will be introduced into this area.
\begin{description}
\item{\tt metric}: This is the metric in the default-lsa.
\item{\tt disable}: This takes the value {\stt true} or {\stt
false}. The default setting is {\stt false} it can be set to {\stt
true} to disable the sending of the default-lsa.
\end{description}
\end{description}
\begin{description}

\item{\tt summaries}: This directive if present is affective for
{\stt stub} or {\stt nssa} areas only. If the router is an Area Border
Router, then this option controls the introduction of Summary-LSAs into
the area.
\begin{description}
\item{\tt disable}: This takes the value {\stt true} or {\stt
false}. The default setting is {\stt false} it can be set to {\stt
true} to disable the sending of Summary-LSAs.
\end{description}
\end{description}

\begin{description}
\item{\tt area-range}: If the router is an Area Border Router the IPv4
network defines how to summarize this area into other areas.
\begin{description}
\item{\tt advertise}: This takes the value {\stt true} or {\stt
false}. The default setting is {\stt true} it can be set to {\stt
false} to disable the sending of Summary-LSAs. 
\end{description}
\end{description}

\begin{description}
\item{\tt virtual-link}: This is the {\stt router-id} of the router
with which a virtual link should be formed. Virtual links can only be
configured in the {\stt BACKBONE} area. The format of the parameter is
a dotted-decimal IPv4 address.
\begin{description}
\item{\tt transit-area}: This is the transit area through which the
virtual link is formed.
\item{\tt hello-interval}: This is the time in seconds between sending
hello packets.
\item{\tt router-dead-interval}: This is the time in seconds to wait before
considering a neighbor dead. If no hello packets are seen from the
neighbor in this time then it is considered dead.
\item{\tt retransmit-interval}: This is the time in seconds between
retransmitting various packets, such as link state update packets or
link state request packets.

\item{\tt authentication}: This directive specifies the authentication
mechanism used to authorise OSPF updates sent and received via this
vif/address.  

The authentication is configured by using one of the following
mutually-exclusive statements:

\begin{description}
\item{\tt simple-password}: this specifies the password used for
  plaintext authentication on this vif/address.
\item{\tt md5}: this specifies an MD5 authentication key. The parameter
  is the key ID and must be in the interval [0, 255].
  The MD5 authentication is configured by using the following
  statements:
\begin{description}
  \item{\tt password}: this specifies the MD5 password for the specific
  key.
  \item{\tt start-time}: this specifies the start time when the key
  becomes active. The format is ``YYYY-MM-DD.HH:MM''. If it is empty,
  then the key should become active immediately.
  \item{\tt end-time}: this specifies the end time when the key becomes
  inactive. The format is ``YYYY-MM-DD.HH:MM''. If it is empty,
  then the key should never expire.
  \item{\tt max-time-drift}: this specifies the maximum time drift
  (in seconds) among all OSPF routers.  The allowed values are in the
  interval [0, 65535]. If the value is 65535, the time drift is
  unlimited. The purpose of this statement is to decide when to start
  accepting the MD5 keys in case other routers's clocks are not
  syncronized and  have started to generate messages with a particular
  key:

\begin{description}
  \item KeyStartAccept = KeyStartGenerate - MaxTimeDrift
  \item KeyStopAccept  = KeyStopGenerate  + MaxTimeDrift
\end{description}

\end{description}

If there are multiple configured keys, among the keys that are valid
for message generation, the one with the most recent {\stt start-time}
(and the largest key ID as a tie breaker) would be used to generate the
messages.

\end{description}
\end{description}

\item{\tt interface}: This specifies a network interface that should
  be used by OSPF for routing.  See Chapter \ref{interfaces} for
  details of interfaces.  The interface must be configured
  in the {\stt interfaces} part of the router configuration.
\begin{description}
\item{\tt link-type}: This specifies the type of the link, {\stt
broadcast}, {\stt p2p} (Point-to-Point) or {\stt p2m} (Point-to-Multipoint).
\begin{description}
\item{\tt vif}: This specifies a vif that should be used by OSPF for routing.
  See Chapter \ref{interfaces} for details of vifs.
\begin{description}
\item{\tt address}: This specifies an IPv4 address that should be used
  by OSPF for routing.  OSPF will peer with other routers on this {\stt
  interface/vif} using this {\stt address}.  The address must be a
  valid configured address for this vif.

  The parameters that can be specified for each address are:
\begin{description}
\item{\tt priority}: This is the priority used to select the
Designated Router on a {\stt broadcast} or {\stt nbma} {\stt
link-type}. The priorities range from 0 to 255. If a value of 0 is
choosen this router will not be a candidate to become the Designated Router
\item{\tt hello-interval}: This is the time in seconds between sending
hello packets.
\item{\tt router-dead-interval}: This is the time in seconds to wait before
considering a neighbor dead. If no hello packets are seen from the
neighbor in this time then it is considered dead.
\item{\tt interface-cost}: The cost for this address that is placed in
the Router-LSA.
\item{\tt retransmit-interval}:  This is the time in seconds between
retransmitting various packets, such as link state update packets or
link state request packets.
\item{\tt transit-delay}: The time to transmit an LSA on this address,
this value is added to the age field of all LSAs.

\item{\tt authentication}: This directive specifies the authentication
mechanism used to authorise OSPF updates sent and received via this
vif/address.  

The authentication is configured by using same configuration statements
as those in case of virtual links (see the {\tt virtual-link}
configuration statement above).

%% The authentication is configured by using one of the following
%% mutually-exclusive statements:

%% \begin{description}
%% \item{\tt simple-password}: this specifies the password used for
%%   plaintext authentication on this vif/address.
%% \item{\tt md5}: this specifies an MD5 authentication key. The parameter
%%   is the key ID and must be in the interval [0, 255].
%%   The MD5 authentication is configured by using the following
%%   statements:
%% \begin{description}
%%   \item{\tt password}: this specifies the MD5 password for the specific
%%   key.
%%   \item{\tt start-time}: this specifies the start time when the key
%%   becomes active. The format is ``YYYY-MM-DD.HH:MM''. If it is empty,
%%   then the key should become active immediately.
%%   \item{\tt end-time}: this specifies the end time when the key becomes
%%   inactive. The format is ``YYYY-MM-DD.HH:MM''. If it is empty,
%%   then the key should never expire.
%%   \item{\tt max-time-drift}: this specifies the maximum time drift
%%   (in seconds) among all OSPF routers.  The allowed values are in the
%%   interval [0, 65535]. If the value is 65535, the time drift is
%%   unlimited. The purpose of this statement is to decide when to start
%%   accepting the MD5 keys in case other routers's clocks are not
%%   syncronized and  have started to generate messages with a particular
%%   key:

%% \begin{description}
%%   \item KeyStartAccept = KeyStartGenerate - MaxTimeDrift
%%   \item KeyStopAccept  = KeyStopGenerate  + MaxTimeDrift
%% \end{description}

%% \end{description}
%% \end{description}
\item{\tt passive}: This directive if present enables loopback mode
(appeared in release 1.5). Can be explicitly disabled by setting {\tt
disable} to {\tt true}. The default is to send a network not a host
route, setting {\tt host} to {\tt true} will send a host route to
reproduce the pre 1.5 release behaviour.
% Nesting too deep:
%\begin{description}
%\item{\tt disable}: This takes the value {\stt true} or {\stt
%false}. The default setting is {\stt false} it can be set to {\stt
%true} to disable lookback mode.
%\item{\tt host}: This takes the value {\stt true} or {\stt
%false}. The default setting is {\stt false} it can be set to {\stt
%true} to announce a host route instead of the default network.
%\end{description}

\item{\tt neighbor}: This allows neighbors to be configured for {\stt
link-type}s of {\stt p2p} or {\stt p2m}. The parameter is the IPv4
address of the neighbor. The {\stt router-id} of the neighbor must
also be configured.
%% \begin{description}
%% \item{\tt router-id}:
%% \end{description}
\item{\tt disable}:  This takes the value {\stt true} or {\stt
false}. The default setting is {\stt false} it can be set to {\stt
true} to disable OSPF on this address without removing all the configuration.
\end{description}
\end{description}
\end{description}
\end{description}

\end{description}
\end{description}

\subsection{Example Configurations}

\vspace{0.1in}
\noindent\framebox[\textwidth][l]{\scriptsize
\begin{minipage}{6in}
\begin{alltt}
\begin{tabbing}
xx\=xx\=xx\=xx\=xx\=\kill
protocols \{\\
\>ospf4 \{\\
\>\>router-id: 10.10.10.10\\
\\
\>\>area 0.0.0.0 \{\\
\>\>\>interface dc0 \{\\
\>\>\>\>vif dc0 \{\\
\>\>\>\>\>address 10.10.10.10 \{\\
\>\>\>\>\>\}\\
\>\>\>\>\}\\
\>\>\>\}\\
\>\>\}\\
\>\}\\
\}\\
\end{tabbing}
\end{alltt}
\end{minipage}
}

This configuration is an example of the minimal possible
configuration. OSPFv2 is running in the {\tt BACKBONE} area, on a single
interface/vif the {\tt router-id} is set to the interface/vif address.

\vspace{0.1in}
\noindent\framebox[\textwidth][l]{\scriptsize
\begin{minipage}{6in}
\begin{alltt}
\begin{tabbing}
xx\=xx\=xx\=xx\=xx\=\kill
protocols \{\\
\>ospf6 \{\\
\>\>router-id: 10.10.10.10\\
\\
\>\>area 0.0.0.0 \{\\
\>\>\>interface dc0 \{\\
\>\>\>\>vif dc0 \{\\
\>\>\>\>\}\\
\>\>\>\}\\
\>\>\}\\
\>\}\\
\}\\
\end{tabbing}
\end{alltt}
\end{minipage}
}

This configuration is an example of the minimal possible
configuration. OSPFv3 is running in the {\tt BACKBONE} area, on a single
interface/vif the {\tt router-id} is set to an an IPv4 address owned
by the router.

\newpage
\section{Clearing OSPF database}

It may be necessary to drop all adjacencies and clear the OSPF
database. After the clear command is run the OSPF database will only
contain the LSAs of the router itself and all adjacencies will have been
removed.

\vspace{0.1in}
\noindent\framebox[\textwidth][l]{\scriptsize
\begin{minipage}{6in}
\begin{alltt}
\begin{tabbing}
xx\=xxxxxxxxxxxxxxxxxxxxx\=\kill
user@hostname> \textbf{clear ospf4 database}\\
Or:\\
user@hostname> \textbf{clear ospf6 database}\\
\end{tabbing}
\end{alltt}
\end{minipage}
}

\newpage
\section{Monitoring OSPF}

On a router running OSPF, the OSPF routing state can be displayed
using the {\stt show ospf4} or {\stt show ospf6} operational-mode
command. Examples are given below of ospf4 commands equivalent ospf6
commands are available. Information is available about the per area
LSA databases and the status of OSPF adjacencies.

As always, command completion using $<$TAB$>$ or ? will display the
available sub-commands and parameters:

\vspace{0.1in}
\noindent\framebox[\textwidth][l]{\scriptsize
\begin{minipage}{6in}
\begin{alltt}
\begin{tabbing}
xx\=xxxxxxxxxxxxxxxxxxxxx\=\kill
user@hostname> \textbf{show ospf4 ?}\\
Possible completions:\\
\>database\>Show LSA database\\
\>neighbor\>Show Neighbors\\
\>|\>Pipe through a command\\
\end{tabbing}
\end{alltt}
\end{minipage}
}

The {\stt show ospf4 database} command will display information about the
LSAs in the database. Many optional parameters are available to narrow
the search to specific LSA type or to an area. The optional parameter
{\stt detail} will print more information about a LSA. The optional
parameter {\stt summary} prints a count of the LSAs.

\vspace{0.1in}
\noindent\framebox[\textwidth][l]{\scriptsize
\begin{minipage}{6in}
\begin{alltt}
\begin{tabbing}
xx\=xxxxxxxxxxxxxxxxxxxxx\=\kill
user@hostname> \textbf{show ospf4 database ?}\\
Possible completions:\\
\><[Enter]>\>Execute this command\\
\>  area            \>Show LSA database\\
\>  asbrsummary     \>Show Summary-LSA (AS boundary router) database\\
\>  brief           \>Display brief output (default)\\
\>  detail          \>Display detailed output\\
\>  external        \>Show External-LSA database\\
\>  netsummary      \>Show Summary-LSA (network) database\\
\>  network         \>Show Network-LSA database\\
\>  nssa            \>Show NSSA-LSA database\\
\>  router          \>Show Router-LSA database\\
\>  summary         \>Display summary output\\
\>|\>Pipe through a command\\
\end{tabbing}
\end{alltt}
\end{minipage}
}

\newpage
The {\stt show ospf4 database} command with no modifiers will show all
LSA types in all areas:

\vspace{0.1in}
\noindent\framebox[\textwidth][l]{\scriptsize
\begin{minipage}{6in}
\begin{alltt}
\begin{tabbing}
%%\=xxxxxxxxx\=xxxxxxxxxxxxxxxxxx\=xxxxxxxxxxxxxxxxxx\=xxxxxxxx\=xxxxxxxx\=xxxxxxxx\=xxxxxxxx\=xxxxxxxx\=xxxxxxxx\=\kill
user@hostname> \textbf{show ospf4 database}\\
   OSPF link state database, Area 0.0.0.0\\
\=xxxxxxxxxxxx\=xxxxxxxxxxxxxxxx\=xxxxxxxxxxxxxxxx\=xxxxxxxx\=xxxxx\=xxxxx\=xxxxxxx\=\kill
\>Type\>       ID\>               Adv Rtr\>           Seq\>      Age\>  Opt\>  Cksum\>  Len\\
\=xxxxxxxx\=x\=xxxxxxxxxxxxxxxx\=xxxxxxxxxxxxxxxx\=xxxxxxxxxxx\=xxxxx\=xxxxx\=xxxxxxx\=\kill
\>Router\>*\>  192.150.187.112\>192.150.187.112\>0x8000018f\>1444\>0x2\>  0x9d24\>          36\\
\>Network\>\>  192.150.187.99\>   192.150.187.99\>   0x80000054\>   167\>  0x22\> 0x7c63\>  40\\
\>SummaryN\>*\>172.16.0.0\>       192.150.187.112\>  0x80000188\>  1453\>  0x2\>  0x9df5\>  28\\
\>SummaryN\>*\>172.16.1.0\>       192.150.187.112\>  0x80000188\>  1453\>  0x2\>  0x92ff\>  28\\
\>SummaryN\>*\>172.16.2.0\>       192.150.187.112\>  0x80000188\>  1453\>  0x2\>  0x870a\>  28\\
\>Router\>\>   192.150.187.99\>   192.150.187.99\>   0x80000e76\>   167\>  0x22\> 0xddf7\>  36\\
\>Router\>\>   192.150.187.5\>    192.150.187.5\>    0x800004a6\>   983\>  0x2\>  0x40a8\>  36\\
\>Router\>\>   192.150.187.108\>  192.150.187.108\>  0x80000301\>   598\>  0x2\>  0xb9a1\>  36\\
\>ASExt-2\>\>  172.16.0.0\>       192.150.187.5\>    0x800003a9\>   990\>  0x2\>  0x77c1\>  36\\
   OSPF link state database, Area 0.0.0.13\\
\=xxxxxxxxxxxx\=xxxxxxxxxxxxxxxx\=xxxxxxxxxxxxxxxx\=xxxxxxxx\=xxxxx\=xxxxx\=xxxxxxx\=\kill
\>Type\>       ID\>               Adv Rtr\>           Seq\>      Age\>  Opt\>  Cksum\>  Len\\
\=xxxxxxxx\=x\=xxxxxxxxxxxxxxxx\=xxxxxxxxxxxxxxxx\=xxxxxxxxxxx\=xxxxx\=xxxxx\=xxxxxxx\=\kill
\>Router\>*\>  192.150.187.112\>  192.150.187.112\>  0x80000001\>  1454\>  0x2\>  0x8a03\>  36\\
\>SummaryN\>*\>192.150.187.0\>    192.150.187.112\>  0x80000001\>  1443\>  0x2\>  0x4ef6\>  28\\
\>SummaryN\>*\>172.16.1.0\>       192.150.187.112\>  0x80000001\>  1453\>  0x2\>  0xa476\>  28\\
\>SummaryN\>*\>172.16.2.0\>       192.150.187.112\>  0x80000001\>  1453\>  0x2\>  0x9980\>  28\\
\>SummaryR\>*\>192.150.187.5\>    192.150.187.112\>  0x80000001\>  1443\>  0x2\>  0xbb4\>   28\\
\>ASExt-2\>\>  172.16.0.0\>       192.150.187.5\>    0x800003a9\>   990\>  0x2\>  0x77c1\>  36\\
   OSPF link state database, Area 0.0.0.2\\
\=xxxxxxxxxxxx\=xxxxxxxxxxxxxxxx\=xxxxxxxxxxxxxxxx\=xxxxxxxx\=xxxxx\=xxxxx\=xxxxxxx\=\kill
\>Type\>       ID\>               Adv Rtr\>           Seq\>      Age\>  Opt\>  Cksum\>  Len\\
\=xxxxxxxx\=x\=xxxxxxxxxxxxxxxx\=xxxxxxxxxxxxxxxx\=xxxxxxxxxxx\=xxxxx\=xxxxx\=xxxxxxx\=\kill
\>Router\>*\>  192.150.187.112\>  192.150.187.112\>  0x80000001\>  1454\>  0x2\>  0x93f8\>  36\\
\>SummaryN\>*\>192.150.187.0\>    192.150.187.112\>  0x80000001\>  1443\>  0x2\>  0x4ef6\>  28\\
\>SummaryN\>*\>172.16.0.0\>       192.150.187.112\>  0x80000001\>  1453\>  0x2\>  0xaf6c\>  28\\
\>SummaryN\>*\>172.16.2.0\>       192.150.187.112\>  0x80000001\>  1453\>  0x2\>  0x9980\>  28\\
\>SummaryR\>*\>192.150.187.5\>    192.150.187.112\>  0x80000001\>  1443\>  0x2\>  0xbb4\>   28\\
\>ASExt-2\>\>  172.16.0.0\>       192.150.187.5\>    0x800003a9\>   990\>  0x2\>  0x77c1\>  36\\
   OSPF link state database, Area 0.0.0.3\\
\=xxxxxxxxxxxx\=xxxxxxxxxxxxxxxx\=xxxxxxxxxxxxxxxx\=xxxxxxxx\=xxxxx\=xxxxx\=xxxxxxx\=\kill
\>Type\>       ID\>               Adv Rtr\>           Seq\>      Age\>  Opt\>  Cksum\>  Len\\
\=xxxxxxxx\=x\=xxxxxxxxxxxxxxxx\=xxxxxxxxxxxxxxxx\=xxxxxxxxxxx\=xxxxx\=xxxxx\=xxxxxxx\=\kill
\>Router\>*\>  192.150.187.112\>  192.150.187.112\>  0x80000001\>  1454\>  0x2\>  0x9cee\>  36\\
\>SummaryN\>*\>192.150.187.0\>    192.150.187.112\>  0x80000001\>  1443\>  0x2\>  0x4ef6\>  28\\
\>SummaryN\>*\>172.16.0.0\>       192.150.187.112\>  0x80000001\>  1453\>  0x2\>  0xaf6c\>  28\\
\>SummaryN\>*\>172.16.1.0\>       192.150.187.112\>  0x80000001\>  1453\>  0x2\>  0xa476\>  28\\
\>SummaryR\>*\>192.150.187.5\>    192.150.187.112\>  0x80000001\>  1443\>  0x2\>  0xbb4\>   28\\
\>ASExt-2\>\>  172.16.0.0\>       192.150.187.5\>    0x800003a9\>   990\>  0x2\>  0x77c1\>  36\\
\end{tabbing}
\end{alltt}
\end{minipage}
}

The {\stt show ospf4 neighbor} command will show state of adjacencies:

\vspace{0.1in}
\noindent\framebox[\textwidth][l]{\scriptsize
\begin{minipage}{6in}
\begin{alltt}
\begin{tabbing}
user@hostname> show ospf4 neighbor\\
\=xxxxxxxxxxxxxxxx\=xxxxxxxxxxxxxxxxxxxx\=xxxxxxxxx\=xxxxxxxxxxxxxxxx\=xxxx\=\kill
\>  Address\>         Interface\>             State\>      ID\>              Pri\>  Dead\\
\>192.150.187.5\>    fxp0/fxp0\>              Full\>      192.150.187.5\>    150\>    34\\
\>192.150.187.99\>   fxp0/fxp0\>              Full\>      192.150.187.99\>   128\>    34\\
\>192.150.187.108\>  fxp0/fxp0\>              TwoWay\>    192.150.187.108\>  128\>    34\\
\>192.150.187.78\>   fxp0/fxp0\>              Down\>      0.0.0.78\>           0\>     0\\
\end{tabbing}
\end{alltt}
\end{minipage}
}

\newpage
The {\stt show ospf4 neighbor detail} command will show state of
adjacencies with extra detail such as the Designated Router and the
time the adjacency has been up:

\vspace{0.1in}
\noindent\framebox[\textwidth][l]{\scriptsize
\begin{minipage}{6in}
\begin{alltt}
\begin{tabbing}
user@hostname> show ospf4 neighbor detail\\
\=xxxxxxxxxxxxxxxx\=xxxxxxxxxxxxxxxxxxxx\=xxxxxxxxx\=xxxxxxxxxxxxxxxx\=xxxx\=\kill
\>  Address\>         Interface\>             State\>      ID\>              Pri\>  Dead\\
\>192.150.187.5\>    fxp0/fxp0\>              Full\>      192.150.187.5\>    150\>    33\\
\=xxxxxxxxxxxxxxxxxx\=xxxxxxxxx\=xxxxxxxxxxxxxxxxxxx\=\kill
\>  Area 0.0.0.0,\> opt 0x2,\> DR 192.150.187.99,\> BDR 192.150.187.5\\
\>  Up 54:09:26,\> adjacent 54:09:16\\
\=xxxxxxxxxxxxxxxx\=xxxxxxxxxxxxxxxxxxxx\=xxxxxxxxx\=xxxxxxxxxxxxxxxx\=xxxx\=\kill
\>192.150.187.99\>   fxp0/fxp0\>              Full\>      192.150.187.99\>   128\>    38\\
\=xxxxxxxxxxxxxxxxxx\=xxxxxxxxx\=xxxxxxxxxxxxxxxxxxx\=\kill
\>  Area 0.0.0.0,\> opt 0x2,\> DR 192.150.187.99,\> BDR 192.150.187.5\\
\>  Up 54:09:26,\> adjacent 54:09:16\\
\=xxxxxxxxxxxxxxxx\=xxxxxxxxxxxxxxxxxxxx\=xxxxxxxxx\=xxxxxxxxxxxxxxxx\=xxxx\=\kill
\>192.150.187.108\>  fxp0/fxp0\>              TwoWay\>    192.150.187.108\>  128\>    33\\
\=xxxxxxxxxxxxxxxxxx\=xxxxxxxxx\=xxxxxxxxxxxxxxxxxxx\=\kill
\>  Area 0.0.0.0,\> opt 0x2,\> DR 192.150.187.99,\> BDR 192.150.187.5\\
\>  Up 54:09:26\\
\=xxxxxxxxxxxxxxxx\=xxxxxxxxxxxxxxxxxxxx\=xxxxxxxxx\=xxxxxxxxxxxxxxxx\=xxxx\=\kill
\>192.150.187.78\>   fxp0/fxp0\>              Down\>      0.0.0.78\>           0\>    0\\
\=xxxxxxxxxxxxxxxxxx\=xxxxxxxxx\=xxxxxxxxxxxxxxxxxxx\=\kill
\>  Area 0.0.0.0,\> opt 0,\> DR 0.0.0.0,\> BDR 0.0.0.0\\
\>  Up 14:47:32\\
\end{tabbing}
\end{alltt}
\end{minipage}
}
