\documentclass[letterpaper]{article}
\usepackage{supertabular}
\usepackage[left=1in,top=1in,bottom=2in,right=1in]{geometry}

\author{Andrea Bittau}
\title{XORP Policy Tasks}
\date{June 18, 2008}

\begin{document}
\maketitle

\section{Introduction}
This document lists tasks that need to be completed in order to bring XORP's
policy framework to a more mature state.  Both functional aspects, such as the
completeness of the framework compared to Juniper's and Cisco's, and non
functional aspects ({\em i.e.,} performance) are enumerated.  They are listed in
the following order / categories:
\begin{enumerate}
\item Route matching.  What are the criteria for identifying particular routes
in policies?

\item Route actions.  Once a route is matched, what actions and modifications
can be done on it?

\item Other policy aspects.  Miscellaneous functionality such as how policies
are attached to protocols and so on.

\item Performance.  What can be done to improve XORP's performance with regards
to policy?
\end{enumerate}

An attempt has been made to capture Juniper's functionality exhaustively.
Regarding Cisco, only a best effort has been made to map its functionality to
that of Juniper, and no real attempt has been made to verify whether Cisco
supports functionality that Juniper does not.  It probably is the case though
that all that can be done with Cisco can also be done with Juniper, and hence
using Juniper as a reference for XORP will most likely yield a satisfactory
policy implementation.

The indication of working days needed per task is very approximate.  Integral
numbers are used throughout and many one day tasks indeed can take less time,
especially when tasks are correlated or similar in nature.  If all tasks need to
be completed, the total day count is probably over-estimated by 50\%.  However,
individual tasks longer than a day have a more optimistic rather than
conservative estimate.  Once the necessary tasks for completion have been
selected, a feasibility study with a more accurate day count will follow.  Tasks
without a day count are already completed, and for those with question marks it
is uncertain what the requirement actually is and whether it is of interest for
XORP in the near future.

\section{Route matching}
\begin{center}
\tablehead{
\hline
Ref.	& Description	& Juniper	& Cisco	& XORP	& Days \\
}
\tabletail{
\hline
}
\begin{supertabular}{|r |p{6cm} |p{2cm} |p{2cm} |p{2cm} |r|}
\hline

100 &
Aggregate contributor.  Match routes that are contributing to a configured
aggregate. This match condition can be used to suppress a contributor in an
aggregate route.
& aggregate-contributor & ?? & ?? & ?? \\
\hline

101 &
OSPF area identifier.  Match a route learned from the specified OSPF area.
& area & match route-type? &  & 1 \\
\hline

102 &
BGP AS path regular expression.
& as-path & match as-path & as-path &\\
\hline

103 &
BGP AS path group regular expression.
& as-path-group & match as-path &  & 1 \\
\hline

104 &
Color values.  The color value can be a number in the range from 0 through
4,294,967,295. A lower number indicates a more preferred route.
& color, color2 & match metric? &  & 1 \\
\hline

105 &
Community.  Name of one or more communities. If you list more than one name,
only one name needs to match for a match to occur (the matching is effectively a
logical OR operation).
& community & match community & community &\\
\hline

106 &
OSPF external.  Match external routes, including routes exported from one level
to another.  When you specify type, this condition matches only OSPF routes with
the specified OSPF metric type.
& external $[$type$]$ & match route-type & ebit? &\\
\hline

107 &
IP family.  Match the address family IP version 4 (IPv4) or IP version 6 (IPv6)
of the route. 
& family & match ip-address? & network4/6 &\\
\hline

108 &
Instance.  Match a route learned from one of the specified instances.
& instance & ?? & ?? & ?? \\
\hline

109 &
Interface.  Match a route learned from one of the specified interfaces (IP
addresses or names).
& interface & match interface &  & 2 \\
\hline

110 &
Internal.  Match a routing policy against the internal flag for simplified
next-hop self policies.
& internal & match route-type &  & 1 \\
\hline

111 &
IS-IS level.  Match a route learned from a specified level.
& level & match route-type & no IS-IS in XORP & ?? \\
\hline

112 &
BGP local preference.  BGP local preference (LOCAL\_PREF) attribute. 
& local-preference & match local-preference & localpref &\\
\hline

113 &
Metric values (1--4).  (BGP only) metric corresponds to the multiple exit
discriminator (MED), and metric2 corresponds to the interior gateway protocol
(IGP) metric if the BGP next hop runs back through another route.
& metric$[$1--4$]$ & match metric & med/metric &\\
\hline

114 &
Multicast scope value of IPv4 or IPv6 multicast group address. The
multicast-scoping name corresponds to an IPv4 prefix. You can match on a
specific multicast-scoping prefix or on a range of prefixes.
& multicast-scoping & match ip address? & ?? & ?? \\
\hline

115 &
Neighbor addresses.  For BGP, the address can be a directly connected or
indirectly connected peer.  For all other protocols, the address is the neighbor
from which the advertisement is received.
& neighbor & match ip route-source & neighbor (BGP-only) & 2 \\
\hline

116 &
Next-hop address or addresses specified in the routing information for a
particular route. For BGP routes, matches are performed against each protocol
next hop.
& next-hop & match next-hop & nexthop4/6 &\\
\hline

117 &
LDP generates next hop based on RSVP and IP next hops available to use combined
with the forwarding-class mapping.
& next-hop-type & ?? & ?? & ?? \\
\hline

118 &
BGP origin attribute.  Can be EGP, IGP, or incomplete.
& origin & match route-type? & origin &\\
\hline

119 &
Policy to evaluate as a subroutine.
& policy & match policy-list &  & 3 \\
\hline

120 &
Preference value. You can specify a primary preference value (preference) and a
secondary preference value (preference2). The preference value can be a number
from 0 through 4,294,967,295 (232 – 1). A lower number indicates a more
preferred route.
& preference$[$1--2$]$ & match metric &  & 1 \\
\hline

121 &
Named list of IP addresses. You can specify an exact match with incoming routes.
& prefix-list & match ip address & network / prefix-length &\\
\hline

122 &
Named prefix list. You can specify prefix length qualifiers for the list of
prefixes in the prefix list.
& prefix-list-filter & match ip address & network-list &\\
\hline

123 &
Name of the protocol from which the route was learned or to which the route is
being advertised. It can be one of the following: access, access-internal,
aggregate, bgp, direct, dvmrp, isis, local, ospf, pim-dense, pim-sparse, rip,
ripng, or static.
& protocol & match source-protocol & protocol &\\
\hline

124 &
Name of a routing table.  Can be unicast/multicast IPv4/6, MPLS, particular
unicast IPv4 routing instance.
& rib & ?? &  & 1 \\
\hline

125 &
List of destination prefixes. When specifying a destination prefix, you can
specify an exact match with a specific route or a less precise match using match
types. 
& route-filter & match ip address & network / prefix-length &\\
\hline

126 &
Type of route (internal or external).
& route-type & match route-type &  & 1 \\
\hline

127 &
List of multicast source addresses. When specifying a source address, you can
specify an exact match with a specific route or a less precise match using match
types.
& source-address-filter & match ip address? & ?? & ?? \\
\hline

128 &
Tag value. You can specify two tag strings: tag (for the first string) and
tag2. These values are local to the router and can be set on configured routes
or by using an import routing policy.  You can specify multiple tags under one
match condition by including the tags within a bracketed list.  For OSPF and
IS-IS, the tag and tag2 match conditions match the 32-bit tag field in external
link-state advertisement (LSA) packets. 
& tag$[$2$]$ & match tag &  & 2 \\
\hline

129 &
To base policy routing on the Level 3 length of a packet, use the match length
command in route-map configuration mode.
& ?? & match length &  & 3 \\

\end{supertabular}
\end{center}

\section{Route actions}
\begin{center}
\tablehead{
\hline
Ref.	& Description	& Juniper	& Cisco	& XORP	& Days \\
}
\tabletail{
\hline
}
\begin{supertabular}{|r |p{6cm} |p{2cm} |p{2cm} |p{2cm} |r|}
\hline

200 &
Accept the route and propagate it. After a route is accepted, no other
terms in the routing policy and no other routing policies are evaluated.
& accept & permit &  accept &\\
\hline

201 &
Accept and override any action intrinsic to the protocol. This is a
nonterminating policy action.
& default-action-accept & ?? &  & 2 \\
\hline

202 &
Reject the route and do not propagate it. After a route is rejected, no
other terms in the routing policy and no other routing policies are
evaluated.
& reject & deny & reject &\\
\hline

203 &
Reject and override any action intrinsic to the protocol. This is a
nonterminating policy action.
& default-action-reject & ?? &  & 2 \\
\hline

204 &
Skip to and evaluate the next term in the same routing policy. Any accept or
reject action specified in the then statement is skipped. Any actions in the
then statement that manipulate route characteristics are applied to the route.
& next term & ?? &  & 1 \\
\hline

205 &
Skip to and evaluate the next routing policy. Any accept or reject action
specified in the then statement is skipped. Any actions in the then statement
that manipulate route characteristics are applied to the route.
& next policy & ?? &  & 1 \\
\hline

206 &
Affix one or more AS numbers at the beginning of the AS path.
& as-path-prepend & set as-path prepend & as-path-prepend &\\
\hline

207 &
Extract the last AS number in the existing AS path and affix that AS number to
the beginning of the AS path n times, where n is a number from 1 through 32. 
& as-path-expand & ?? & as-path-append &\\
\hline

208 &
Apply the specified class-of-service parameters to routes installed into the
routing table. 
& class & ?? & ?? & ?? \\
\hline

209 &
Set the color.
& color & set tag &  & 1 \\
\hline

210 &
Change the color.
& color add / subtract & ?? &  & 1 \\
\hline

211 &
Add the specified communities to the set of communities in the route. 
& community add & set community additive & community-add &\\
\hline

212 &
Delete the specified communities from the set of communities in the route.
& community delete & set comm-list delete & community-del &\\
\hline

213 &
Replace any communities that were in the route in with the specified
communities. 
& community set & set community & community-set &\\
\hline

214 &
Set CoS-based next-hop map in forwarding table.
& cos-next-hop-map & ?? & ?? & ?? \\
\hline

215 &
Apply the specified route-damping parameters to the route.
& damping & set dampening &  & 5 \\
\hline

216 &
Maintain packet counts for a route passing through your network, based on the
destination address in the packet.
& destination-class & set traffic-index? & ?? & ?? \\
\hline

217 &
Calculate a metric based on the current values of metric and metric2.
& metric & set metric &  & 2 \\
\hline

218 &
Set the external metric type for routes exported by OSPF. 
& external type & set metric-type &  & 1 \\
\hline

219 &
Create the forwarding class that includes packets based on both the destination
address and the source address in the packet. 
& forwarding-class & ?? & ?? & ?? \\
\hline

220 & 
Choose which next hops, among a set of equal LSP next hops, are installed in the
forwarding table. 
& install-nexthop & ?? & ?? & ?? \\
\hline

221 &
Install all next-hop addresses in the forwarding table and have the forwarding
table perform per-packet load balancing.
& load-balance per-packet & ?? & ?? & ?? \\
\hline

222 &
Set the BGP local preference (LOCAL\_PREF) attribute. 
& local-preference & set local-preference & localpref &\\
\hline

223 &
Change the local preference value by the specified amount. 
& local-preference add / subtract & ?? &  & 1 \\
\hline

224 &
Set the metric.
& metric & set metric & metric &\\
\hline

225 &
Change the metric value by the specified amount. 
& metric add / subtract & ?? &  & 1 \\
\hline

226 &
Change the metric (MED) value by the specified negative or positive offset.
& metric igp & set metric-type internal &  (can only set) & 1 \\
\hline

227 &
Set the next hop (address, discard, next-table, peer-address, reject, self).
& nexthop & set next-hop &  (only address) & 4 \\
\hline

228 &
Set the BGP origin attribute.
& origin & set origin & origin &\\
\hline

229 &
Set the preference value. 
& preference & set weight? &  & 3 \\
\hline

230 &
Change the preference value by the specified amount. 
& preference add / subtract & ?? &  & 3 \\
\hline

231 &
Maintain packet counts for a route passing through your network, based on the
source address.
& source-class & set-traffic index? & ?? & ?? \\
\hline

232 &
Set the tag value. 
& tag & set tag & tag &\\
\hline

233 &
Change the tag value by the specified amount.
& tag add / subtract & ?? &  & 1 \\
\hline

234 &
To indicate where to forward packets that pass a match clause of a route map for
policy routing.
& ?? & set interface &  & 4 \\
\hline

235 &
To indicate where to import routes, use the set level command in route-map
configuration mode.
& ?? & set level & ?? & ?? \\

\end{supertabular}
\end{center}

\section{Miscellaneous policy functionality}
\begin{center}
\tablehead{
\hline
Ref.	& Description	& Days \\
}
\tabletail{
\hline
}
\begin{supertabular}{|r |p{12cm} |r|}
\hline

300 &
Named match conditions.  Essentially a macro for match conditions.
& 4 \\
\hline

301 &
Tracing policy execution in a file.
& 3 \\
\hline

302 &
Final action.  In addition to specifying an action using the then statement in a
named term, you can also specify an action using the then statement in an
unnamed term.  (A bit like ``finally'' in exception handling.)
& 2 \\
\hline

303 &
Policy expressions.  Policy expressions give the policy framework software a
different way to evaluate routing policies. A policy expression uses Boolean
logical operators with policies. The logical operators establish rules by which
the policies are evaluated.
& 4 \\
\hline

304 &
Route list actions.  If you specify route lists in the from statement, for each
route in the list, you can specify an action to take on that individual route
directly, without including a then statement.
& 5 \\
\hline

305 &
Routing policies and forwarding table interactions.  Allows you to set ``class
of service'' and do load balancing.
& ?? \\
\hline

306 &
Testing routing policies.  Before applying a routing policy in a live
environment, you can use the test policy command to ensure that the policy
produces the results that you expect. 
& 5 \\
\hline

307 &
Ensuring consistency and compatibility with Juniper.  Ensure that commands
match, {\em e.g.}, localpref $\to$ local-preference.  Ensure that syntax
matches, {\em e.g.,} BGP AS-path regular expressions.  Ensure that outcome of
commands matches that of Juniper.
& 5 \\
\hline

308 &
Per peer BGP policies.  Ability to attach policies to specific BGP peers.
&
3 \\
\hline

309 &
Per group BGP policies.  Attach policies at a group granularity.  XORP currently
does not support groups.
&
?? \\
\hline

310 &
Per area OSPF policies.
&
5 \\
\hline

311 &
Per group RIP policies.  XORP does not support RIP groups.
&
?? \\
\hline

312 &
Aggregate route policies.  XORP does not support route aggregation.
&
?? \\
\hline

313 &
Policies on generated routes.  XORP does not support route generation.
&
?? \\
\hline

314 &
Policies on a routing table / instance (RIB) granularity.
&
5 \\
\hline

315 &
Make the front-end policy configuration syntax more natural.  Currently the
syntax for configuring policies is at times awkward due to limitations in XORP's
configuration parser.
& 10 \\
\hline

\end{supertabular}
\end{center}

\section{Performance}
\begin{center}
\tablehead{
\hline
Ref.	& Description	& Days \\
}
\tabletail{
\hline
}
\begin{supertabular}{|r |p{12cm} |r|}
\hline

400 &
Specialized types and operators.  The policy back-end supports very generic types
({\em e.g.} strings) and operations on these can be slow.  Creating specialized
types and operators will speed up policy execution.
& 4 \\
\hline

401 &
Bytecode / machine representation for running the policy.  Currently policies
are represented as a parse-tree.  It may be more efficient to represent them as
bytecode or indeed native code that can be quickly run.
& 15 \\
\hline

402 &
Avoid route copying.  In some cases ({\em e.g.} BGP), the only way to modify a
route is create a new one and modify it.  Given that policies often modify
routes this incurs both a time and memory overhead.  It may be possible to
reduce this.
& 5 \\
\hline

403 &
General performance optimizations.  Various optimizations such as, where
possible, using directly indexed arrays rather than (say) STL maps.  Any other
general programming tricks.
& 3 \\
\hline

404 &
Profiling and optimizing specific cases.  This will involve taking individual
protocols and studying their overhead at a more microscopic scale in order to
eliminate inefficiencies.  The policy back-end will be scrutinized too.
& 5  \\

\end{supertabular}
\end{center}

\section{Current policy-related bugs}
\begin{center}
\tablehead{
\hline
Bugzilla ID	& Task ref.	& Description	& Days \\
}
\tabletail{
\hline
}
\begin{supertabular}{|r |r |p{10cm} |r|}
\hline

245 &  
& Show policy commands do not exist.
& 4 \\
\hline

252 &  
& Creating different policy lists with same name results in commit
error. 
& 1 \\
\hline

307 &  
& Policy prefix-length4 is a range rather than single instance.
& 1 \\
\hline

312 & 
& Policy node "origin" should use ``igp/egp/incomplete'' values instead of
numbers.
& 1 \\
\hline

357 & 
& policy statement accepts non-integer ``term'' values.
& 1 \\
\hline

373 & 
& Load configuration fails if policy is applied as import or export.
& 1 \\
\hline

567 & 
& The policy tag per route is not reset in case the policies are removed.
& 1 \\
\hline

568 & 314
& The policy framework does not support filtering or selecting MRIB routes.
& 5 \\
\hline

628 & 312
&   	Add a mechanism to export aggregated routes into routing protocols.
& ?? \\
\hline

652 & 
& Implement something along the lines of Cisco RPL (Routing Policy Language).
& 10 \\
\hline

667 & 308
& Implement per-peer BGP import/export policies.
& 3 \\
\hline

674 & 
& Enhance policy network-lists to included prefix length modifiers.
& 2 \\
\hline

699 & 
& Import policy ``reject all'' statement rejects accept statements on export
policy.
& 2 \\

\end{supertabular}
\end{center}

\section{Summary}
\begin{center}
\begin{tabular}{|l |r|}
\hline
Component	& Days \\
\hline
Route match criteria (current completion: 52\%)	& 19 \\
Route actions (current completion: 40\%)	& 34 \\
\hline
Miscellaneous		& 51 \\
Performance		& 32 \\
\hline
Bug fixing		& 24 (+8) \\
\hline
Contingency (10\%)	& 16 \\
\hline
\hline
{\bf Total}		& {\bf 176} \\
\hline

\end{tabular}
\end{center}

\section{New tasks}
\begin{center}
\tablehead{
\hline
Ref.	& Description	& Days \\
}
\tabletail{
\hline
}
\begin{supertabular}{|r |p{12cm} |r|}
\hline

500 &
Design and implement route pushing.  Route pushing ({\em i.e.,} refiltering
routes upon a policy change) is currently implemented in an ad-hoc manner in
each protocol.  This code is very difficult to get right and it indeed causes
many bugs.  It needs to be designed and implemented by the generic policy
framework, and protocols should make use of this single implementation.
& 10 \\
\hline

501 &
Design and refactor the policy front-end.  The policy front-end was never
designed properly but was instead just coded in an ad-hoc manner and very
quickly.  As requirements increased with time, the current functionality of the
policy manager is rather complicated.  It needs to deal with dependencies
between policies, lists and policies, and special cases such as BGP per-peer
policies.  The code is currently unmantinable and will be a major source of bugs
and development slowdown.
& 10 \\

\end{supertabular}
\end{center}

\end{document}
