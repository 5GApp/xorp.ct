%
% $XORP: xorp/docs/libxorp/libxorp_overview.tex,v 1.18 2005/04/06 22:34:18 pavlin Exp $
%

\documentclass[11pt]{article}

%\usepackage[dvips]{changebar}

\usepackage{subfigure}
\usepackage{fullpage}
\usepackage{setspace}
\usepackage{times}
\usepackage{latexsym}
\usepackage{psfig}
\usepackage{graphicx}
\usepackage{xspace}
\usepackage{color}
\usepackage{amsmath}
%\usepackage[dvipdf]{graphics}
%\usepackage[dvips]{graphicx}
%\usepackage{xorp}

\definecolor{gray}{rgb}{0.5,0.5,0.5}
\newcommand{\etc}{\emph{etc.}\xspace}
\newcommand{\ie}{\emph{i.e.,}\xspace}
\newcommand{\eg}{\emph{e.g.,}\xspace}
%\newcommand{\comment}[1]{{\color{gray}[\textsf{#1}]}}
\newcommand{\comment}[1]{}

% Changebar stuff
% \newenvironment{colorcode}{\color{blue}}{}
% \renewcommand{\cbstart}{\begin{colorcode}}
% \renewcommand{\cbend}{\end{colorcode}}

% \pagestyle{empty}

\begin{document}

\title{XORP Libxorp Library Overview \\
\vspace{1ex}
Version 1.1}
\author{ XORP Project					\\
	 International Computer Science Institute	\\
	 Berkeley, CA 94704, USA			\\
         {\it http://www.xorp.org/}			\\
	 {\it feedback@xorp.org}
}
\date{April 12, 2005}

\maketitle

\thispagestyle{empty}


%%%%%%%%%%%%%%%%%%%%%%%%%%%%%%%%%%%%%%%%%%%%%%%%%%%%%%%%%%%%%%%%%%%%%%%
\section{Introduction}

The \emph{libxorp} library contains a set of classes for basic
XORP functionality such as IP addresses and subnets, timers, event
loops, etc. It is used by virtually every other XORP component, and
its main purpose is to simplify the implementation of those components.

%%%%%%%%%%%%%%%%%%%%%%%%%%%%%%%%%%%%%%%%%%%
\subsection{Overview}

Currently, the libxorp library contains the following classes and
components (in alphabetical order):

\begin{itemize}

  \item \emph{asnum.hh: class AsNum}: A class for storing an AS number
  used by protocols such as BGP.

  \item \emph{asyncio.hh: class AsyncFileReader, class AsyncFileWriter}:
  Asynchronous file transfer classes.

  \item \emph{buffer.hh: class Buffer}:  A class for storing buffered
  data.

  \item \emph{buffered\_asyncio.hh: class BufferedAsyncReader}:
  Buffered asynchronous file reader class.

  \item \emph{c\_format.hh: c\_format()}: A macro that creates a C++ string
  from a C-style printf(3)-formatted string.

  \item \emph{callback.hh, callback\_debug.hh, callback\_nodebug.hh,
  safe\_callback\_obj.hh}:
  Callback mechanism.

  \item \emph{clock.hh: class SystemClock}: A class for providing the
  interface to obtain the system clock.

  \item \emph{config\_param.hh: template class ConfigParam}: A class for
  storing a configuration parameter.

  \item \emph{debug.h}: Provides facility for generating debug messages.

  \item \emph{ether\_compat.h}: Ethernet manipulation compatibility functions.

  \item \emph{eventloop.hh: class EventLoop}: Event loop class for
  coordinated operations between timers and I/O operations on file
  descriptors.

  \item \emph{exceptions.hh}: Standard XORP C++ exceptions.

  \item \emph{heap.hh: class Heap}: Provides Heap data structure.

  \item \emph{ipnet.hh, ipv4net.hh, ipv6net.hh, ipvxnet.hh: class
  IPv4Net, class IPv6Net, class IPvXNet}: Implementation of classes for
  basic subnet addresses (for IPv4, IPv6 and dual IPv4/6 address family
  respectively).

  \item \emph{ipv4.hh, ipv6.hh, ipvx.hh: class
  IPv4, class IPv6, class IPvX}: Implementation of classes for
  basic IP addresses (for IPv4, IPv6 and dual IPv4/6 address family
  respectively).

  \item \emph{mac.hh: class Mac, class EtherMac}: Containers for
  MAC types.

  \item \emph{nexthop.hh}: Classes that contain routing next-hop
  information.

  \item \emph{popen.hh}: The interface for the local implementation
  of \emph{popen(2)} and \emph{pclose(2)}.

  \item \emph{profile.hh}: Implementation of a mechanism for event profiling.

  \item \emph{ref\_ptr.hh: template class ref\_ptr}: Reference counted
  pointer class.

  \item \emph{ref\_trie.hh}: Implementation of a trie to support route
  lookups.  Based on trie.hh, but with reference-counted storage
  supporting delayed deletion.

  \item \emph{run\_command.hh: class RunCommand}: A class for running an
  external command.

  \item \emph{safe\_callback\_obj.hh, class CallbackSafeObject}:
  Implementation of a base class for objects that are callback safe.

  \item \emph{selector.hh}: I/O multiplexing interface.

  \item \emph{service.hh}: Provides base for asynchronous service classes.

  \item \emph{status\_codes.h}: Process states status codes used by
  processes when reporting their operational status to the router manager.

  \item \emph{time\_slice.hh: class TimeSlice}: A class for computing
  whether some processing is taking too long.

  \item \emph{timer.hh, class XorpTimer}: XORP timer facility.

  \item \emph{timespent.hh: class TimeSpent}: A class used for debugging
  purpose to find code that has taken too long to execute.

  \item \emph{timeval.hh: class TimeVal}: A class for storing time values
  (similar to \emph{struct timeval}).

  \item \emph{token.hh}: Token related definitions.

  \item \emph{transactions.hh}: Facility for transaction operations.

  \item \emph{trie.hh}: Implementation of a trie to support route
  lookups.

  \item \emph{utility.h}: Contains various mini-utilities
  (mostly compiler-related helpers).

  \item \emph{utils.hh}: Contains various utilities (\eg to delete a
  list or array of pointers and the objects pointed to).

  \item \emph{vif.hh: class Vif, class VifAddr}: Virtual interface and
  virtual interface address classes.

  \item \emph{xlog.h}: Provides facility for log messages generation.

  \item \emph{xorp.h}: The XORP main include file that should be included
  by all XORP C and C++ files.

\end{itemize}

Each of the components is described in
Section~\ref{sec:components_description}.


%%%%%%%%%%%%%%%%%%%%%%%%%%%%%%%%%%%%%%%%%%%%%%%%%%%%%%%%%%%%%%%%%%%%%%%
\section{Components Description}
\label{sec:components_description}

This section contains a brief description of each of the components of
the \emph{libxorp} library. This description is for informative
purpose only. The source code for each component is the ultimate source
for programming reference, and implementation details.

%%%%%%%%%%%%%%%%%%%%%%%%%%%%%%%%%%%%%%%%%%%
\subsection{asnum.hh}

This file contains \emph{class AsNum}: a class for storing an AS number
used by protocols such as BGP.
This class can be used to store an AS number that can be either
16 or 32 bits.  Originally, the AS numbers were defined as 16-bit
unsigned numbers.  Later the ``extended'' AS numbers were introduced,
which are unsigned 32-bit numbers.

%%%%%%%%%%%%%%%%%%%%%%%%%%%%%%%%%%%%%%%%%%%
\subsection{asyncio.hh}
\label{asyncio.hh}

This file contains
asynchronous file transfer classes.  These utilize XORP EventLoop
and its SelectorList to read or write files asynchronously.  The
user creates an AsyncFile\{Reader,Writer\} and adds a buffer for
reading or writing with add\_buffer().  A callback provided with
each buffer is called every time I/O happens on the buffer.
Reading or writing only begins when start() is called, and normally
continues until there are no buffers left.

From the developer's point of view, the following classes are of
interest: \emph{class AsyncFileReader, class AsyncFileWriter}.

%%%%%%%%%%%%%%%%%%%%%%%%%%%%%%%%%%%%%%%%%%%
\subsection{buffer.hh}

This file contains \emph{class Buffer}: a class for conveniently storing
and accessing buffered data.
Currently it has limited applicability.

%%%%%%%%%%%%%%%%%%%%%%%%%%%%%%%%%%%%%%%%%%%
\subsection{buffered\_asyncio.hh}

This file contains \emph{class BufferedAsyncReader}: a class for
buffered asynchronous file reading. Unlike the \emph{class AsyncFileReader}
(see Section~\ref{asyncio.hh}) which delivers data whenever it is available,
the \emph{class BufferedAsyncReader} buffers the data and delivers it
whenever its size has reached a predefined threshold.

%%%%%%%%%%%%%%%%%%%%%%%%%%%%%%%%%%%%%%%%%%%
\subsection{c\_format.hh}

This file contains \emph{c\_format()}: a macro that creates a C++ string
from a C-style printf(3)-formatted string.
It takes the same arguments as printf(3), but \%n is illegal and
will cause abort to be called.

In practice, \emph{c\_format()} is a nasty macro, but by doing this we can
check the compile time arguments are sane and the run time arguments.


%%%%%%%%%%%%%%%%%%%%%%%%%%%%%%%%%%%%%%%%%%%
\subsection{callback.hh, callback\_debug.hh, callback\_nodebug.hh,
 safe\_callback\_obj.hh}

These files contain an implementation of a callback mechanism.
XORP is an asynchronous programming environment and as a result there
are many places where callbacks are useful.  Callbacks are typically
invoked to signify the completion or advancement of an asynchronous
operation.

XORP provides a generic and flexible callback interface that utilizes
overloaded templatized functions for generating callbacks in
conjunction with many small templatized classes. Whilst this makes the
syntax a little unpleasant, it provides a great deal of flexibility.

XorpCallback objects are objects created by the callback()
function which returns a reference pointer to a newly created
callback object.  The callback is invoked by calling the dispatch()
method on that object.

There are two versions of the callback mechanism: debug and non-debug
version. The debug version includes additional information with
each callback (\eg file name and line number where the callback was invoked),
records callback tracing events, etc, but creates additional overhead
to the system. Non-debug callbacks are used by default; the debug
callbacks can be enabled by defining \emph{DEBUG\_CALLBACK}
before including \emph{callback.hh}, or by running
\emph{./configure --enable-callback-debug} before compiling XORP.

For more details on the callback mechanism, and for usage examples, see
the beginning of \emph{callback\_debug.hh} or \emph{callback\_nodebug.hh}.
Note that these files are auto-generated by \emph{callback-gen.py}
(a Python script), therefore they should never be edited.

%%%%%%%%%%%%%%%%%%%%%%%%%%%%%%%%%%%%%%%%%%%
\subsection{clock.hh}

This file contains the implementation of \emph{class SystemClock}
that provides the interface for obtaining the system clock.

%%%%%%%%%%%%%%%%%%%%%%%%%%%%%%%%%%%%%%%%%%%
\subsection{config\_param.hh}

This file contains the implementation of \emph{template class
ConfigParam}: a class for storing a configuration parameter.

This class can be used to store the value of a configuration parameter.
Such parameter has a current and a default value.
The \emph{ConfigParam} class has the facility to add a callback that is
invoked whenever the value of the configuration parameter is changed.

%%%%%%%%%%%%%%%%%%%%%%%%%%%%%%%%%%%%%%%%%%%
\subsection{debug.h}

This file provides facility for debug messages generation.
More specifically, it defines the \verb=debug_msg()=, the macro
responsible for generating debug messages.
It takes the same arguments as printf(3). For example:

\begin{verbatim}
debug_msg("The number is %d\n", 5);
\end{verbatim}

For more details see the comments inside that file.

%%%%%%%%%%%%%%%%%%%%%%%%%%%%%%%%%%%%%%%%%%%
\subsection{ether\_compat.h}

This file contains Ethernet-related manipulation compatibility
functions. For example, it includes the appropriate system files,
and declares functions \verb=ether_aton()= and \verb=ether_ntoa()=
(implemented locally in \emph{ether\_compat.c}) if the system is missing
the corresponding \verb=ether_aton(3)= and \verb=ether_ntoa(3)=.

%%%%%%%%%%%%%%%%%%%%%%%%%%%%%%%%%%%%%%%%%%%
\subsection{eventloop.hh}

This file defines \emph{class EventLoop}.
It is used to co-ordinate interactions between a TimerList and a
SelectorList for XORP processes.  All XorpTimer and select operations
should be co-ordinated through this interface.

%%%%%%%%%%%%%%%%%%%%%%%%%%%%%%%%%%%%%%%%%%%
\subsection{exceptions.hh}

This file contains \emph{class XorpException}: a base class for XORP C++
exceptions. It contains also all standard XORP C++ exceptions.
An example of such exception is \emph{class InvalidFamily} which is
thrown if the address family is invalid (for example, by an IPvX
constructor when invoked with an invalid address family).

%%%%%%%%%%%%%%%%%%%%%%%%%%%%%%%%%%%%%%%%%%%
\subsection{heap.hh}

This file contains \emph{class Heap}.  The Heap class is used by the
TimerList class as it's priority queue for timers. This implementation
supports removal of arbitrary objects from the heap, even if they are
not located at the top.

%%%%%%%%%%%%%%%%%%%%%%%%%%%%%%%%%%%%%%%%%%%
\subsection{ipnet.hh, ipv4net.hh, ipv6net.hh, ipvxnet.hh}

These files contain the declaration of the following classes:
\emph{class IPv4Net, class IPv6Net, class IPvXNet}, which are
classes for basic subnet addresses (for IPv4, IPv6 and dual IPv4/6
address family respectively). IPvXNet can be used to store a subnet
address that has either IPv4 or IPv6 address family.

Most of the implementation is contained in file \emph{ipnet.hh}, which
contains a \emph{template class IPNet}. The IPv4Net, IPv6Net, and
IPvXNet classes are derived from that template.

%%%%%%%%%%%%%%%%%%%%%%%%%%%%%%%%%%%%%%%%%%%
\subsection{ipv4.hh, ipv6.hh, ipvx.hh}

These files contain the declaration for the following classes:
\emph{class IPv4, class IPv6, class IPvX}, which are
classes for basic IP addresses (for IPv4, IPv6 and dual IPv4/6
address family respectively). IPvX can be used to store an
address that has either IPv4 or IPv6 address family.

%%%%%%%%%%%%%%%%%%%%%%%%%%%%%%%%%%%%%%%%%%%
\subsection{mac.hh}

This file declares the following classes: \emph{class Mac, class
EtherMac}. The first class is a generic container for any type of MAC.
The second class is a container for Ethernet MAC address.

%%%%%%%%%%%%%%%%%%%%%%%%%%%%%%%%%%%%%%%%%%%
\subsection{nexthop.hh}

This file declares a number of classes that can be used to contain
routing next-hop information. For example, \emph{class NextHop}
is the generic class for holding information about routing
next hops.  NextHops can be of many types, including immediate
neighbors, remote routers (with IBGP), discard interfaces,
encapsulation endpoints, etc.  NextHop itself doesn't really do
anything useful, except to provide a generic handle for the
specialized subclasses. The specialized subclasses are:

\begin{itemize}

  \item IPPeerNextHop is for next hops that are local peers.

  \item IPEncapsNextHop is for ``next hops'' that are non-local, and require
   encapsulation to reach. An example is the PIM Register Encapsulation.

  \item IPExternalNextHop An IP nexthop that is not an intermediate
  neighbor.

  \item DiscardNextHop is a discard interface.

\end{itemize}

%%%%%%%%%%%%%%%%%%%%%%%%%%%%%%%%%%%%%%%%%%%
\subsection{popen.hh}

This file contains the interface for the local implementation
of \emph{popen(2)} and \emph{pclose(2)}.
The corresponding local names are \emph{popen2()} and \emph{pclose2()}
respectively. Unlike the system's \emph{popen(2)}, the local
\emph{popen2()} implementation allows the user to specify the streams
where the \emph{stdout} and \emph{stderr} of the command will be
redirected to.

%%%%%%%%%%%%%%%%%%%%%%%%%%%%%%%%%%%%%%%%%%%
\subsection{profile.hh}

This file implements the mechanism for event profiling.
The developer can add profiling entries at various places of the program.
Each profiling entry has a name and it can be enabled or disabled.
In addition, a number of strings can be added to each profiling entry,
and those strings can be read at some later stage.

% TODO: add an example of using the profiling mechanism.

%%%%%%%%%%%%%%%%%%%%%%%%%%%%%%%%%%%%%%%%%%%
\subsection{ref\_ptr.hh}

This file declares \emph{template class ref\_ptr}: reference counted
pointer class.

The ref\_ptr class is a strong reference class.  It maintains a count of
how many references to an object exist and releases the memory associated
with the object when the reference count reaches zero.  The reference
pointer can be dereferenced like an ordinary pointer to call methods
on the reference counted object.

At the time of writing the only supported memory management is
through the new and delete operators.  At a future date, this class
should support the STL allocator classes or an equivalent to
provide greater flexibility.

%%%%%%%%%%%%%%%%%%%%%%%%%%%%%%%%%%%%%%%%%%%
\subsection{ref\_trie.hh}

This file implements a trie to support route lookups.  The
implementation is template-based, and is based on the code in
trie.hh. From deleloper's point of view, templates RefTrie, RefTrieNode,
RefTriePreOrderIterator, and RefTriePostOrderIterator are the most important.
Those templates should be invoked
with two classes, the basetype ``A'' for the search Key (which is a
subnet, \verb=IPNet<A>=), and the Payload.

RefTrie differs from Trie (and its associated classes) in that the
RefTrieNode includes a reference count of how many RefTrieIterators
are pointing at it.  If a RefTrieNode is deleted, but has a non-zero
reference count, deletion will be delayed until the reference count
becomes zero.  In this way, additions and deletions to the RefTrie
cannot cause a RefTriePreOrderIterator or RefTriePostOrderIterator
to reference invalid memory, although a deletion and subsequent addition can
cause the payload data referenced by an iterator to change.

%%%%%%%%%%%%%%%%%%%%%%%%%%%%%%%%%%%%%%%%%%%
\subsection{run\_command.hh}

This file implements \emph{class RunCommand} which provides the
mechanism for running an external command. In addition to the command
name and its arguments, the developer can specify three callbacks:

\begin{itemize}

  \item \emph{stdout\_cb}: the callback to call when there is data on
   the standard output.

  \item \emph{stderr\_cb}: the callback to call when there is data on
   the standard error.

  \item \emph{done\_cb}: the callback to call when the command is completed.

\end{itemize}

%%%%%%%%%%%%%%%%%%%%%%%%%%%%%%%%%%%%%%%%%%%
\subsection{safe\_callback\_obj.hh}

This file declares class \emph{CallbackSafeObject}.
Objects that wish to be callback safe should be derived from this
class. When a CallbackSafeObject is destructed it informs all the callbacks
that refer to it that this is the case and invalidates (sets to null)
the object they point to.

%%%%%%%%%%%%%%%%%%%%%%%%%%%%%%%%%%%%%%%%%%%
\subsection{selector.hh}

This file contains the I/O multiplexing interface. The particular class
of interest is \emph{class SelectorList}.

A SelectorList provides an entity where callbacks for pending I/O
operations on file descriptors may be registered.  The callbacks
are invoked when one of the select methods is called and I/O
is pending on the particular descriptors.

%%%%%%%%%%%%%%%%%%%%%%%%%%%%%%%%%%%%%%%%%%%
\subsection{service.hh}

This declares \emph{class ServiceBase}.  A service is a class that can
be started and stopped and would typically involve some asynchronous
processing to transition between states.  The base class provides a
state model and methods for transitioning between states.  Mandatory
transition methods, like start and stop, are abstract in the base
class.

%%%%%%%%%%%%%%%%%%%%%%%%%%%%%%%%%%%%%%%%%%%
\subsection{status\_codes.h}

This file contains the enumerated \emph{ProcessStatus} codes
that a XORP process should report to the XORP router manager
(\emph{rtrmgr})~\cite{xorp:rtrmgr}.
The file itself contains a detailed explanation of the process states
(valid transaction between states, triggering events, actions, etc).

%%%%%%%%%%%%%%%%%%%%%%%%%%%%%%%%%%%%%%%%%%%
\subsection{time\_slice.hh}

This file declares \emph{class TimeSlice}.
This class can be used to compute whether some processing is taking
too long time to complete. It is up to the program that uses
TimeSlice to check whether the processing is taking too long,
and suspend processing of that task if necessary.

%%%%%%%%%%%%%%%%%%%%%%%%%%%%%%%%%%%%%%%%%%%
\subsection{timer.hh}

This file declares the XORP timer facility. The only class of interest
from a developer's point of view is \emph{class XorpTimer}.

%%%%%%%%%%%%%%%%%%%%%%%%%%%%%%%%%%%%%%%%%%%
\subsection{timespent.hh}

This files declares and implements \emph{class TimeSpent}.
This class used for debugging purpose to find code that has taken too long to
execute.

%%%%%%%%%%%%%%%%%%%%%%%%%%%%%%%%%%%%%%%%%%%
\subsection{timeval.hh}

This file contains implementation of \emph{class TimeVal} for
storing time values (similar to \emph{struct timeval}).
\emph{TimeVal} implements the appropriate constructors and numerous
helper methods (\eg Less-Than and Addition operators, etc).

%%%%%%%%%%%%%%%%%%%%%%%%%%%%%%%%%%%%%%%%%%%
\subsection{token.hh}

This file contains various token-related definitions. Token is a
sequence of symbols separated from other tokens by some pre-defined
symbols. In this implementation, the separators are the is\_space(3) and
'|' characters.
The facilities in that file are to copy tokens, removing them from a
token line, etc.
Currently, this file is used only by the CLI, therefore in the future it
may be moved to the CLI itself.

%%%%%%%%%%%%%%%%%%%%%%%%%%%%%%%%%%%%%%%%%%%
\subsection{transactions.hh}

This file contains facility for transactions.  A transaction consists
of a sequence of transaction operations, each of which is a command.
The TransactionManager class provides a front-end for creating,
dispatching, and destroying transactions.

%%%%%%%%%%%%%%%%%%%%%%%%%%%%%%%%%%%%%%%%%%%
\subsection{trie.hh}

This file implements a trie to support route lookups.
The implementation is template-based. From deleloper's point
of view, templates Trie, TrieNode, TriePreOrderIterator,
and TriePostOrderIterator
are the most important. Those templates should be invoked with two
classes, the basetype ``A'' for the search Key (which is a subnet,
\verb=IPNet<A>=), and the Payload.

%%%%%%%%%%%%%%%%%%%%%%%%%%%%%%%%%%%%%%%%%%%
\subsection{utility.h}

This file contains various mini-utilities.  Those utilities are mostly
compiler-related helpers; \eg compile-time assertion, \emph{UNUSED(var)}
macro to suppress warnings about unused functions arguments, etc.

%%%%%%%%%%%%%%%%%%%%%%%%%%%%%%%%%%%%%%%%%%%
\subsection{utils.hh}

This file contains various helper utilities. Currently, the only two
utilities are template functions to delete a list or array
of pointers and the objects pointed to.

%%%%%%%%%%%%%%%%%%%%%%%%%%%%%%%%%%%%%%%%%%%
\subsection{vif.hh}

This file declares the following classes: \emph{class Vif, class
VifAddr}.

Class Vif holds information about a virtual interface.  A Vif may
represent a physical interface, or may represent more abstract
entities such as the Discard interface, or a VLAN on a physical
interface.
VifAddr holds information about an address of a virtual interface.
A virtual interface may have more than one VifAddr.

%%%%%%%%%%%%%%%%%%%%%%%%%%%%%%%%%%%%%%%%%%%
\subsection{xlog.h}

This file provides facility for log messages generation, similar to
syslog. The log messages may be output to multiple output streams
simultaneously. Below is a description of how to use the log utility.

\begin{itemize}

  \item The xlog utility assumes that \verb=XORP_MODULE_NAME= is defined
   (per module). To do so, you must have in your directory a file like
   ``foo\_module.h'', and inside it should contain something like:

\begin{verbatim}
#define XORP_MODULE_NAME "BGP"
\end{verbatim}

   This file then has to be included by each *.c and *.cc file,
   and MUST be the first of the included local files.

  \item Before using the xlog utility, a program MUST initialize it
   first (think of this as the xlog constructor):

\begin{verbatim}
int xlog_init(const char *process_name, const char *preamble_message);
\end{verbatim}

   Further, if a program tries to use xlog without initializing it
   first, the program will exit.

  \item To add output streams, you MUST use one of the following (or both):

\begin{verbatim}
int xlog_add_output(FILE* fp);
int xlog_add_default_output(void);
\end{verbatim}

  \item To change the verbosity of all xlog messages, use:

\begin{verbatim}
xlog_set_verbose(xlog_verbose_t verbose_level);
\end{verbatim}

   where ``verbose\_level'' is one of the following (\verb=XLOG_VERBOSE_MAX=
   excluded):

\begin{verbatim}
typedef enum {
    XLOG_VERBOSE_LOW = 0,       /* 0 */
    XLOG_VERBOSE_MEDIUM,        /* 1 */
    XLOG_VERBOSE_HIGH,          /* 2 */
    XLOG_VERBOSE_MAX
} xlog_verbose_t;
\end{verbatim}

   Default value is \verb=XLOG_VERBOSE_LOW= (least details).
   Larger value for ``verbose\_level'' adds more details to the
   preamble message (e.g., file name, line number, etc, about the place
   where the log message was initiated).

   Note that the verbosity level of message type \verb=XLOG_LEVEL_FATAL= (see
   below) cannot be changed and is always set to the most verbose level
   (\verb=XLOG_VERBOSE_HIGH=).

  \item To change the verbosity of a particular message type, use:

\begin{verbatim}
void xlog_level_set_verbose(xlog_level_t log_level,
	xlog_verbose_t verbose_level);
\end{verbatim}

where ``log\_level'' is one of the following (\verb=XLOG_LEVEL_MAX=
excluded):

\begin{verbatim}
typedef enum {
    XLOG_LEVEL_FATAL = 0,       /* 0 */
    XLOG_LEVEL_ERROR,           /* 1 */
    XLOG_LEVEL_WARNING,         /* 2 */
    XLOG_LEVEL_INFO,            /* 3 */
    XLOG_LEVEL_TRACE,           /* 4 */
    XLOG_LEVEL_MAX
} xlog_level_t;
\end{verbatim}

   Note that the verbosity level of message type \verb=XLOG_LEVEL_FATAL=
   cannot be changed and is always set to the most verbose level
   (\verb=XLOG_VERBOSE_HIGH=).

  \item To start the xlog utility, you MUST use:

\begin{verbatim}
int xlog_start(void);
\end{verbatim}


  \item To enable or disable a particular message type, use:

\begin{verbatim}
int xlog_enable(xlog_level_t log_level);
int xlog_disable(xlog_level_t log_level);
\end{verbatim}

	By default, all levels are enabled.
	Note that \verb=XLOG_LEVEL_FATAL= cannot be disabled.

  \item To stop the logging, use:

\begin{verbatim}
int xlog_stop(void);
\end{verbatim}

	Later you can restart it again by \verb=xlog_start()=

  \item To gracefully exit the xlog utility, use

\begin{verbatim}
int     xlog_exit(void);
\end{verbatim}

	(think of this as the xlog destructor).
	An example:

\begin{verbatim}
int
main(int argc, char *argv[])
{
    //
    // Initialize and start xlog
    //
    xlog_init(argv[0], NULL);
    xlog_set_verbose(XLOG_VERBOSE_LOW);	// Least verbose messages
    // Increase verbosity of the error messages
    xlog_level_set_verbose(XLOG_LEVEL_ERROR, XLOG_VERBOSE_HIGH);
    xlog_add_default_output();
    xlog_start();

    // Do something

    //
    // Gracefully stop and exit xlog
    //
    xlog_stop();
    xlog_exit();

    exit (0);
}
\end{verbatim}

\end{itemize}


Typically, a developer would use the macros described below
to print a message, add an assert statement, place a marker, etc.
If a macro accepts a message to print, the format of the message is same
as printf(3). The only difference is that the xlog utility automatically
adds \verb='\n'=, (i.e. end-of-line) at the end of each string
specified by \verb=format=:

\begin{itemize}

  \item \verb=XLOG_FATAL(const char *format, ...)=
  \newline
  Write a FATAL message to the xlog output streams and abort the program.

  \item \verb=XLOG_ERROR(const char *format, ...)=
  \newline
  Write an ERROR message to the xlog output streams.

  \item \verb=XLOG_WARNING(const char *format, ...)=
  \newline
  Write a WARNING message to the xlog output streams.

  \item \verb=XLOG_INFO(const char *format, ...)=
  \newline
  Write an INFO message to the xlog output streams.

  \item \verb=XLOG_TRACE(int cond_boolean, const char *format, ...)=
  \newline
  Write a TRACE message to the xlog output stream, but only
  if \verb=cond_boolean= is not 0.

  \item \verb=XLOG_ASSERT(assertion)=
  \newline
  The XORP replacement for assert(3), except that it cannot be
  conditionally disabled and logs error messages through the standard
  xlog mechanism. It calls \verb=XLOG_FATAL()= if the assertion fails.

  \item \verb=XLOG_UNREACHABLE()=
  \newline
   A marker that can be used to indicate code that should never be executed.

  \item \verb=XLOG_UNFINISHED()=
  \newline
  A marker that can be used to indicate code that is not yet implemented
  and hence should not be run.

\end{itemize}

%%%%%%%%%%%%%%%%%%%%%%%%%%%%%%%%%%%%%%%%%%%
\subsection{xorp.h}

This is the XORP main include file that should be included by all XORP
C and C++ files. This file itself includes a number of frequently used system
header files, defines several commonly used values, etc.


%%%%%%%%%%%%%%%%%%%%%%%%%%%%%%%%%%%%%%%%%%%%%%%%%%%%%%%%%%%%%%%%%%%%%%%
%     APPENDIX
%%%%%%%%%%%%%%%%%%%%%%%%%%%%%%%%%%%%%%%%%%%%%%%%%%%%%%%%%%%%%%%%%%%%%%%
\appendix
\section{Modification History}

\begin{itemize}

  \item December 11, 2002: Version 0.1 completed.

  \item March 10, 2003: Updated to match XORP version 0.2 release code;
  add information about RefTrie; cleanup.

  \item June 9, 2003: Updated to match XORP version 0.3 release code.

  \item August 28, 2003: Updated to match XORP version 0.4 release
  code.

  \item November 6, 2003: Updated to match XORP version 0.5 release code.

  \item July 8, 2004: Updated to match XORP version 1.0 release code.

  \item April 12, 2005: Updated to match XORP version 1.1 release code:
   added information for \emph{buffered\_asyncio.hh}, \emph{clock.hh},
   \emph{popen.hh}, \emph{profile.hh} and \emph{run\_command.hh}.

\end{itemize}

%%%%%%%%%%%%%%%%%%%%%%%%%%%%%%%%%%%%%%%%%%%%%%%%%%%%%%%%%%%%%%%%%%%%%%%
%     BIBLIOGRAPHY
%%%%%%%%%%%%%%%%%%%%%%%%%%%%%%%%%%%%%%%%%%%%%%%%%%%%%%%%%%%%%%%%%%%%%%%
\bibliography{../tex/xorp}
\bibliographystyle{plain}

%%%%%%%%%%%%%%%%%%%%%%%%%%%%%%%%%%%%%%%%%%%%%%%%%%%%%%%%%%%%%%%%%%%%%%%
\end{document}
