%
% $XORP: xorp/docs/user_manual/forwarding.tex,v 1.6 2005/04/07 03:51:22 pavlin Exp $
%

\chapter{Forwarding Engine}

\section{Terminology and Concepts}

The forwarding engine is that part of a router that receives packets
and forwards then from one interface to another.  In the case of XORP,
the forwarding engine may be the kernel forwarding path on Linux or
FreeBSD, or it may reside in external forwarding hardware.  

On any particular router, it might be desirable to enable or disable
different parts of the forwarding functionality.  For example, a
router might only be intended to forward IPv6 packets but not IPv4
packets, or it might be intended to forward unicast packets but not
multicast packets.    Thus XORP provides the ability to enable and
configure various forwarding functionality.

In XORP, the term ``{\stt fea}'' refers to {\it Forwarding Engine
Abstraction} and the term ``{\stt mfea}'' refers to {\it Multicast
Forwarding Engine Abstraction}.  The term abstraction here refers to a
high-level configuration interface that should be the same
irrespective of whether the forwarding engine is provided in software
in the operating system kernel or in external forwarding hardware.

\newpage
\section{Configuration of the Forwarding Engine}

On a XORP router, forwarding functionality must be explicitly enabled
or no packets will be forwarded.  Forwarding can be separately enabled
for unicast and multicast, and for IPv4 and IPv6.  In addition,
multicast interfaces/vifs need to be explicitly enabled individually,
and certain special-purpose forwarding functionality can also be
enabled for multicast.

\subsection{Configuration Syntax}
\vspace{0.1in}
\noindent\framebox[\textwidth][l]{\scriptsize
\begin{minipage}{6in}
\begin{alltt}
\begin{tabbing}
xx\=xx\=xx\=xx\=xx\=\kill
fea \{\\
\>targetname: {\it txt}\\
\>unicast-forwarding4 \{\\
\>\>disable: {\it bool}\\
\>\}\\
\>unicast-forwarding6 \{\\
\>\>disable: {\it bool}\\
\>\}\\
\}
\\
plumbing \{\\
\>mfea4 \{\\
\>\>disable: {\it bool}\\
\>\>interface {\it text} \{\\
\>\>\>vif {\it text} \{\\
\>\>\>\>disable: {\it bool}\\
\>\>\>\}\\
\>\>\}\\
\>\>interface register\_vif \{\\
\>\>\>vif register\_vif \{\\
\>\>\>\>disable: {\it bool}\\
\>\>\>\}\\
\>\>\}\\
\>\>traceoptions \{\\
\>\>\>flag all \{\\
\>\>\>\>disable: {\it bool}\\
\>\>\>\}\\
\>\>\}\\
\>\}\\
\\
\>mfea6 \{\\
\>\>disable: {\it bool}\\
\>\>interface {\it text} \{\\
\>\>\>vif {\it text} \{\\
\>\>\>\>disable: {\it bool}\\
\>\>\>\}\\
\>\>\}\\
\>\>interface register\_vif \{\\
\>\>\>vif register\_vif \{\\
\>\>\>\>disable: {\it bool}\\
\>\>\>\}\\
\>\>\}\\
\>\>traceoptions \{\\
\>\>\>flag \{\\
\>\>\>\>all \{\\
\>\>\>\>\>disable: {\it bool}\\
\>\>\>\>\}\\
\>\>\>\}\\
\>\>\}\\
\>\}\\
\}
\end{tabbing}
\end{alltt}
\end{minipage}
}
\vspace{0.1in}

\begin{description}
\item{\stt fea}: this delimits the configuration for the unicast
  forwarding engine functionality. 

  The following unicast forwarding engine parameters can be configured:
\begin{description}
\item{\tt targetname}: this is the name for this instance of the
  forwarding engine abstraction.  It defaults to ``{\stt fea}'', and
  it is strongly recommended that this default is {\it not} overridden
  under normal usage scenarios.
\end{description}
\item{\stt unicast-forwarding4}: this directive is used to configure the IPv4
  forwarding~\footnote{Note that prior to XORP Release-1.1, the
  {\tt enable-unicast-forwarding4} flag was used instead to enable or disable
  the IPv4 forwarding.}. Possible parameters are:
\begin{description}
\item{\stt disable}: this takes the value {\stt true} or {\stt false},
  and disables or enables all IPv4 unicast forwarding on the router.
\end{description}
\item{\stt unicast-forwarding6}: this directive is used to configure the IPv6
  forwarding~\footnote{Note that prior to XORP Release-1.1, the
  {\tt enable-unicast-forwarding6} flag was used instead to enable or disable
  the IPv6 forwarding.}. Possible parameters are:
\begin{description}
\item{\stt disable}: this takes the value {\stt true} or {\stt false},
  and disables or enables all IPv6 unicast forwarding on the router.
\end{description}
\item{\stt plumbing}: this delimits a part of the router configuration
  used for the plumbing together of packet forwarding functionality.
  Multicast forwarding configuration must be part of this grouping.
\item{\stt mfea4}: this delimits the part of the router configuration
  related to multicast forwarding of IPv4 packets.

  The following multicast forwarding parameters can be configured:
\begin{description}
\item{\stt disable}: this takes the value {\stt true} or {\stt false},
  and disables or enables all IPv4 multicast forwarding on the
  router~\footnote{Note that prior to XORP Release-1.1, the {\tt enable}
  flag was used instead of {\tt disable}.}.
  The default is {\stt false}.
\item{\stt interface}: this specifies an interface to be used for
  multicast IPv4 forwarding.  Each interface to be used for multicast
  forwarding needs to be explicitly listed.

  In addition to the normal network interfaces, a special-purpose
  interface called {\stt register\_vif} needs to be configured for
  PIM-SM (see Chapter \ref{pimsm}) to be able to send
  register-encapsulated packets to the PIM Rendezvous Point.  PIM-SM
  will not work correctly unless this is configured.  The {\stt
  register\_vif} interface must be configured with a vif also called
  {\stt register\_vif}.
\item{\stt vif}: this specifies a vif to be used for multicast IPv4
  forwarding.  Each vif to be used for multicast forwarding needs to
  be explicitly listed.

  Each vif can take the following parameter:
\begin{description}
\item{\stt disable}: this takes the value {\stt true} or {\stt false},
  and disables or enables multicast forwarding on this vif~\footnote{Note
  that prior to XORP Release-1.1, the {\tt enable} flag was used instead of
  {\tt disable}.}.
  The default is {\stt false}.
\end{description}
\end{description}
\item{\stt traceoptions}: this directive delimits the configuration of
  debugging and tracing options for multicast forwarding.
\begin{description}
\item{\stt flag}: this directive is used to specify which tracing
  options are enabled.  Possible parameters are:
\begin{description}
\item{\stt all}: this directive specifies that all tracing
  options should be enabled.  Possible parameters are:
\begin{description}
\item{\stt disable}: this takes the value {\stt true} or {\stt false},
  and disables or enables tracing~\footnote{Note
  that prior to XORP Release-1.1, the {\tt enable} flag was used instead of
  {\tt disable}.}. The default is {\stt false}.
\end{description}
\end{description}
\end{description}
\item{\stt mfea6}: this delimits the part of the router configuration
  related to multicast forwarding of IPv6 packets.  The possible
  parameters are the same as for {\stt mfea4}, but affect IPv6
  multicast forwarding rather than IPv4.
\end{description}


\subsection{Example Configurations}
\vspace{0.1in}
\noindent\framebox[\textwidth][l]{\scriptsize
\begin{minipage}{6in}
\begin{alltt}
\begin{tabbing}
xx\=xx\=xx\=xx\=xx\=\kill
fea \{\\
\>unicast-forwarding4 \{\\
\>\>disable: false\\
\>\}\\
\>unicast-forwarding6 \{\\
\>\>disable: true\\
\>\}\\
\}
plumbing \{\\
\>mfea4 \{\\
\>\>disable: false\\
\>\>interface dc0 \{\\
\>\>\>vif dc0 \{\\
\>\>\>\>disable: false\\
\>\>\>\}\\
\>\>\}\\
\>\>interface register\_vif \{\\
\>\>\>vif register\_vif \{\\
\>\>\>\>/* Note: this vif should be always enabled */\\
\>\>\>\>disable: false\\
\>\>\>\}\\
\>\>\}\\
\>\>traceoptions \{\\
\>\>\>flag all \{\\
\>\>\>\>disable: false\\
\>\>\>\}\\
\>\>\}\\
\>\}\\
\\
\>mfea6 \{\\
\>\>disable: false\\
\>\>interface dc0 \{\\
\>\>\>vif dc0 \{\\
\>\>\>\>disable: false\\
\>\>\>\}\\
\>\>\}\\
\>\>interface register\_vif \{\\
\>\>\>vif register\_vif \{\\
\>\>\>\>/* Note: this vif should be always enabled */\\
\>\>\>\>disable: false\\
\>\>\>\}\\
\>\>\}\\
\>\}\\
\}
\end{tabbing}
\end{alltt}
\end{minipage}
}

\vspace{0.1in}
The configuration above enables unicast IPv4 forwarding, but disables
IPv6 unicast forwarding.

In addition, it enables multicast forwarding for IPv4 and IPv6 on
interface/vif {\stt dc0/dc0}, and enables the register vif for use by
PIM-SM multicast routing.

\section{Monitoring the Forwarding Engine}

The {\stt show mfea dataflow} command can be used to display
information about MFEA IPv4 dataflow filters:

\vspace{0.1in}
\noindent\framebox[\textwidth][l]{\scriptsize
\begin{minipage}{6in}
\begin{alltt}
\begin{tabbing}
xx\=xxxxxxxxxxxxxxxxxxxxxxxxxxxxxx\=xxxxx\=xx\=xxxxxxxxxxxxxxxxxxxxxxxxxxxxxxx\=xx\=\kill
Xorp> \textbf{show mfea dataflow}\\
Group                              \>\>\>\>Source\\
224.0.1.20                         \>\>\>\>10.2.0.1\\
\>Measured(Start|Packets|Bytes)\>Type\>Thresh(Interval|Packets|Bytes)\>\>Remain\\
\>1091667269.982158|0|?        \><= \>\>210.0|0|?                \>\>202.434319\\
\>1091667269.984406|?|0        \>>= \>\>100.0|?|102400           \>\>92.436567
\end{tabbing}
\end{alltt}
\end{minipage}
}
\vspace{0.1in}

Note that the above information is shown only if the filters are kept at
user-space. If the filters are kept at kernel-space (\eg in case of UNIX
system with advanced multicast API support), then currently xorpsh
cannot be used to show the information. In that case, the appropriate system
command should be used instead (e.g., the UNIX {\stt netstat -gn} command).

The {\stt show mfea interface} command can be used to display
information about MFEA IPv4 interfaces:

\vspace{0.1in}
\noindent\framebox[\textwidth][l]{\scriptsize
\begin{minipage}{6in}
\begin{alltt}
\begin{tabbing}
xxxxxxxxxxxxx\=xxxxxxxxx\=xxxxxxxxxxxxx\=xxxxxxxxxxxxxxxx\=xxxxx\=\kill
Xorp> \textbf{show mfea interface}\\
Interface \>State  \>Vif/PifIndex \>Addr       \>Flags\\
dc0       \>UP     \>0/6          \>10.4.0.1   \>MULTICAST BROADCAST KERN\_UP\\
dc2       \>UP     \>1/8          \>10.3.0.2   \>MULTICAST BROADCAST KERN\_UP\\
register\_vif \>UP \>2/6          \>10.4.0.1   \>PIM\_REGISTER KERN\_UP
\end{tabbing}
\end{alltt}
\end{minipage}
}
\vspace{0.1in}

The {\stt show mfea interface address} command can be used to display
information about MFEA IPv4 interface addresses:

\vspace{0.1in}
\noindent\framebox[\textwidth][l]{\scriptsize
\begin{minipage}{6in}
\begin{alltt}
\begin{tabbing}
xxxxxxxxxxxxx\=xxxxxxxxxxxxxxxx\=xxxxxxxxxxxxxxxxxxx\=xxxxxxxxxxxxxxxx\=xxxxxxx\kill
Xorp> \textbf{show mfea interface address}\\
Interface  \>Addr          \>Subnet           \>Broadcast     \>P2Paddr\\
dc0        \>10.4.0.1      \>10.4.0.0/24      \>10.4.0.255    \>0.0.0.0\\
dc2        \>10.3.0.2      \>10.3.0.0/24      \>10.3.0.255    \>0.0.0.0\\
register\_vif \>10.4.0.1   \>10.4.0.1/32      \>10.4.0.1      \>0.0.0.0
\end{tabbing}
\end{alltt}
\end{minipage}
}

The equivalent commands for IPv6 multicast forwarding are:
\begin{description}
\item{\stt show mfea6 dataflow}
\item{\stt show mfea6 interface}
\item{\stt show mfea6 interface address}
\end{description}
\vspace{0.1in}
