%
% $XORP: xorp/docs/user_manual/vrrp.tex,v 1.1 2008/10/09 18:05:00 abittau Exp $
%

\chapter{\label{vrrp}VRRP}

\section{VRRP Terminology and Concepts}
XORP supports Virtual Router Redundancy Protocol (VRRP) version 2 as described
in RFC 3768.  VRRP increases the robustness of networks where a default gateway
is defined.  Rather than having a single point of failure (the default gateway),
VRRP allows multiple routers to act as the default gateway.  Routers
participating in VRRP will elect one master that will act as the default
gateway, and the other routers will act as a backup.  When the master fails, a
backup router is elected as the new master.  When the original master returns to
life, it will obtain its role as master again.  

To detect the failure of a master, backup routers listen to advertisements that
are sent out by the master at a periodic interval.  To elect a new master, each
router is assigned a priority which will indicate the router's preference in
becoming a master.  A preemption mode is available that will force a backup
router to become master if another backup router with lower priority is
currently acting as a master.  Note that the router that owns the IP addresses
of the VRRP group will always preempt a backup router, regardless of the
preemption setting.

\section{Configuration of VRRP}
The configuration syntax for XORP VRRP is given below.

\vspace{0.1in}                              
\noindent\framebox[\textwidth][l]{\scriptsize
\begin{minipage}{6in}
\begin{alltt}
\begin{tabbing}
xx\=xx\=xx\=xx\=xx\=\kill
protocols \{\\
\>vrrp \{\\
\>\>interface: {\it text} \{\\
\>\>\>vif: {\it text} \{\\
\>\>\>\>vrid: {\it int(0..255)} \{\\
\>\>\>\>\>priority: {\it int(1..254)} \\
\>\>\>\>\>interval: {\it int(1..255)} \\
\>\>\>\>\>preempt: {\it bool} \\
\>\>\>\>\>ip {\it IPv4} \{\}\\
\>\>\>\>\>disable: {\it bool} \\
\>\>\>\>\}\\
\>\>\>\}\\
\>\>\}\\
\>\}\\
\}
\end{tabbing}
\end{alltt}
\end{minipage}
}
\vspace{0.1in}

\noindent
The parameters are used as follows:
\begin{description}
\item[interface, vif] The interface on which to run a VRRP instance.

\item[vrid] The ID of the VRRP instance.  Must be unique per interface.

\item[priority] The priority of the router.  The higher the priority, the more
likely this router will become a master when acting as a backup.  Priority 255
is reserved for the router that owns the IP addresses of the VRRP group.
Priority 0 is reserved as it is used to indicate when a master leaves a VRRP
group.  The default priority is 100.

\item[interval]  The interval in seconds between VRRP advertisements.  The
default is 1 second.

\item[preempt]  Whether preemption is used.  If preempt is true, when a backup
router has higher priority than the current master, it will preempt the master
in order to become the new master.  Preemption is false by default.  Note that a
router that owns the IP addresses will preempt a backup router regardless of the
setting of this flag.

\item[ip]  The IP addresses associated with this VRRP group.  These are the IP
addresses that client machines will use as their default gateway.

\item[disable]  A flag that can be used to disable or enable this VRRP instance.
\end{description}

\section{Monitoring VRRP}
One can inspect VRRP's state with the following command:

\vspace{0.1in}
\noindent\framebox[\textwidth][l]{\scriptsize
\begin{minipage}{6in}
\begin{alltt}
\begin{tabbing}
xx\=xxxxxxxxxxxxxxxxxxxxx\=\kill
user@hostname> \textbf{show vrrp}\\
\>Interface\>dummy0\\
\>Vif\>dummy0\\
\>VRID\>1\\
\>State\>master\\
\>Master IP\>9.9.9.9\\
\\
\>Interface\>tap3\\
\>Vif\>tap3\\
\>VRID\>1\\
\>State\>initialize\\
\>Master IP\>0.0.0.0
\end{tabbing}
\end{alltt}
\end{minipage}
}
\vspace{0.1in}

The command will show all configured VRRP instances, printing their physical
interface, logical interface (Vif), the VRID the state and the master's IP
address.  The state can be one of three values: initialize, master or backup.
The initialize state means that VRRP is not running.  Reasons for this include
the VRRP instance being disabled with the {\tt disable} configuration option,
the physical interface being disabled, or no IP addresses configured on the
interface in which VRRP is supposed to run.  When in the initialize state, the
master's IP address is undefined.  In the backup state, it represents the
address of the master according to the last advertisement received.  In the
master state it is the router's own IP address.

Rather than viewing all configured VRRP instances, one can display the instances
configured on a particular instance by supplying a physical and logical interface
name, or view a particular instance by additionally supplying the VRID.

\section{Limitations}
The current implementation has the following known limitations:
\begin{itemize}
\item {\bf Not RFC compliant when a backup router owns only some of the IP
addresses.}  When acting as a backup router, the router must not accept any
traffic directed to the IP addresses configured in VRRP.  XORP's implementation
though will accept data arriving to any of the router's configured IP addresses.
If any of these are IP addresses configured in VRRP, their traffic will be
accepted rather than dropped.  Note that if the router owns all IP addresses it
will never act as a backup router (by definition).  Hence this case occurs only
when a backup router owns only some of the IP addresses configured in VRRP.

\item {\bf Only one VRRP instance per interface.}  Acting as a VRRP router
requires listening to a special virtual router MAC address.  One of these is
defined for each VRID.  Running multiple VRRP instances on a single interface
implies multiple VRIDs and hence the ability to listen on multiple unicast MAC
addresses.  We do not support this since only one unicast MAC address can be
assigned to a physical network card.  An alternative would be putting the
interface into promiscuous mode, a solution which we are considering to
implement.

One thing we currently do though is to add additional MAC addresses as multicast
addresses.  With some hardware, this allows the kernel to receive packets for
these destinations.  It is possible that some kernels will accept this data and
hence multiple VRRP instances on one interface actually work with the current
implementation.  Modern Linux kernels though drop these packets, so we are
rather pessimistic on this hack working---use it at your own risk.
\end{itemize}
