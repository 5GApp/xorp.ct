%
% $XORP: xorp/docs/user_manual/pimsm.tex,v 1.29 2007/04/02 21:53:23 pavlin Exp $
%

\chapter{PIM Sparse-Mode}
\label{pimsm}

\section{Terminology and Concepts}

PIM stands for {\it Protocol Independent Multicast}, and denotes a
class of multicast routing protocols.  The term {\it protocol
independent} comes from the fact that PIM does not have its own
topology discovery protocol, but instead relies on routing information
supplied by protocols such as RIP and BGP.  What PIM does do is to
build multicast trees from senders to receivers based on paths
determined by this external topology information.  

There are two PIM protocols:
\begin{itemize}
\item PIM Sparse-Mode (PIM-SM) is the most commonly used multicast
  routing protocol, and explicitly builds distribution trees from the
  receivers back towards senders.
\item PIM Dense-Mode (PIM-DM) is less commonly used, and builds trees
  by flooding multicast traffic domain-wide, and then pruning off
  branches from the tree where there are no receivers.  
\end{itemize}
At the present time, XORP only implements PIM Sparse Mode.

\subsection{PIM-SM Protocol Overview}

{\it The following description is adapted from the PIM-SM
  specification.  }

PIM-SM relies on an underlying topology-gathering protocol to populate a
routing table with routes.  This routing table is called the {\it MRIB} or
{\it Multicast Routing Information Base}.  The routes in this table may be
taken directly from the unicast routing table, or it may be
different and provided by a separate routing protocol such as
Multi-protocol BGP.

Regardless of how it is created, the primary role of the MRIB in the
PIM-SM protocol
is to provide the next-hop router along a multicast-capable
path to each destination subnet.
The MRIB is used to determine the next-hop neighbor to which any PIM
Join/Prune message is sent.
Data flows along the reverse path of the Join messages.
Thus, in contrast to the unicast RIB which specifies the
next-hop that a data packet would take to get {\it to} some subnet,
the MRIB gives reverse-path information, and indicates the path that
a multicast data packet would take {\it from} 
its origin subnet to the router that has the MRIB.  

Like all multicast routing protocols that implement the ASM service model,
PIM-SM must be able to route data packets from sources to receivers
without either the sources or receivers knowing a-priori of the
existence of the others.  This is essentially done in three phases,
although as senders and receivers may come and go at any time, all
three phases may be occur simultaneously.

\subsubsection*{Phase One: RP Tree}

In phase one, a multicast receiver expresses its interest in receiving
traffic destined for a multicast group.  Typically it does this using
IGMP or MLD.  One of the receiver's local PIM routers is elected as the
Designated Router (DR) for that subnet.  On receiving the receiver's
expression of interest, the DR then sends a PIM Join message towards
the Rendezvous Point (RP) for that multicast group.  The RP is a
PIM-SM router that has been configured to serve a bootstrapping role
for certain multicast groups.  This Join message is known as a (*,G)
Join because it joins group G for all sources to that group.  The
(*,G) Join travels hop-by-hop towards the RP for the group, and in
each router it passes through, multicast tree state for group G is
instantiated.  Eventually the (*,G) Join either reaches the RP, or
reaches a router that already has (*,G) Join state for that group.
When many receivers join the group, their Join messages converge on
the RP, and form a distribution tree for group G that is rooted at the
RP.  This is known as the RP Tree (RPT), and is also known as the
shared tree because it is shared by all sources sending to that group.
Join messages are resent periodically so long as the receiver remains
in the group.  When all receivers on a leaf-network leave the group,
the DR will send a PIM (*,G) Prune message towards the RP for that
multicast group. However if the Prune message is not sent for any
reason, the state will eventually time out.

A multicast data sender just starts sending data destined for a
multicast group.  The sender's local router (DR) takes those data
packets, unicast-encapsulates them, and sends them directly to the RP.
The RP receives these encapsulated data packets, decapsulates them,
and forwards them onto the shared tree.  The packets then follow the
(*,G) multicast tree state in the routers on the RP Tree, being
replicated wherever the RP Tree branches, and eventually reaching all
the receivers for that multicast group.  The process of encapsulating
data packets to the RP is called {\it registering}, and the encapsulation
packets are known as PIM Register packets.  

At the end of phase one, multicast traffic is flowing encapsulated to
the RP, and then natively over the RP tree to the multicast receivers.

\subsubsection*{Phase Two: Register-Stop}

Register-encapsulation of data packets is inefficient for two reasons:
\begin{itemize}
\item Encapsulation and decapsulation may be relatively expensive
operations for a router to perform, depending on whether or not the
router has appropriate hardware for these tasks.

\item Traveling all the way to the RP, and then back down the shared
tree may entail the packets traveling a relatively long distance to
reach receivers that are close to the sender.  For some applications,
this increased latency is undesirable.
\end{itemize}
Although Register-encapsulation may continue indefinitely, for the
reasons above, the RP will normally choose to switch to native forwarding.
To do this, when the RP receives a register-encapsulated data packet
from source S on group G, it will normally initiate an (S,G)
source-specific Join towards S.  This Join message travels hop-by-hop
towards S, instantiating (S,G) multicast tree state in the routers
along the path.  (S,G) multicast tree state is used only to forward
packets for group G if those packets come from source S.  Eventually
the Join message reaches S's subnet or a router that already has (S,G)
multicast tree state, and then packets from S start to flow following
the (S,G) tree state towards the RP.  These data packets may also
reach routers with (*,G) state along the path towards the RP - if so,
they can short-cut onto the RP tree at this point.

While the RP is in the process of joining the source-specific tree for
S, the data packets will continue being encapsulated to the RP.
When packets from S also start to arrive natively at the the RP, the
RP will be receiving two copies of each of these packets.  At this
point, the RP starts to discard the encapsulated copy of these
packets, and it sends a {\it Register-Stop} message back to S's DR to
prevent the DR unnecessarily encapsulating the packets.

At the end of phase 2, traffic will be flowing natively from S along a
source-specific tree to the RP, and from there along the shared tree
to the receivers.  Where the two trees intersect, traffic may transfer
from the source-specific tree to the RP tree, and so avoid taking a
long detour via the RP.

It should be noted that a sender may start sending before or after a
receiver joins the group, and thus phase two may happen before the
shared tree to the receiver is built.

\subsubsection*{Phase 3: Shortest-Path Tree}

Although having the RP join back towards the source removes the
encapsulation overhead, it does not completely optimize the forwarding
paths.  For many receivers the route via the RP may involve a
significant detour when compared with the shortest path from the
source to the receiver.  

To obtain lower latencies, a router on the receiver's LAN, typically
the DR, may optionally initiate a transfer from the shared tree to a
source-specific shortest-path tree (SPT).  To do this, it issues an
(S,G) Join towards S.  This instantiates state in the routers along
the path to S.  Eventually this join either reaches S's subnet, or
reaches a router that already has (S,G) state.  When this happens,
data packets from S start to flow following the (S,G) state until they
reach the receiver.

At this point the receiver (or a router upstream of the receiver) will
be receiving two copies of the data - one from the SPT and one from
the RPT.  When the first traffic starts to arrive from the SPT, the DR
or upstream router starts to drop the packets for G from S that arrive
via the RP tree.  In addition, it sends an (S,G) Prune message towards
the RP.  This is known as an (S,G,rpt) Prune.  The Prune message
travels hop-by-hop, instantiating state along the path towards the RP
indicating that traffic from S for G should NOT be forwarded in this
direction.  The prune is propagated until it reaches the RP or a
router that still needs the traffic from S for other receivers.

By now, the receiver will be receiving traffic from S along the
shortest-path tree between the receiver and S.  In addition, the RP is
receiving the traffic from S, but this traffic is no longer reaching
the receiver along the RP tree.  As far as the receiver is concerned,
this is the final distribution tree.

\subsubsection{Multi-access Transit LANs}

The overview so far has concerned itself with point-to-point links.
However, using multi-access LANs such as Ethernet for transit is not
uncommon.  This can cause complications for three reasons:
\begin{itemize}
\item Two or more routers on the LAN may issue (*,G) Joins to different
upstream routers on the LAN because they have inconsistent MRIB
entries regarding how to reach the RP.  Both paths on the RP tree will
be set up, causing two copies of all the shared tree traffic to appear
on the LAN.
\item Two or more routers on the LAN may issue (S,G) Joins to different
upstream routers on the LAN because they have inconsistent MRIB
entries regarding how to reach source S.  Both paths on the
source-specific tree will be set up, causing two copies of all the
traffic from S to appear on the LAN.
\item A router on the LAN may issue a (*,G) Join to one upstream router on
the LAN, and another router on the LAN may issue an (S,G) Join to a
different upstream router on the same LAN.  Traffic from S may reach
the LAN over both the RPT and the SPT.  If the receiver behind the
downstream (*,G) router doesn't issue an (S,G,rpt) prune, then this
condition would persist.
\end{itemize}
All of these problems are caused by there being more than one upstream
router with join state for the group or source-group pair.  PIM-SM does
not prevent such duplicate joins from occurring - instead when
duplicate data packets appear on the LAN from different routers, these
routers notice this, and then elect a single forwarder.  This election
is performed using PIM {\it Assert} messages, which resolve the problem in
favor of the upstream router which has (S,G) state, or if neither or
both router has (S,G) state, then in favor of the router with the best
metric to the RP for RP trees, or the best metric to the source to
source-specific trees.

These Assert messages are also received by the downstream routers on
the LAN, and these cause subsequent Join messages to be sent to the
upstream router that won the Assert.

\subsection*{RP Discovery}

PIM-SM routers need to know the address of the RP for each group for
which they have (*,G) state.  This address is obtained either through
a bootstrap mechanism or through static configuration.

One dynamic way to do this is to use the {\it Bootstrap Router} (BSR)
mechanism. 
One router in each PIM-SM domain is elected the Bootstrap Router through
a simple election process.  All the routers in the domain that are
configured to be candidates to be RPs periodically unicast their
candidacy to the BSR.  From the candidates, the BSR picks an RP-set,
and periodically announces this set in a Bootstrap message.  Bootstrap
messages are flooded hop-by-hop throughout the domain until all
routers in the domain know the RP-Set.

To map a group to an RP, a router hashes the group address into the
RP-set using an order-preserving hash function (one that minimizes
changes if the RP-Set changes).  The resulting RP is the one that it
uses as the RP for that group.

\section{Standards}

XORP is compliant with the following PIM-SM specification:
\begin{description}
\item{\bf draft-ietf-pim-sm-v2-new-11}.  Protocol Independent
  Multicast - Sparse Mode (PIM-SM): Protocol Specification (Revised).
\item{\bf draft-ietf-pim-sm-bsr-03}.  Bootstrap Router (BSR) Mechanism
  for PIM Sparse Mode.
\end{description}

\section{Configuring PIM-SM}

\subsection{Configuring Multicast Routing on UNIX Systems}

If XORP is to be run on a UNIX-based system, the following steps
must be taken to enable the system for PIM-SM multicast routing
before starting XORP:

\begin{itemize}

  \item Make sure that the underlying system supports multicast routing and
  has PIM-SM kernel support. Unfortunately, there is no trivial guideline how
  to check this, but the following OS-specific information can be useful:

  \begin{itemize}

    \item {\tt DragonFlyBSD}: DragonFlyBSD-1.0 and later.

    \item {\tt FreeBSD}: IPv4 (FreeBSD-4.9 and later, FreeBSD-5.2 and later),
    IPv6 (FreeBSD-4.x and later).

    \item {\tt Linux}: IPv4 (Linux-2.2.11 and later, Linux-2.3.6 and later),
    IPv6 (only with the IPv6 USAGI toolkit after 2005/02/14:
    http://www.linux-ipv6.org/).

    \item {\tt MacOS X}: No multicast routing support (as of MacOS X 10.4.x).

    \item {\tt NetBSD}: IPv4 (NetBSD-3.0 and later), IPv6 (NetBSD-1.5
    and later).

    \item {\tt OpenBSD}: IPv4 (OpenBSD-3.7 and later), IPv6 (OpenBSD-2.7 and
    later).

  \end{itemize}

  \item If necessary, configure the kernel to enable multicast routing and
  PIM-SM:

  \begin{itemize}

    \item {\tt DragonFlyBSD}:

    IPv4: enable the following options in the kernel:

\begin{verbatim}
options         MROUTING                # Multicast routing
options         PIM                     # PIM multicast routing
\end{verbatim}

    IPv6: no kernel options are required.

    \item {\tt FreeBSD}:

    IPv4: enable the following options in the kernel:

\begin{verbatim}
options         MROUTING                # Multicast routing
options         PIM                     # PIM multicast routing
\end{verbatim}

    IPv6: no kernel options are required.

    \item {\tt Linux}:

    IPv4: enable the following options in the kernel:

\begin{verbatim}
CONFIG_IP_MULTICAST=y
CONFIG_IP_MROUTE=y
CONFIG_IP_PIMSM_V2=y
\end{verbatim}

    IPv6: Enable the following options in the kernel:

\begin{verbatim}
CONFIG_IPV6_MROUTE=y
CONFIG_IPV6_PIMSM_V2=y
\end{verbatim}

    \item {\tt NetBSD}:

    IPv4: enable the following options in the kernel:

\begin{verbatim}
options         MROUTING        # IP multicast routing
options         PIM             # Protocol Independent Multicast
\end{verbatim}

    IPv6: no kernel options are required.

    \item {\tt OpenBSD}:

    IPv4: enable the following options in the kernel:

\begin{verbatim}
option          MROUTING        # Multicast router
option          PIM             # Protocol Independent Multicast
\end{verbatim}

    IPv6: no kernel options are required.

  \end{itemize}

  \item Apply additional system configuration (if necessary):

  \begin{itemize}

    \item {\tt DragonFlyBSD}:

    IPv4: Enable IPv4 unicast forwarding:

\begin{verbatim}
sysctl net.inet.ip.forwarding=1
\end{verbatim}

    IPv6: Enable IPv6 unicast forwarding:

\begin{verbatim}
sysctl net.inet6.ip6.forwarding=1
\end{verbatim}

See sysctl.conf(5) for information how to add the sysctl configuration
permanently.

    \item {\tt FreeBSD}:

    IPv4: Enable IPv4 unicast forwarding:

\begin{verbatim}
sysctl net.inet.ip.forwarding=1
\end{verbatim}

    IPv6: Enable IPv6 unicast forwarding:

\begin{verbatim}
sysctl net.inet6.ip6.forwarding=1
\end{verbatim}

See sysctl.conf(5) for information how to add the sysctl configuration
permanently.

    \item {\tt Linux}:

    IPv4: Enable IPv4 unicast forwarding:

\begin{verbatim}
echo 1 > /proc/sys/net/ipv4/ip_forward
\end{verbatim}

    If the unicast Reverse Path Forwarding information is different from the
    multicast Reverse Path Forwarding information, the Return Path Filtering
    should be disabled:

\begin{verbatim}
echo 0 > /proc/sys/net/ipv4/conf/all/rp_filter
\end{verbatim}
    OR
\begin{verbatim}
echo 0 > /proc/sys/net/ipv4/conf/eth0/rp_filter
echo 0 > /proc/sys/net/ipv4/conf/eth1/rp_filter
...
\end{verbatim}

    IPv6: unknown

    \item {\tt NetBSD}: none.

    \item {\tt OpenBSD}:

     Enable multicast routing by adding the following lines to
     {\stt /etc/rc.conf.local} and reboot:

\begin{verbatim}
# Enable multicast routing (see netstart(8) for details).
multicast_host=NO
multicast_router=YES
\end{verbatim}

    IPv4: Enable IPv4 multicast forwarding (for OpenBSD-3.9 and later):

\begin{verbatim}
sysctl net.inet.ip.mforwarding=1
\end{verbatim}

    IPv6: Enable IPv6 multicast forwarding (for OpenBSD-4.0 and later):

\begin{verbatim}
sysctl net.inet6.ip6.mforwarding=1
\end{verbatim}

See sysctl.conf(5) for information how to add the sysctl configuration
permanently.

Note that XORP itself automatically enables the above {\it sysctl}
multicast forwarding flags, hence it is not really necessary to set
them manually.
The flags are described for completness and in case someone needs
better control over multicast forwarding.

  \end{itemize}

\end{itemize}

\subsection{Configuring Multicast Tunnels on UNIX Systems}

All PIM routers need to be directly connected so they can exchange
control messages. Occasionally this might not be possible (\eg if
someone wants to forward multicast traffic to a remote host, and the
routers in the middle are not running PIM).

In that case a tunnel between two remote routers can be used to create
an artificial adjacency.

\begin{itemize}

  \item Creating a GRE tunnel.

  The GRE (Generic Routing Encapsulation) mechanism is described in RFC
  1701 and RFC 1702. It is implemented on a variety of systems and is
  widely used for routing tunnels.

  Below is an example how to configure a GRE tunnel between two Linux
  systems.

\begin{verbatim}
# GRE tunnel between two machines (host 11.11.11.11 and 33.33.33.33)
#
# Physical interfaces:    [11.11.11.11]         [33.33.33.33]
# GRE tunnel:              22.22.22.11<--------->22.22.22.33
#

# ==== host 11.11.11.11 (GRE interface 22.22.22.11)
ip link set gre1 down
ip tunnel del gre1
ip tunnel add gre1 mode gre remote 33.33.33.33 local 11.11.11.11 ttl 127
ip addr add 22.22.22.11/24 peer 22.22.22.33/24 dev gre1
ip link set gre1 up multicast on

# ==== host 33.33.33.33 (GRE interface 22.22.22.33)
ip link set gre1 down
ip tunnel del gre1
ip tunnel add gre1 mode gre remote 11.11.11.11 local 33.33.33.33 ttl 127
ip addr add 22.22.22.33/24 peer 22.22.22.11/24 dev gre1
ip link set gre1 up multicast on
\end{verbatim}


  \item Creating an OpenVPN tunnel.

  OpenVPN (http://openvpn.net/) is free software to create a VPN tunnel.
  It is very popular and works on a variety of systems.
  One of the advantages of OpenVPN is that it is over TCP or UDP,
  hence unlike GRE it does not require special support from NAT devices
  that might be in the middle.

  Below is an example how to configure an OpenVPN tunnel.

\begin{verbatim}
# OpenVPN tunnel between two machines (host 11.11.11.11 and 33.33.33.33)
#
# Physical interfaces:    [11.11.11.11]         [33.33.33.33]
# OpenVPN tunnel:          22.22.22.11<--------->22.22.22.33
#

# ==== host 11.11.11.11 (OpenVPN interface 22.22.22.11)
openvpn --local 11.11.11.11 --remote 33.33.33.33 --ifconfig 22.22.22.11
    22.22.22.33 --dev tun0

# ==== host 33.33.33.33 (OpenVPN interface 22.22.22.33)
openvpn --local 33.33.33.33 --remote 11.11.11.11 --ifconfig 22.22.22.33
    22.22.22.11 --dev tun0
\end{verbatim}

  \item Reverse Path Forwarding (RPF) information setup.

  It is important that the Reverse Path Forwarding (RPF) information
  is set to consider the tunnel for PIM-SM to operate properly.

  On Linux systems the following UNIX command needs to be used to
  disable the Return Path filtering:

\begin{verbatim}
echo 0 > /proc/sys/net/ipv4/conf/all/rp_filter
\end{verbatim}

  In addition, the RPF information must be set appropriately for all
  systems that are suppose to be reached via the tunnel (\eg RPs and
  remote sources). This can be achieved by adding MRIB-specific static
  routing entries in the XORP configuration. For example, if the IP
  address of the other side of the tunnel is 22.22.22.33, then the
  following XORP configuration can be used to specify that the tunnel is
  the default route for all destinations:

\begin{verbatim}
protocols {
    static {
        mrib-route 0.0.0.0/0 {
            next-hop 22.22.22.33
        }
    }
}
\end{verbatim}

\end{itemize}

\newpage
\subsection{Configuration Syntax}

\vspace{0.1in}
\noindent\framebox[\textwidth][l]{\scriptsize
\begin{minipage}{4.5in}
\begin{alltt}
\begin{tabbing}
xx\=xx\=xx\=xx\=xx\=\kill
protocols \{\\
\>pimsm4 \{\\
\>\>targetname: {\it text}\\
\>\>disable: {\it bool}\\
\>\>interface {\it text} \{\\
\>\>\>vif {\it text} \{\\
\>\>\>\>disable: {\it bool}\\
\>\>\>\>enable-ip-router-alert-option-check: {\it bool}\\
\>\>\>\>dr-priority: {\it uint}\\
\>\>\>\>hello-period: {\it uint(1..18724)}\\
\>\>\>\>hello-triggered-delay: {\it uint(1..255)}\\
\>\>\>\>alternative-subnet {\it IPv4}/{\it int(0..32)}\\
\>\>\>\}\\
\>\>\}\\
\>\>interface register\_vif \{\\
\>\>\>vif register\_vif \{\\
\>\>\>\>disable: {\it bool}\\
\>\>\>\}\\
\>\>\}\\
\\
\>\>static-rps \{\\
\>\>\>rp {\it IPv4} \{\\
\>\>\>\>group-prefix {\it IPv4Mcast}/{\it int(4..32)} \{\\
\>\>\>\>\>rp-priority: {\it uint(0..255)}\\
\>\>\>\>\>hash-mask-len: {\it uint(4..32)}\\
\>\>\>\>\}\\
\>\>\>\}\\
\>\>\}\\
\\
\>\>bootstrap \{\\
\>\>\>disable: {\it bool}\\
\>\>\>cand-bsr \{\\
\>\>\>\>scope-zone  {\it IPv4Mcast}/{\it int(4..32)} \{\\
\>\>\>\>\>is-scope-zone: {\it bool}\\
\>\>\>\>\>cand-bsr-by-vif-name: {\it text}\\
\>\>\>\>\>cand-bsr-by-vif-addr: {\it IPv4}\\
\>\>\>\>\>bsr-priority: {\it uint(0..255)}\\
\>\>\>\>\>hash-mask-len: {\it uint(4..32)}\\
\>\>\>\>\}\\
\>\>\>\}\\
\\
\>\>\>cand-rp \{\\
\>\>\>\>group-prefix  {\it IPv4Mcast}/{\it int(4..32)} \{\\
\>\>\>\>\>is-scope-zone: {\it bool}\\
\>\>\>\>\>cand-rp-by-vif-name: {\it text}\\
\>\>\>\>\>cand-rp-by-vif-addr: {\it IPv4}\\
\>\>\>\>\>rp-priority: {\it uint(0..255)}\\
\>\>\>\>\>rp-holdtime: {\it uint(0..65535)}\\
\>\>\>\>\}\\
\>\>\>\}\\
\>\>\}\\
\\
\>\>switch-to-spt-threshold \{\\
\>\>\>disable: {\it bool}\\
\>\>\>interval: {\it uint(3..2147483647)}\\
\>\>\>bytes: {\it uint}\\
\>\>\}\\
\\
{\rm continued overleaf....}
\end{tabbing}
\end{alltt}
\end{minipage}
}
\newpage
\vspace{0.1in}
\noindent\framebox[\textwidth][l]{\scriptsize
\begin{minipage}{4.5in}
\begin{alltt}
\begin{tabbing}
xx\=xx\=xx\=xx\=xx\=\kill
\>\>traceoptions \{\\
\>\>\>flag all \{\\
\>\>\>\>disable: {\it bool}\\
\>\>\>\}\\
\>\>\}\\
\>\}\\
\}
\\
\\
protocols \{\\
\>pimsm6 \{\\
\>\>disable: {\it bool}\\
\>\>interface {\it text} \{\\
\>\>\>vif {\it text} \{\\
\>\>\>\>disable: {\it bool}\\
\>\>\>\>enable-ip-router-alert-option-check: {\it bool}\\
\>\>\>\>dr-priority: {\it uint}\\
\>\>\>\>hello-period: {\it uint(1..18724)}\\
\>\>\>\>hello-triggered-delay: {\it uint(1..255)}\\
\>\>\>\>alternative-subnet  {\it IPv6}/{\it int(0..128)}\\
\>\>\>\}\\
\>\>\}\\
\>\>interface register\_vif \{\\
\>\>\>vif register\_vif \{\\
\>\>\>\>disable: {\it bool}\\
\>\>\>\}\\
\>\>\}\\
\\
\>\>static-rps \{\\
\>\>\>rp {\it IPv6} \{\\
\>\>\>\>group-prefix  {\it IPv6Mcast}/{\it int(8..128)} \{\\
\>\>\>\>\>rp-priority: {\it uint(0..255)}\\
\>\>\>\>\>hash-mask-len: {\it uint(8..128)}\\
\>\>\>\>\}\\
\>\>\>\}\\
\>\>\}\\
\\
\>\>bootstrap \{\\
\>\>\>disable: {\it bool}\\
\>\>\>cand-bsr \{\\
\>\>\>\>scope-zone {\it IPv6Mcast}/{\it int(8..128)} \{\\
\>\>\>\>\>is-scope-zone: {\it bool}\\
\>\>\>\>\>cand-bsr-by-vif-name: {\it text}\\
\>\>\>\>\>cand-bsr-by-vif-addr: {\it IPv6}\\
\>\>\>\>\>bsr-priority: {\it uint(0..255)}\\
\>\>\>\>\>hash-mask-len: {\it uint(8..128)}\\
\>\>\>\>\}\\
\>\>\>\}\\
\\
\>\>\>cand-rp \{\\
\>\>\>\>group-prefix {\it IPv6Mcast}/{\it int(8..128)} \{\\
\>\>\>\>\>is-scope-zone: {\it bool}\\
\>\>\>\>\>cand-rp-by-vif-name: {\it text}\\
\>\>\>\>\>cand-rp-by-vif-addr: {\it IPv6}\\
\>\>\>\>\>rp-priority: {\it uint(0..255)}\\
\>\>\>\>\>rp-holdtime: {\it uint(0..65535)}\\
\>\>\>\>\}\\
\>\>\>\}\\
\>\>\}\\
\\
\>\>switch-to-spt-threshold \{\\
\>\>\>disable: {\it bool}\\
\>\>\>interval: {\it uint(3..2147483647)}\\
\>\>\>bytes: {\it uint}\\
\>\>\}\\
\\
\>\>traceoptions \{\\
\>\>\>flag all \{\\
\>\>\>\>disable: {\it bool}\\
\>\>\>\}\\
\>\>\}\\
\>\}\\
\}
\end{tabbing}
\end{alltt}
\end{minipage}
}
\vspace{0.1in}

\begin{description}
\item{\tt protocols}: this delimits the configuration for all routing
  protocols in the XORP router configuration.  It is mandatory that
  PIM-SM configuration is under the {\stt protocols} node in the
  configuration.
\item{\tt pimsm4}: this delimits the PIM-SM configuration part of the XORP
  router configuration related to IPv4 multicast.
\item{\tt targetname}: this is the name for this instance of PIM-SM for
  IPv4.  It defaults to ``{\stt PIMSM\_4}'', and it is not recommended
  that this default is overridden under normal usage scenarios.
\item{\tt disable}: this takes the value {\stt true} or {\stt false},
  and indicates whether PIM-SM IPv4 multicast routing is currently
  disabled~\footnote{Note
  that prior to XORP Release-1.1, the {\tt enable} flag was used instead of
  {\tt disable}.}.
  This allows multicast to be taken down temporarily without
  removing the configuration.
\item{\tt interface}: this directive specifies that this {\stt
  interface} is to be used for PIM-SM IPv4 multicast routing.  The
  parameter value must be the name of an interface that has been
  configured in the {\stt interfaces} section of the router
  configuration.
\item{\tt vif}: this directive specifies that this {\stt vif}
  on the specified {\stt interface} is to be used for PIM-SM IPv4
  multicast routing.  The parameter value must be the name of a vif
  that has been configured in the {\stt interfaces} section of the
  router configuration.

  A special logical interface called {\stt register\_vif} with a
  special vif called {\stt register\_vif} must be configured if a
  PIM-SM router is to be able to send Register messages to the RP.  In
  general this should {\it always} be configured if the router is to
  support the ASM multicast service model.

  Each {\stt vif} can take the following optional parameters:
\begin{description}
\item{\tt disable}: this takes the value {\stt true} or {\stt false},
  and indicates whether PIM-SM IPv4 multicast routing is currently
  disabled on this interface/vif~\footnote{Note
  that prior to XORP Release-1.1, the {\tt enable} flag was used instead of
  {\tt disable}.}.
\item{\tt enable-ip-router-alert-option-check}: this directive specifies
  whether the router should check that the link-local protocol packets
  received on this interface/vif have the IP Router Alert option (see
  RFC-2213) in them~\footnote{Note that the {\tt
  enable-ip-router-alert-option-check} statement appeared after XORP
  Release-1.1.}. If it is enabled, all link-local protocol packets that
  do not contain the IP Router Alert option will be dropped.
\item{\tt dr-priority}: this directive takes a non-negative integer as
  its parameter giving this router's Designated Router (DR) priority
  for this interface/vif.  The default is 1.  The PIM router on this
  subnet with the highest value of DR priority will become the DR for
  the subnet.
\item{\tt hello-period}: this directive specifies the PIM Hello period
  (in seconds) for this interface/vif~\footnote{Note that the {\tt
  hello-period} statement appeared after XORP Release-1.1.}. It takes a
  non-negative integer in the interval [1..18724].  The default is 30.
  Every {\stt hello-period} seconds the PIM router will transmit a PIM
  Hello message on the interface/vif. If the receivers of the PIM Hello
  message do not receive another Hello message for 3.5 * {\stt
  hello-period} seconds, they will timeout the neighbor state for this
  router.
\item{\tt hello-triggered-delay}: this directive specifies the randomized
  triggered delay of the PIM Hello messages (in seconds) for this
  interface/vif~\footnote{Note that the {\tt hello-triggered-delay} statement
  appeared after XORP Release-1.1.}. It takes a non-negative integer in the
  interval [1..255].  The default is 5. When PIM is enabled on an interface
  or a router first starts, the Hello Timer of that interface is set to a
  random value between 0 and {\tt hello-triggered-delay}.  This prevents
  synchronization of Hello messages if multiple routers are powered on
  simultaneously.
\item{\tt alternative-subnet}: this directive is used to associate
  additional IP subnets with a network interface. The parameter value
  is an IPv4 subnet address in the {\it address/prefix-length} format.

  One use of this directive is to make incoming traffic with a
  non-local source address appear as it is coming from a local
  subnet. Typically, this is needed as a work-around solution when
  uni-directional interfaces such as satellite links are used for
  receiving traffic.  The {\stt alternative-subnet} directive should be
  used with extreme care, because it is possible to create forwarding
  loops.
\end{description}
\item{\tt static-rps}: this delimits the part of the PIM-SM
  configuration used to manually configure PIM RP router information.
  A PIM-SM router must either have some RPs configured as static RPs,
  or it must run the PIM-SM bootstrap mechanism (see the {\stt
  bootstrap} directive).  

  Under the {\stt static-rps} part of the configuration, one or more
  RPs can be configured.  It is important that all routers in a PIM
  domain make the same choice of RP for the same multicast group, so
  generally they should be configured with the same RP information.
\begin{description}
\item{\tt rp}: this specifies the IPv4 address of a router to be a
  static RP.  

  For each RP, the following parameters can be configured:
\begin{description}
\item{\tt group-prefix}: this specifies the range of multicast
  addresses for which the specified router is willing to be the RP.
  The value is in the form of an IP address and prefix-length in the
  {\it address/prefix-length} format.
\begin{description}
\item{\tt rp-priority}: this specifies the priority of the specified
  RP router.  It takes the form of a non-negative integer in the
  interval [0, 255]. Smaller value means higher priority.

  If multiple RP routers are known for a particular multicast group,
  then the one with the most specific {\stt group-prefix} will be
  used.  If more than one router has the same most specific {\stt
  group-prefix}, then the one with the highest {\stt rp-priority} is
  used.  See also {\stt hash-mask-len}.

  The default value is 192.
\item{\tt hash-mask-len}: If multiple routers have the most specific
  {\stt group-prefix} and the same highest {\stt rp-priority}, then to
  balance load, a hash function is used to choose the RP.  However, it
  is usually desirable for closely associated multicast groups to use
  the same RP.  Thus the hash function is only applied to the first
  $n$ bits of the group IP address, ensuring that if two groups have
  the same first $n$ bits, they will hash to the same RP address.  The
  {\stt hash-mask-len} parameter specifies the value of $n$.
  For IPv4 it must be in the interval [4, 32], and defaults to 30 bits.
  Typically its value shouldn't be changed.
  If it is modified then all PIM-SM routers must be configured with the
  same value.
\end{description}
\end{description}
\end{description}
\item{\tt bootstrap}: this delimits the part of the PIM-SM
  configuration used to configure the automatic bootstrap of PIM RP
  router information using the PIM {\it BootStrap Router} mechanism.  A
  PIM-SM router must either run the PIM-SM bootstrap mechanism, or
  have some RPs configured as static RPs (see the {\stt static-rps}
  directive).

  Under the {\stt bootstrap} directive, the following additional
  information can be configured.
\begin{description}
\item{\tt disable}: this takes the value {\stt true} or {\stt false},
  and determines whether or not the router will run the {\stt
  bootstrap} mechanism~\footnote{Note
  that prior to XORP Release-1.1, the {\tt enable} flag was used instead of
  {\tt disable}.}.  The default is {\stt false}.
\item{\tt cand-bsr}: this directive specifies that this router is to
  be a candidate to be the BootStrap Router (BSR) for this PIM-SM domain.
  It will become the BSR only if it wins the BSR election process.

  One or more {\stt scope-zone}s must be specified for a candidate BSR
  router:
\begin{description}
\item{\tt scope-zone}: this directive specifies one multicast group
  prefix for which this router is willing to be BSR. 
 
  For each scope zone, the following information can be specified: 
\begin{description}
\item{\tt is-scope-zone}: this directive takes the value {\stt true}
  or {\stt false}.  When the value is {\stt true}, this indicates that
  this multicast group prefix defines a multicast scope zone.  When
  the value is {\stt false}, this indicates that the group prefix in
  the {\stt scope-zone} directive merely represents a range of
  multicast groups for which this router is willing to be BSR.  The
  default is {\stt false}.
\item{\tt cand-bsr-by-vif-name}: this specifies the name of the {\stt
  vif} whose IP address will be used in the PIM bootstrap messages.  It
  is a mandatory parameter.
\item{\tt cand-bsr-by-vif-addr}: this specifies the address that will be used
  in the PIM bootstrap messages. This address must belong to the vif
  specified by {\tt cand-bsr-by-vif-name}. If it is omitted, a domain-wide
  address (if exists) that belongs to that interface is chosen by the router
  itself~\footnote{Note that the {\tt cand-bsr-by-vif-addr} statement
  appeared after XORP Release-1.1.}.
\item{\tt bsr-priority}: this specifies the BSR priority for this
  router.  It takes a positive integer value in the interval [0, 255],
  which is used in the PIM-SM BSR election process. Larger value means higher
  priority. For each {\stt scope-zone}, the
  candidate bootstrap router with the highest BSR priority will be
  chosen to be BSR. Its default value is 1.
\item{\tt hash-mask-len}: The BSR mechanism announces a list of
  candidate RPs (C-RPs) for each scope zone to the other routers in
  the scope zone.  To balance load, those routers then use a hash
  function to choose the RP for each multicast group from amongst the
  C-RPs.  However, it is usually desirable for closely associated
  multicast groups to use the same RP.  Thus the hash function is only
  applied to the first $n$ bits of the group IP address, ensuring that
  if two groups have the same first $n$ bits, they will hash to the
  same RP address.  Should this router become the BSR for this
  scope-zone, the {\stt hash-mask-len} parameter gives the value of
  $n$ that this router will inform other routers they must use.
  For IPv4 it must be in the interval [4, 32], and defaults to 30 bits.
  Typically its value shouldn't be changed.
  If it is modified then all PIM-SM routers must be configured with the
  same value.
\end{description}
\end{description}
\end{description}
\item{\tt cand-rp}: this directive specifies that this router is to be
  a candidate to be an RP for this PIM-SM domain.  It will become an
  RP only if the BSR chooses it to be.  

  One or more group-prefixes must be specified for this router
  to function as an RP:
\begin{description}
\item{\tt group-prefix}: this specifies the range of multicast
  addresses for which the specified router is willing to be the RP.
  The value is in the form of an IP address and prefix length in the
  {\it address/prefix-length} format.  

  For each {\stt group-prefix}, the following parameters can be
  specified:
\begin{description}
\item{\tt is-scope-zone}: this directive takes the value {\stt true}
  or {\stt false}.  When the value is {\stt true}, this indicates that
  this multicast group prefix defines a multicast scope zone.  When
  the value is {\stt false}, this indicates that the group prefix in
  the {\stt scope-zone} directive merely represents a range of
  multicast groups for which this router is willing to be RP.  The
  default is {\stt false}.
\item{\tt cand-rp-by-vif-name}: this specifies the name of the {\stt
  vif} whose IP address will be used as the RP address if this router
  becomes an RP.  It is a mandatory parameter.
\item{\tt cand-rp-by-vif-addr}: this specifies the address that will be used
  as the RP address if this router becomes an RP.
  This address must belong to the vif
  specified by {\tt cand-rp-by-vif-name}. If it is omitted, a domain-wide
  address (if exists) that belongs to that interface is chosen by the router
  itself~\footnote{Note that the {\tt cand-rp-by-vif-addr} statement
  appeared after XORP Release-1.1.}.
\item{\tt rp-priority}: this specifies the RP priority of this router
  for this {\stt group-prefix}.  It takes the form of a non-negative
  integer in the interval [0, 255].

  If multiple RP routers are known for a particular multicast group,
  then the one with the most specific {\stt group-prefix} will be
  used.  If more than one router has the same most specific {\stt
  group-prefix}, then the one with the highest {\stt rp-priority} is
  used.  See also {\stt hash-mask-len}.

  The default value for {\stt rp-priority} is 1.
\item{\tt rp-holdtime}: this specifies the holdtime that this router
  will advertise when talking to the BSR.  If the BSR has not heard a
  Candidate RP Advertisement from this router for {\stt rp-holdtime}
  seconds, then the BSR will conclude it is dead, and will remove it
  from the set of possible RPs.  It takes the form of a non-negative
  integer in the interval [0, 65535] and its default value is 150 seconds.
\end{description}
\end{description}
\item{\tt switch-to-spt-threshold}: this directive permits the
  specification of a bitrate threshold at a last-hop router or RP
  for switching from the RP Tree to the Shortest-Path Tree.  The
  following parameters can be specified:
\begin{description}
\item{\tt disable}: this takes the value {\stt true} or {\stt false},
  and determines whether bitrate-based switching to the shortest
  path tree is disabled~\footnote{Note
  that prior to XORP Release-1.1, the {\tt enable} flag was used instead of
  {\tt disable}.}.  The default is false.
\item{\tt interval}: this specifies the measurement interval in
  seconds for measuring the bitrate of traffic from a  multicast
  sender~\footnote{Note that prior to XORP Release-1.3, the
  {\tt interval-sec} statement was used instead of {\tt interval}.}.
  The measurement interval should normally not be set too
  small - values greater than ten seconds are recommended.
  It takes the form of a non-negative integer in the interval
  [3, 2147483647] and its default value is 100 seconds.
\item{\tt bytes}: this specifies the maximum number of bytes from a
  multicast sender that can be received in {\stt interval}
  seconds.  If this threshold is exceeded, the router will attempt to
  switch to the shortest-path tree from that multicast sender.
  If the shortest-path switch should happen right after the first packet
  is forwarded, then {\stt bytes} should be set to 0.
\end{description}
\item{\tt traceoptions}: this directive delimits the configuration of
  debugging and tracing options for PIM-SM.
\begin{description}
\item{\tt flag}: this directive is used to specify which tracing
  options are enabled.  Possible parameters are:
\begin{description}
\item{\tt all}: this directive specifies that all tracing
  options should be enabled.  Possible parameters are:
\begin{description}
\item{\tt disable}: this takes the value {\stt true} or {\stt false},
  and disables or enables tracing~\footnote{Note
  that prior to XORP Release-1.1, the {\tt enable} flag was used instead of
  {\tt disable}.}. The default is {\stt false}.
\end{description}
\end{description}
\end{description}
\end{description}

Note that in case of PIM-SM for IPv4 each enabled interface must have a
valid IPv4 address.

The configuration for PIM-SM for IPv6 is identical to PIM-SM for IPv4,
except for the following:
\begin{itemize}
\item The {\stt pimsm6} directive is used in place of the
  {\stt pimsm4} directive.
\item The default value of {\stt targetname} is {\stt ``PIMSM\_6''}
  instead of {\stt ``PIMSM\_4''}.
\item All IP addresses used in the configuration are IPv6 addresses
  instead of IPv4 addresses.
\item The {\stt hash-mask-len} value must be in the interval [8, 128],
  and defaults to 126.
\item Each enabled interface must have a valid link-local and a valid
  domain-wide IPv6 addresses.
\end{itemize}

\newpage
\subsection{Example Configurations}

\vspace{0.1in}
\noindent\framebox[\textwidth][l]{\scriptsize
\begin{minipage}{4.5in}
\begin{alltt}
\begin{tabbing}
xx\=xx\=xx\=xx\=xx\=\kill
protocols \{\\
\>pimsm4 \{\\
\>\>disable: false\\
\>\>interface dc0 \{\\
\>\>\>vif dc0 \{\\
\>\>\>\>disable: false\\
\>\>\>\>/* enable-ip-router-alert-option-check: false */\\
\>\>\>\>/* dr-priority: 1 */\\
\>\>\>\>/* hello-period: 30 */\\
\>\>\>\>/* alternative-subnet 10.40.0.0/16 */\\
\>\>\>\}\\
\>\>\}\\
\>\>interface register\_vif \{\\
\>\>\>vif register\_vif \{\\
\>\>\>\>/* Note: this vif should be always enabled */\\
\>\>\>\>disable: false\\
\>\>\>\}\\
\>\>\}\\
\\
\>\>static-rps \{\\
\>\>\>rp 10.60.0.1 \{\\
\>\>\>\>group-prefix 224.0.0.0/4 \{\\
\>\>\>\>\>/* rp-priority: 192 */\\
\>\>\>\>\>/* hash-mask-len: 30 */\\
\>\>\>\>\}\\
\>\>\>\}\\
\>\>\}\\
\\
\>\>bootstrap \{\\
\>\>\>disable: false\\
\>\>\>cand-bsr \{\\
\>\>\>\>scope-zone 224.0.0.0/4 \{\\
\>\>\>\>\>/* is-scope-zone: false */\\
\>\>\>\>\>cand-bsr-by-vif-name: "dc0"\\
\>\>\>\>\>/* cand-bsr-by-vif-addr: 10.10.10.10 */\\
\>\>\>\>\>/* bsr-priority: 1 */\\
\>\>\>\>\>/* hash-mask-len: 30 */\\
\>\>\>\>\}\\
\>\>\>\}\\
\\
\>\>\>cand-rp \{\\
\>\>\>\>group-prefix 224.0.0.0/4 \{\\
\>\>\>\>\>/* is-scope-zone: false */\\
\>\>\>\>\>cand-rp-by-vif-name: "dc0"\\
\>\>\>\>\>/* cand-rp-by-vif-addr: 10.10.10.10 */\\
\>\>\>\>\>/* rp-priority: 192 */\\
\>\>\>\>\>/* rp-holdtime: 150 */\\
\>\>\>\>\}\\
\>\>\>\}\\
\>\>\}\\
\\
\>\>switch-to-spt-threshold \{\\
\>\>\>/* approx. 1K bytes/s (10Kbps) threshold */\\
\>\>\>disable: false\\
\>\>\>interval: 100\\
\>\>\>bytes: 102400\\
\>\>\}\\
\\
{\rm continued overleaf....}
\end{tabbing}
\end{alltt}
\end{minipage}
}
\newpage
\vspace{0.1in}
\noindent\framebox[\textwidth][l]{\scriptsize
\begin{minipage}{4.5in}
\begin{alltt}
\begin{tabbing}
xx\=xx\=xx\=xx\=xx\=\kill
\>\>traceoptions \{\\
\>\>\>flag all \{\\
\>\>\>\>disable: false\\
\>\>\>\}\\
\>\>\}\\
\>\}\\
\}
\\
\\
protocols \{\\
\>pimsm6 \{\\
\>\>disable: false\\
\>\>interface dc0 \{\\
\>\>\>vif dc0 \{\\
\>\>\>\>disable: false\\
\>\>\>\>/* enable-ip-router-alert-option-check: false */\\
\>\>\>\>/* dr-priority: 1 */\\
\>\>\>\>/* hello-period: 30 */\\
\>\>\>\>/* alternative-subnet 2001:DB8:40:40::/64 */\\
\>\>\>\}\\
\>\>\}\\
\>\>interface register\_vif \{\\
\>\>\>vif register\_vif \{\\
\>\>\>\>/* Note: this vif should be always enabled */\\
\>\>\>\>disable: false\\
\>\>\>\}\\
\>\>\}\\
\\
\>\>static-rps \{\\
\>\>\>rp 2001:DB8:50:50:50:50:50:50 \{\\
\>\>\>\>group-prefix ff00::/8 \{\\
\>\>\>\>\>/* rp-priority: 192 */\\
\>\>\>\>\>/* hash-mask-len: 126 */\\
\>\>\>\>\}\\
\>\>\>\}\\
\>\>\}\\
\\
\>\>bootstrap \{\\
\>\>\>disable: false\\
\>\>\>cand-bsr \{\\
\>\>\>\>scope-zone ff00::/8 \{\\
\>\>\>\>\>/* is-scope-zone: false */\\
\>\>\>\>\>cand-bsr-by-vif-name: "dc0"\\
\>\>\>\>\>/* cand-bsr-by-vif-addr: 2001:DB8:10:10:10:10:10:10 */\\
\>\>\>\>\>/* bsr-priority: 1 */\\
\>\>\>\>\>/* hash-mask-len: 126 */\\
\>\>\>\>\}\\
\>\>\>\}\\
\\
\>\>\>cand-rp \{\\
\>\>\>\>group-prefix ff00::/8 \{\\
\>\>\>\>\>/* is-scope-zone: false */\\
\>\>\>\>\>cand-rp-by-vif-name: "dc0"\\
\>\>\>\>\>/* cand-rp-by-vif-addr: 2001:DB8:10:10:10:10:10:10 */\\
\>\>\>\>\>/* rp-priority: 192 */\\
\>\>\>\>\>/* rp-holdtime: 150 */\\
\>\>\>\>\}\\
\>\>\>\}\\
\>\>\}\\
\\
\>\>switch-to-spt-threshold \{\\
\>\>\>/* approx. 1K bytes/s (10Kbps) threshold */\\
\>\>\>disable: false\\
\>\>\>interval: 100\\
\>\>\>bytes: 102400\\
\>\>\}\\
\\
\>\>traceoptions \{\\
\>\>\>flag all \{\\
\>\>\>\>disable: false\\
\>\>\>\}\\
\>\>\}\\
\>\}\\
\}
\end{tabbing}
\end{alltt}
\end{minipage}
}
\vspace{0.1in}

\section{Monitoring PIM-SM}

All operational commands for monitoring PIM-SM for IPv4 begin with
{\tt show pim}. This section describes those commands in details. All
operational commands for monitoring PIM-SM for IPv6 are similar except that
they begin with {\tt show pim6}.

\subsection{Monitoring PIM-SM Bootstrap Information}

The {\stt show pim bootstrap} command can be used to display
information about PIM bootstrap routers:

\vspace{0.1in}
\noindent\framebox[\textwidth][l]{\scriptsize
\begin{minipage}{6in}
\begin{alltt}
\begin{tabbing}
xxxxxxxxxxxxxxx\=xx\=xx\=xxxxxxxxxxxxxxxxxx\=xx\=xx\=xxxxxxxxxxxx\=xxxxx\=xxx\=xxxxxxx\=\kill
user@hostname> \textbf{show pim bootstrap}\\
Active zones:\\
BSR             \>Pri\>\>LocalAddress    \>Pri\>\>State           \>Timeout\>\>SZTimeout\\
10.4.0.1        \>\>1\>10.2.0.2          \>\>1\>Candidate          \>\>75
  \>\>-1\\
Expiring zones:\\
BSR             \>Pri\>\>LocalAddress    \>Pri\>\>State           \>Timeout\>\>SZTimeout\\
Configured zones:\\
BSR             \>Pri\>\>LocalAddress    \>Pri\>\>State           \>Timeout\>\>SZTimeout\\
10.2.0.2        \>\>1\>10.2.0.2          \>\>1\>Init               \>\>-1        \>\>-1
\end{tabbing}
\end{alltt}
\end{minipage}
}
\vspace{0.1in}

The bootstrap information is separated in three sections:
\begin{itemize}

  \item {\tt Active zones:} This section contains the bootstrap zones that are
  currently in use.

  \item {\tt Expiring zones:} If new bootstrap information is received and it
  replaces the old bootstrap information, the old information is deleted.
  However, if some of the old bootstrap information was not replaced,
  that information is moved to the {\tt Expiring zones} section until
  it times out.

  \item {\tt Configured zones:} This section contains the bootstrap zones that
  are configured on the router.

\end{itemize}

The fields for each entry (in order of appearance) are:
\begin{itemize}

  \item {\tt BSR:} The address of the Bootstrap router for the zone.

  \item {\tt Pri:} The priority of the Bootstrap router.

  \item {\tt LocalAddress:} The local Candidate-BSR address for the zone
  (if the router is configured as a Candidate-BSR).

  \item {\tt Pri:} The local Candidate-BSR priority for the zone
  (if the router is configured as a Candidate-BSR).

  \item {\tt State:} The state of the per-scope-zone state machine.
  In the above example, the router is configured as a Candidate-BSR, 
  but it is not the elected BSR, hence its state is {\tt Candidate}.

  \item {\tt Timeout:} The number of seconds until the BSR times-out.
  If it is -1, it will never timeout.

  \item {\tt SZTimeout:} The number of seconds until the scoped zone
  times-out. If it is -1, it will never timeout.

\end{itemize}

The {\stt show pim bootstrap rps} command can be used to display
information about Candidate RP information received by the Bootstrap
mechanism:

\vspace{0.1in}
\noindent\framebox[\textwidth][l]{\scriptsize
\begin{minipage}{6in}
\begin{alltt}
\begin{tabbing}
xxxxxxxxxxxxxxxx\=xxxx\=xxxx\=x\=xxx\=xxxxxxxxxxxx\=xxxxxxxxxxx\=xxxxxxxxxxxxxx\=\kill
user@hostname> \textbf{show pim bootstrap rps}\\
Active RPs:\\
RP              \>Pri \>Timeout \>\>\>GroupPrefix        \>BSR             \>CandRpAdvTimeout\\
10.4.0.1        \>192     \>\>148 \>\>224.0.0.0/4        \>10.4.0.1
    \>\>-1\\
10.2.0.2        \>192     \>\>148 \>\>224.0.0.0/4        \>10.4.0.1                  \>\>-1\\
Expiring RPs:\\
RP              \>Pri \>Timeout \>\>\>GroupPrefix        \>BSR             \>CandRpAdvTimeout\\
Configured RPs:\\
RP              \>Pri \>Timeout \>\>\>GroupPrefix        \>BSR             \>CandRpAdvTimeout\\
10.2.0.2        \>192      \>\>\>-1 \>224.0.0.0/4        \>10.2.0.2                      \>\>58
\end{tabbing}
\end{alltt}
\end{minipage}
}
\vspace{0.1in}

The Candidate RPs information is separated in three sections:
\begin{itemize}

  \item {\tt Active RPs:} This section contains the Candidate RPs that are
  currently in use.

  \item {\tt Expiring RPs:} If new bootstrap information is received and it
  replaces the old bootstrap information, the old information is deleted.
  However, if some of the old bootstrap information was not replaced,
  the Candidate RPs contained in that information are moved to the {\tt
  Expiring RPs} section until they time-out.

  \item {\tt Configured RPs:} This section contains the Candidate RP
  information that is configured on the router.
\end{itemize}

The fields for each entry (in order of appearance) are:
\begin{itemize}
  \item {\tt RP:} The address of the Candidate RP for the entry.

  \item {\tt Pri:} The priority of the Candidate RP.

  \item {\tt Timeout:} The number of seconds until the Candidate RP times-out.
  If it is -1, it will never timeout.

  \item {\tt GroupPrefix:} The multicast group prefix address the Candidate RP
  is advertising.

  \item {\tt BSR:} The address of the BSR that advertised this Candidate RP.

  \item {\tt CandRpAdvTimeout:} The number of seconds until the Candidate RP
  is advertised to the BSR. This applies only for the Candidate-RPs
  configured in this router.
  If it is -1, the Candidate RP is not advertised to the BSR.

\end{itemize}

\subsection{Monitoring PIM-SM Interface Information}

The {\stt show pim interface} command can be used to display
information about PIM network interfaces:

\vspace{0.1in}
\noindent\framebox[\textwidth][l]{\scriptsize
\begin{minipage}{6in}
\begin{alltt}
\begin{tabbing}
xxxxxxxxxxxxx\=xxxxxxxxx\=xxxxxxx\=xx\=xxxxxxxxx\=xxxxxxx\=xx\=xxxxxxxxxxxxxxx\=xxxxxxxx\=\kill
user@hostname> \textbf{show pim interface}\\
Interface    \>State    \>Mode   \>V \>PIMstate \>Priority \>\>DRaddr          \>Neighbors\\
dc1          \>UP       \>Sparse \>2 \>NotDR           \>\>1 \>10.3.0.2                \>\>1\\
dc2          \>UP       \>Sparse \>2 \>DR              \>\>1 \>10.2.0.2                \>\>0\\
register\_vif \>UP       \>Sparse \>2 \>DR              \>\>1 \>10.3.0.1                \>\>0
\end{tabbing}
\end{alltt}
\end{minipage}
}
\vspace{0.1in}

The fields for each entry (in order of appearance) are:
\begin{itemize}

  \item {\tt Interface:} The name of the interface.

  \item {\tt State:} The state of the interface. E.g. {\tt UP}, {\tt DOWN},
  {\tt DISABLED}, etc.

  \item {\tt Mode:} The PIM mode of the interface. E.g. {\tt Sparse} means
  PIM-SM.

  \item {\tt V:} The protocol version.

  \item {\tt PIMstate:} The protocol state on that interface. E.g., {\tt DR}
  means the router is the Designated Router on that interface.

  \item {\tt Priority:} The configured Designated Router priority on that
  interface.

  \item {\tt DRaddr:} The address of the elected Designated Router on the
  subnet connected to that interface.

  \item {\tt Neighbors:} The number of PIM neighbor routers on that interface.

\end{itemize}

The {\stt show pim interface address} command can be used to display
address information about PIM network interfaces:

\vspace{0.1in}
\noindent\framebox[\textwidth][l]{\scriptsize
\begin{minipage}{6in}
\begin{alltt}
\begin{tabbing}
xxxxxxxxxxxxx\=xxxxxxxxxxxxxxxx\=xxxxxxxxxxxxxxxx\=\kill
user@hostname> \textbf{show pim interface address}\\
Interface    \>PrimaryAddr     \>DomainWideAddr  \>SecondaryAddr  \\
dc1          \>10.3.0.1        \>10.3.0.1                       \\
dc2          \>10.2.0.2        \>10.2.0.2                       \\
register\_vif\>10.3.0.1        \>10.3.0.1                       \\
\end{tabbing}
\end{alltt}
\end{minipage}
}
\vspace{0.1in}

The fields for each entry (in order of appearance) are:
\begin{itemize}

  \item {\tt Interface:} The name of the interface.

  \item {\tt PrimaryAddr:} The primary address on the interface.

  \item {\tt DomainWideAddr:} The domain-wide address on the interface.

  \item {\tt SecondaryAddr:} The first secondary address on the interface
  (if any). If there is more than one secondary address on the interface,
  they are printed one per new line (in the same column).

\end{itemize}

\newpage
\subsection{Monitoring PIM-SM Multicast Routing State Information}

The {\stt show pim join} command can be used to display
information about PIM multicast routing state:

\vspace{0.1in}
\noindent\framebox[\textwidth][l]{\scriptsize
\begin{minipage}{6in}
\begin{alltt}
\begin{tabbing}
xxxx\=xxxxxxxxxxxx\=xxxxxxxxxxxxxxx\=x\=xxxxxxxxxxxxxx\=\kill
user@hostname> \textbf{show pim join}\\
Group           \>\>Source          \>\>RP              \>Flags\\
224.0.1.20      \>\>0.0.0.0         \>\>10.2.0.2        \>WC   \\
    \>Upstream interface (RP):   \>\>register\_vif\\
    \>Upstream MRIB next hop (RP): UNKNOWN\\
    \>Upstream RPF'(*,G):        \>\>UNKNOWN\\
    \>Upstream state:            \>\>Joined \\
    \>Join timer:                \>\>21\\
    \>Local receiver include WC: \>\>.O.\\
    \>Joins RP:                  \>\>...\\
    \>Joins WC:                  \>\>...\\
    \>Join state:                \>\>...\\
    \>Prune state:               \>\>...\\
    \>Prune pending state:       \>\>...\\
    \>I am assert winner state:  \>\>...\\
    \>I am assert loser state:   \>\>...\\
    \>Assert winner WC:          \>\>...\\
    \>Assert lost WC:            \>\>...\\
    \>Assert tracking WC:        \>\>.OO\\
    \>Could assert WC:           \>\>.O.\\
    \>I am DR:                   \>\>.OO\\
    \>Immediate olist RP:        \>\>...\\
    \>Immediate olist WC:        \>\>.O.\\
    \>Inherited olist SG:        \>\>.O.\\
    \>Inherited olist SG\_RPT:   \>\>.O.\\
    \>PIM include WC:            \>\>.O.
\end{tabbing}
\end{alltt}
\end{minipage}
}
\vspace{0.1in}

The fields for each entry (in order of appearance) are:
\begin{itemize}

  \item {\tt Group:} The group address.

  \item {\tt Source:} The source address.

  \item {\tt RP:} The address of the RP for this entry.

  \item {\tt Flags:} The set of flags for this entry. For example:

  \begin{itemize}

    \item {\tt RP:} (*,*,RP) routing entry.

    \item {\tt WC:} (*,G) routing entry.

    \item {\tt SG:} (S,G) routing entry.

    \item {\tt SG\_RPT:} (S,G,rpt) routing entry.

    \item {\tt SPT:} The routing entry has the Shortest-Path Tree flag set.

    \item {\tt DirectlyConnectedS:} The routing entry is for a
    directly-connected source.

  \end{itemize}

\end{itemize}

The remaining lines per entry display various additional information for that
entry. Some of the information below contains a set of network interfaces:
there is either ``.'' or ``O'' per interface (starting with the first
interface according to the {\tt show pim interface} command), and if an
interface is included, it is marked with ``O''.

\begin{itemize}

  \item {\tt Upstream interface (RP):} The name of the upstream interface
  toward the RP.

  \item {\tt Upstream MRIB next hop (RP):} The address of the next-hop router
  (according to the MRIB) toward the RP. In the above example the router
  itself is the RP, hence there is no next-hop router.

  \item {\tt Upstream RPF'(*,G):} The address of the next-hop router
  (according to PIM) toward the RP. Note that this address may be different,
  because it may be affected by PIM-specific events such as PIM Assert
  messages on the upstream interface. In the above example the router
  itself is the RP, hence there is no next-hop router.

  \item {\tt Upstream state:} The upstream state of this entry.

  \item {\tt Join timer:} The number of seconds until the upstream Join
  timer timeout.

  \item {\tt Local receiver include WC:} The set of interfaces that
  have local (*,G) receivers according to the MLD/IGMP module.

  \item {\tt Joins RP:} The set of interfaces that have received (*,*,RP)
  Join.

  \item {\tt Joins WC:} The set of interfaces that have received (*,G) Join.

  \item {\tt Join state:} The set of interfaces that are in Join state.

  \item {\tt Prune state:} The set of interfaces that are in Prune state.

  \item {\tt Prune pending state:} The set of interfaces that are in
  Prune-Pending state.

  \item {\tt I am assert winner state:} The set of interfaces that are in
  Assert Winner state.

  \item {\tt I am assert loser state:} The set of interfaces that are in
  Assert Loser state.

  \item {\tt Assert winner WC:} The set of interfaces for which the
  corresponding (*,G) entry is in Assert Winner state.

  \item {\tt Assert lost WC:} The set of interfaces for which the
  corresponding (*,G) entry has lost the PIM Assert.

  \item {\tt Assert tracking WC:} The set of interfaces for which the
  corresponding (*,G) entry desires to track the PIM Asserts.

  \item {\tt Could assert WC:} The set of interfaces for which the
  corresponding (*,G) entry could trigger a PIM Assert.

  \item {\tt I am DR:} The set of interfaces for which this is the Designated
  Router.

  \item {\tt Immediate olist RP:} The set of interfaces that are included
  in the immediate outgoing interfaces for the corresponding (*,*,RP) entry.

  \item {\tt Immediate olist WC:} The set of interfaces that are included
  in the immediate outgoing interfaces for the corresponding (*,RP) entry.

  \item {\tt Inherited olist SG:} The set of interfaces that are included
  in the outgoing interface list for packets forwarded on (S,G) state taking
  into account (*,*,RP) state, (*,G) state, asserts, etc.

  \item {\tt Inherited olist SG\_RPT:} The set of interfaces that are included
  in the outgoing interface list for packets forwarded
  on (*,*,RP) or (*,G) state taking into account (S,G,rpt) prune state,
  and asserts, etc.

  \item {\tt PIM include WC:} The set of interfaces to which traffic might be
    forwarded because of hosts that are local members on that interface.

\end{itemize}

The {\stt show pim join all} command can be used to display
information about all PIM multicast routing entries including those that may
be created internally by the PIM implementation. Typically, those are the
(*,*,RP) entries that are created per RP for implementation-specific reasons
even though there is no requirement to do so. Currently, this command is
used only for debugging purpose.

\subsection{Monitoring PIM-SM Multicast Routing State Information}

The {\stt show pim mfc} command can be used to display
information about PIM multicast forwarding entries that are installed
in the multicast forwarding engine:

\vspace{0.1in}
\noindent\framebox[\textwidth][l]{\scriptsize
\begin{minipage}{6in}
\begin{alltt}
\begin{tabbing}
xxxx\=xxxxxxxxxxxx\=xxxxxxxxxxxxxxx\=x\=xxxxxxxxxxxxxx\=\kill
user@hostname> \textbf{show pim mfc}\\
Group           \>\>Source          \>\>RP\\
224.0.1.20      \>\>10.4.0.2        \>\>10.2.0.2\\
    \>Incoming interface :   \>\>register\_vif\\
    \>Outgoing interfaces:    \>\>.O.
\end{tabbing}
\end{alltt}
\end{minipage}
}
\vspace{0.1in}

The fields for each entry (in order of appearance) are:
\begin{itemize}

  \item {\tt Group:} The group address.

  \item {\tt Source:} The source address.

  \item {\tt RP:} The address of the RP for this entry.

\end{itemize}

The remaining lines per entry display various additional information for that
entry. Some of the information below contains a set of network interfaces:
there is either ``.'' or ``O'' per interface (starting with the first
interface according to the {\tt show pim interface} command), and if an
interface is included, it is marked with ``O''.

\begin{itemize}

  \item {\tt Incoming interface:} The name of the incoming interface.

  \item {\tt Outgoing interfaces:} The set of outgoing interfaces.

\end{itemize}

\subsection{Monitoring PIM-SM Multicast Routing Information Base}

The {\stt show pim mrib} command can be used to display
information about the Multicast Routing Information Base (MRIB) that is used
by PIM:

\vspace{0.1in}
\noindent\framebox[\textwidth][l]{\scriptsize
\begin{minipage}{6in}
\begin{alltt}
\begin{tabbing}
xxxxxxxxxxxxxxxxxxx\=xxxxxxxxxxxxxxx\=xxxxxxxx\=xxxxxxxxx\=xxxxxxx\=xx\=xx\=x\=xxxx\=\kill
user@hostname> \textbf{show pim mrib}\\
DestPrefix         \>NextHopRouter   \>VifName \>VifIndex \>MetricPref \>\>\>Metric\\
10.2.0.0/24        \>10.2.0.2        \>dc2     \>1                 \>\>\>0      \>\>\>0\\
10.3.0.0/24        \>10.3.0.1        \>dc1     \>0                 \>\>\>0      \>\>\>0\\
10.4.0.0/24        \>10.3.0.2        \>dc1     \>0               \>\>254  \>\>\>65535\\
10.5.0.0/24        \>10.2.0.4        \>dc2     \>1               \>\>254  \>\>\>65535\\
10.6.0.0/24        \>10.2.0.1        \>dc2     \>1               \>\>254  \>\>\>65535
\end{tabbing}
\end{alltt}
\end{minipage}
}
\vspace{0.1in}

The fields for each entry (in order of appearance) are:
\begin{itemize}

  \item {\tt DestPrefix:} The destination prefix address.

  \item {\tt NextHopRouter:} The address of the next-hop router toward the
  destination.

  \item {\tt VifName:} The name of the virtual interface toward the
  destination.

  \item {\tt VifIndex:} The virtual interface index of the virtual interface
  toward the destination.

  \item {\tt MetricPref:} The metric preference of the entry.

  \item {\tt Metric:} The routing metric of the entry.

\end{itemize}


\subsection{Monitoring PIM-SM Multicast Routing Information Base}

The {\stt show pim neighbors} command can be used to display
information about the PIM neighbor routers:

\vspace{0.1in}
\noindent\framebox[\textwidth][l]{\scriptsize
\begin{minipage}{6in}
\begin{alltt}
\begin{tabbing}
xxxxxxxxxxxxx\=xxxxxxxxx\=xx\=xxxxxxxxxxxxxxxx\=xx\=xxxxxxx\=xxxxx\=xxxx\=xxxxx\=\kill
user@hostname> \textbf{show pim neighbors}\\
Interface    \>DRpriority \>\>NeighborAddr    \>V \>Mode   \>Holdtime \>\>Timeout\\
dc1                   \>\>1 \>10.3.0.2        \>2 \>Sparse      \>\>105      \>\>97
\end{tabbing}
\end{alltt}
\end{minipage}
}
\vspace{0.1in}

The fields for each entry (in order of appearance) are:
\begin{itemize}

  \item {\tt Interface:} The name of the interface toward the neighbor:

  \item {\tt DRpriority:} The DR priority of the neighbor.

  \item {\tt NeighborAddr:} The primary address of the neighbor.

  \item {\tt V:} The PIM protocol version used by the neighbor.

  \item {\tt Mode:} The PIM mode of the neighbor. E.g. {\tt Sparse} means
  PIM-SM.

  \item {\tt Holdtime:} The PIM Hello holdtime of the neighbor (in seconds).

  \item {\tt Timeout:} The number of seconds until the neighbor timeout (in
  case no more PIM Hello messages are received from it).

\end{itemize}


\subsection{Monitoring PIM-SM Candidate RP Set Information}

The {\stt show pim rps} command can be used to display
information about the Candidate RP Set:

\vspace{0.1in}
\noindent\framebox[\textwidth][l]{\scriptsize
\begin{minipage}{6in}
\begin{alltt}
\begin{tabbing}
xxxxxxxxxxxxxxxx\=xxxxxxxxxx\=xxxx\=xxxxx\=xxxx\=xxxx\=xxxx\=xxxxxxxxxxx\=xx\=\kill
user@hostname> \textbf{show pim rps}\\
RP              \>Type      \>Pri \>Holdtime \>\>Timeout \>\>ActiveGroups \>\>GroupPrefix\\
10.4.0.1        \>bootstrap \>192      \>\>150     \>\>134            \>\>0 \>224.0.0.0/4\\
10.2.0.2        \>bootstrap \>192      \>\>150     \>\>134            \>\>1 \>224.0.0.0/4
\end{tabbing}
\end{alltt}
\end{minipage}
}
\vspace{0.1in}

The fields for each entry (in order of appearance) are:
\begin{itemize}

  \item {\tt RP:} The address of the Candidate RP.

  \item {\tt Type:} The type of the mechanism that provided the Candidate RP.

  \item {\tt Pri:} The priority of the Candidate RP.

  \item {\tt Holdtime:} The holdtime (in number of seconds) of the Candidate
  RP.

  \item {\tt Timeout:} The number of seconds until the Candidate RP timeout.
  If it is -1, the Candidate RP will never timeout.

  \item {\tt ActiveGroups:} The number of groups that use this Candidate RP.

  \item {\tt GroupPrefix:} The multicast group prefix address for this
  Candidate RP.

\end{itemize}


\subsection{Monitoring PIM-SM Scope Zone Information}

The {\stt show pim scope} command can be used to display
information about the PIM scope zones:

\vspace{0.1in}
\noindent\framebox[\textwidth][l]{\scriptsize
\begin{minipage}{6in}
\begin{alltt}
\begin{tabbing}
xxxxxxxxxxxxxxxxxxxxxxxxxxxxxxxxxxxxxxxxxxxx\=\kill
user@hostname> \textbf{show pim scope}\\
GroupPrefix                                 \>Interface\\
225.1.2.0/24                                \>dc1
\end{tabbing}
\end{alltt}
\end{minipage}
}
\vspace{0.1in}

The fields for each entry (in order of appearance) are:
\begin{itemize}

  \item {\tt GroupPrefix:} The multicast group prefix address of the scoped
  zone.

  \item {\tt Interface:} The name of the interface that is the boundary
  of the scoped zone.

\end{itemize}

Note that currently (March 2007), configuring multicast scoped zones is not
supported. This feature should be added in the future.
