%
% $XORP: xorp/docs/user_manual/bgp.tex,v 1.12 2005/04/11 21:21:13 pavlin Exp $
%

\chapter{BGP}
\label{bgp}
\section{BGP Terminology and Concepts}

BGP is the Border Gateway Protocol, which is the principal inter-domain
routing protocol in the Internet.  BGP version 4 is specified in RFC
1771.  However, RFC 1771 is shortly to be replaced by a much improved
newer RFC, and XORP BGP is compliant with the new RFC.  Earlier
versions of BGP are now considered historic.  XORP implements what is
known as BGP4+.  This is the core BGP-4 protocol, plus the
multiprotocol extensions needed to route IPv6 traffic and to provide
separate topology information for multicast routing protocols to that
used for unicast routing.

A complete description of BGP is outside the scope of this manual, but
we will mention a few of the main concepts.  

\subsection{Key BGP Concepts}

The main concept used in BGP is that of the Autonomous System, or AS
for short.  An AS corresponds to a routing domain that is under one
administrative authority, and which implements its own routing
policies.  BGP is used in two different ways:
\begin{itemize}
\item EBGP is used to exchange routing information between routers
that are in different ASes.
\item IBGP is used to exchange routing information between routers
that are in the same AS.  Typically these routes were originally
learned from EBGP.
\end{itemize}
Each BGP route carries with it an AS Path, which essentially records
the autonomous systems through which the route has passed between the
AS where the route was originally advertised and the current AS.  When
a BGP router passes a route to a router in a neighboring AS, it
prepends its own AS number to the AS path.  The AS path is used to
prevent routes from looping, and also can be used in policy filters to
decide whether or not to accept a route.

When a route reaches a router over an EBGP connection, the router
first decides if this is the best path to the destination, based on a
complex decision process and local policy configuration.  If the route
is the best path, the route is passed on to all the other BGP routers
in the same domain using IBGP connections, as well as on to all the
EBGP peers (as allowed by policy).

When a router receives a route from an IBGP peer, if the router decides
this route is the best route to the destination, then it will pass the
route on to its EBGP peers, but it will not normally pass the route
onto another IBGP peer.  This prevents routing information looping
within the AS, but it means that by default every BGP router in a
domain must be peered with every other BGP router in the domain.

Of course such a full mesh of configured BGP peerings does not scale
well to large domains, so two techniques can be used to improve
scaling:
\begin{itemize}
\item Confederations.
\item Route Reflectors.
\end{itemize}
As of the 1.0 release, XORP BGP does not support confederations, and
does not support being used as a route reflector, although it can be
used as a route reflector client.  This capability will be added soon.

BGP peerings are conducted over TCP connections which must be manually
configured.  A connection is an IBGP peering if both routers are
configured to be in the same AS; otherwise it is an EBGP peering.

Routers typically have multiple IP addresses, with at least one for
each interface, and often an additional routable IP address associated
with the loopback interface\footnote{Note: 127.0.0.1 is {\it not} routable.}.
When configuring an IBGP connection, it is good practice to set up the
peering to be between the IP addresses on the loopback interfaces.
This makes the connection independent of the state of any particular
interface.  However, most EBGP peerings will be configured using the
IP address of the router that is directly connected to the EBGP peer
router.  Thus if the interface to that peer goes down, the peering
session will also go down, causing the routing to correctly fail over
to an alternative path.

\section{Standards}

XORP BGP complies with the following standards:
\begin{description}
\item{\bf draft-ietf-idr-bgp4-22.txt}: BGP-4 Specification (obsoletes RFC 1771)
\item{\bf RFC 1657}: Definitions of Managed Objects for the Fourth Version
     of the Border Gateway Protocol (BGP-4) using SMIv2.
\item{\bf RFC 2858}: Multiprotocol Extensions for BGP-4.
\item{\bf RFC 2545}: Use of BGP-4 Multiprotocol Extensions for IPv6
     Inter-Domain Routing.
\end{description}

\noindent We also have limited support for:
\begin{description}
\item{\bf RFC 1997}: BGP Communities Attribute.
\end{description}

\newpage
\section{Configuring BGP}

\subsection{Configuration Syntax}

The configuration syntax for XORP BGP is given below.

\vspace{0.1in}
\noindent\framebox[\textwidth][l]{\scriptsize
\begin{minipage}{6in}
\begin{alltt}
\begin{tabbing}
xx\=xx\=xx\=xx\=xx\=\kill
protocols \{\\
\>bgp \{\\
\>\>targetname: {\it text}\\
\>\>bgp-id: {\it IPv4}\\
\>\>local-as: {\it int(1..65535)}\\
\\
\>\>peer {\it text} \{\\
\>\>\>local-ip: {\it IPv4}\\
\>\>\>as: {\it int(1..65535)}\\
\>\>\>next-hop: {\it IPv4}\\
\>\>\>local-port: {\it int(1..65535)}\\
\>\>\>peer-port: {\it int(1..65535)}\\
\>\>\>holdtime: {\it uint}\\
\>\>\>disable: {\it bool}\\
\>\>\>ipv4-unicast: {\it bool}\\
\>\>\>ipv4-multicast: {\it bool}\\
\>\>\>ipv6-unicast: {\it bool}\\
\>\>\>ipv6-multicast: {\it bool}\\
\>\>\}\\
\\
\>\>network4 {\it IPv4}/{\it int(1..32)} \{\\
\>\>\>next-hop: {\it IPv4}\\
\>\>\>unicast: {\it bool}\\
\>\>\>multicast: {\it bool}\\
\>\>\}\\
\\
\>\>network6 {\it IPv6}/{\it int(1..128)} \{\\
\>\>\>next-hop: {\it IPv6}\\
\>\>\>unicast: {\it bool}\\
\>\>\>multicast: {\it bool}\\
\>\>\}\\
\}
\end{tabbing}
\end{alltt}
\end{minipage}
}
\vspace{0.1in}

\noindent
The configuration parameters are used as follows:
\begin{description}
\item{\tt protocols}: this delimits the configuration for all routing
  protocols in the XORP router configuration.  It is mandatory that
  BGP configuration is under the {\stt protocols} node in the
  configuration.
\item{\tt bgp}: this delimits the BGP configuration part of the XORP
  router configuration.
\item{\tt targetname}: this is the name for this instance of BGP.  It
  defaults to ``{\stt bgp}'', and it is not recommended that this
  default is overridden under normal usage scenarios.
\item{\tt bgp-id}: this is the BGP identifier for the BGP instance on
  this router.  It is typically set to one of the router's IP
  addresses, and it is normally required that this is globally unique.
  The required format of the BGP ID is a dotted-decimal IPv4 address,
  as mandated by the BGP specification.  This is required even if the
  router only supports IPv6 forwarding.
\item{\tt local-as}: this is the autonomous system number for the AS
  in which this router resides.  Any peers of this router must be
  configured to know this AS number - if there is a mismatch, a
  peering will not be established.  It is a 16-bit integer.
\item{\tt peer}: this delimits the configuration of a BGP peering
  association with another router.  Most BGP routers will have
  multiple peerings configured.  The {\stt peer} directive takes a
  parameter which is the peer identifier for the peer router. This
  peer identifier should normally be the IPv4 unicast address of the
  router we are peering with.  The syntax allows it to be the domain
  names of the peer router for convenience, but this is {\it not}
  recommended in production settings.  

  For IBGP peerings the peer identifier will normally be an IP address
  bound to the router's loopback address, so it is not associated with
  a specific interface, meaning that the peering will not go down if a
  single internal interface fails.  

  For EBGP peerings, the peer identifier will normally be the IP
  address of the peer router on the interface over which we wish to
  exchange traffic, so that if the interface goes down, the peering
  will drop.

  For each configured {\stt peer}, the following
  configuration options can be specified:
\begin{description}
\item{\tt local-ip}: This is the IP address of this router that we
  will use for BGP connections to this peer.  It is mandatory to
  specify, and must be the same as the IP address configured on the
  peer router for this peering.
\item{\tt as}: this gives the AS number of this peer.  This must match
  the AS number that the peer itself advertises to us, or the BGP
  peering will not be established.  It is a 16-bit integer, and is
  mandatory to specify.
\item{\tt next-hop}: this is the IPv4 address that will be sent as the
  nexthop router address in routes that we send to this peer.
  Typically this is only specified for EBGP peerings.
\item{\tt next-hop6}: this is the IPv6 address that will be sent as the
  nexthop router address in routes that we send to this peer.
  Typically this is only specified for EBGP peerings.
\item{\tt local-port}: by default, BGP establishes its BGP connections
  over a TCP connection between port 179 on the local router and port
  179 on the remote router.  The local port for this peering can be
  changed by modifying this attribute.  This must be the same as the
  corresponding {\stt remote-port} on the remote peer router or a
  connection will not be established.
\item{\tt peer-port}: The port for this peering on the remote router
  can be changed by modifying this attribute. See also: {\stt
  local-port}.
\item{\tt holdtime}: This is the holdtime BGP should use when
  negotiating the connection with this peer.  If no message is
  received from a BGP peer during the negotiated holdtime, the
  peering will be shut down.
\item{\tt disable}: This takes the value {\stt true} or {\stt false},
  and indicates whether the peering is currently disabled.  This allows
  a peering to be taken down temporarily without removing the
  configuration~\footnote{Note that prior to XORP Release-1.1, the {\tt
  enable} flag was used instead of {\tt disable}.}.
\item{\tt ipv4-unicast}: This takes the value {\stt true} or {\stt false},
  and specifies whether BGP should
  negotiate multiprotocol support with this peer to allow IPv4 unicast
  routes to be exchanged. It is enabled by default.
\item{\tt ipv4-multicast}: This takes the value {\stt true} or {\stt false},
  and specifies whether BGP should
  negotiate multiprotocol support with this peer to allow separate
  routes to be used for IPv4 unicast and IPv4 multicast.  Normally
  this would only be enabled if PIM-SM multicast routing is running on
  the router.
\item{\tt ipv6-unicast}: This takes the value {\stt true} or {\stt false},
  and specifies whether BGP should
  negotiate multiprotocol support with this peer to allow IPv6 unicast
  routes to be exchanged.
\item{\tt ipv6-multicast}: This takes the value {\stt true} or {\stt false},
  and specifies whether BGP should
  negotiate multiprotocol support with this peer to allow IPv6
  multicast routes to be exchanged separately from IPv6 unicast
  routes.  It is possible to enable IPv6 multicast support without
  enabling IPv6 unicast support.
\end{description}
\item{\tt network4}: this specifies an IPv4 route to be advertised
  into the BGP routing, originating from this router.  The parameter
  is an IPv4 subnet in the format {\it address/prefix-length}.  For
  example {\stt 10.0.0.0/24}.
\begin{description}
\item{\tt next-hop}: this is the IPv4 address of the next-hop router
  to be used in BGP advertisements for this route.  Typically it will
  be one of the router's IP addresses.
\item{\tt unicast}: this takes the value {\stt true} or {\stt false}
and specifies whether this route should be advertised to be used to
route unicast packets.
\item{\tt multicast}: this takes the value {\stt true} or {\stt false}
and specifies whether this route should be advertised to be used to
route multicast packets.  Either {\stt unicast} or {\stt multicast} or
both must be true.
\end{description}

\item{\tt network6}: this specifies an IPv6 route to be advertised
  into the BGP routing, originating from this router.  The parameter
  is an IPv6 subnet in the format {\it address/prefix-length}.  For
  example {\stt 10:0:0:10::/64}.
\end{description}

\subsection{Example Configurations}
\vspace{0.1in}
\noindent\framebox[\textwidth][l]{\scriptsize
\begin{minipage}{6in}
\begin{alltt}
\begin{tabbing}
xx\=xx\=xx\=xx\=xx\=\kill
protocols \{\\
\>bgp \{\\
\>\>bgp-id: 128.16.32.1\\
\>\>local-as: 45678\\
\\
\>\>peer 192.168.150.1 \{\\
\>\>\>local-ip: 128.16.64.4\\
\>\>\>as: 34567\\
\>\>\>next-hop: 128.16.64.4\\
\>\>\>holdtime: 120\\
\\
\>\>\>/* IPv4 unicast is enabled by default  */\\
\>\>\>ipv4-unicast: true\\
\\
\>\>\>/* Optionally enable other AFI/SAFI combinations */\\
\>\>\>ipv4-multicast: true\\
\>\>\>ipv6-unicast: true\\
\>\>\>ipv6-multicast: true\\
\>\>\}\\
\\
\>\>/* Originate IPv4 Routes */\\
\>\>network4 128.16.16/24 \{\\
\>\>\>next-hop: 128.16.64.1\\
\>\>\>unicast: true\\
\>\>\>multicast: true\\
\>\>\}\\
\\
\>\>/* Originate IPv6 Routes */\\
\>\>network6 10:10:10:10::/64 \{\\
\>\>\>next-hop: 10:10:10:10:10:10:10:10\\
\>\>\>unicast: true\\
\>\>\>multicast: false\\
\>\>\}\\
\>\}\\
\}
\end{tabbing}
\end{alltt}
\end{minipage}
}
\vspace{0.1in}

This configuration is from a BGP router in AS 45678.  The router has a
BGP identifier of 128.16.32.1, which will normally be one of the router's
IP addresses.

This router has only one BGP peering configured, with a peer on IP
address 192.168.150.1.  This peering is an EBGP connection because the
peer is in a different AS (34567).  This router's IP address used for
this peering is 128.16.64.4, and the router is also configured to set
the next hop router field in routes it advertises to the peer to be
128.16.64.4.  Setting local-ip and next-hop to be the same is common
for EBGP connections.  The holdtime for the peering is configured to
be 120 seconds, but the precise value of the holdtime actually used
depends on negotiation with the peer.  In addition to IPv4 unicast
routing, which is enabled by default, this peering is configured to
allow the sending an receiving of IPv4 multicast routes and IPv6
unicast routes.

This router is also configured to {\it originate} routing
advertisements for two subnets.  These subnets might be directly
connected, or might be reachable via IGP routing.

The first advertisement this router originates is for subnet
128.16.16/24, reachable via both unicast and multicast.  The nexthop
specified in 128.16.64.1, and this must be reachable via other routes
in the routing table, or this advertisement will not be made.  If this
router had any IBGP peerings, then the BGP route advertised to those
peers would indicate that 128.16.16/24 was reachable via next hop
128.16.64.1. However in this case the only peering is an EBGP peering,
and the next hop in {\it all} routes sent to that peer is set to
128.16.64.4 according to the {\stt nexthop} directive for the peering.

The second advertisement is for an IPv6 route, configured to be usable
only by IPv6 unicast traffic.

\newpage
\section{Monitoring BGP}

On a router running BGP, the BGP routing state can be displayed using
the {\stt show bgp} operational-mode command.  Information is available
about the status of BGP peerings and about the routes received and
used.  In the 1.0 release, the set of commands is fairly limited, and
will be increased in future releases to provide better ways to display
subsets of this information.

As always, command completion using $<$TAB$>$ or ? will display the
available sub-commands and parameters:

\vspace{0.1in}
\noindent\framebox[\textwidth][l]{\scriptsize
\begin{minipage}{6in}
\begin{alltt}
\begin{tabbing}
xx\=xxxxxxxxxxxxxxxx\=\kill
user@hostname> \textbf{show bgp ?}\\
Possible completions:\\
\><[Enter]>\>Execute this command\\
\>peers\>Show BGP peers info\\
\>routes\>Print BGP routes\\
\>|\>Pipe through a command\\
\end{tabbing}
\end{alltt}
\end{minipage}
}

The {\stt show bgp peers} command will display information about the
BGP peerings that have been configured.  It supports the optional
parameter {\stt detail} to give a lot more information:

\vspace{0.1in}
\noindent\framebox[\textwidth][l]{\scriptsize
\begin{minipage}{6in}
\begin{alltt}
\begin{tabbing}
xx\=xxxxxxxxxxxxxxxx\=\kill
user@hostname> \textbf{show bgp peers ?}\\
Possible completions:\\
\><[Enter]>\>Execute this command\\
\>detail\>Show detailed BGP peers info\\
\>|\>Pipe through a command\\
\end{tabbing}
\end{alltt}
\end{minipage}
}

By itself, {\stt show bgp peers} provides a short list of the peerings
that are configured, irrespective of whether the peering is
in established state or not:

\vspace{0.1in}
\noindent\framebox[\textwidth][l]{\scriptsize
\begin{minipage}{6in}
\begin{alltt}
\begin{tabbing}
xx\=xxxxxxxxxxxxxxxx\=\kill
user@hostname> \textbf{show bgp peers} \\
Peer 1: local 192.150.187.112/179 remote 69.110.224.158/179\\
Peer 2: local 192.150.187.112/179 remote 192.150.187.2/179\\
Peer 3: local 192.150.187.112/179 remote 192.150.187.78/179\\
Peer 4: local 192.150.187.112/179 remote 192.150.187.79/179\\
Peer 5: local 192.150.187.112/179 remote 192.150.187.109/179\\
\end{tabbing}
\end{alltt}
\end{minipage}
}

\newpage
The command {\stt show bgp peers detail} will give a large amount of
information about all the peerings:  

\vspace{0.1in}
\noindent\framebox[\textwidth][l]{\scriptsize
\begin{minipage}{6in}
\begin{alltt}
\begin{tabbing}
xx\=xx\=xx\=xx\=xx\=\kill
user@hostname> \textbf{show bgp peers detail}\\
Peer 1: local 192.150.187.112/179 remote 69.110.224.158/179\\
\>Peer ID: none\\
\>Peer State: ACTIVE\\
\>Admin State: START\\
\>Negotiated BGP Version: n/a\\
\>Peer AS Number: 65014\\
\>Updates Received: 0,  Updates Sent: 0\\
\>Messages Received: 0,  Messages Sent: 0\\
\>Time since last received update: n/a\\
\>Number of transitions to ESTABLISHED: 0\\
\>Time since last in ESTABLISHED state: n/a\\
\>Retry Interval: 120 seconds\\
\>Hold Time: n/a,  Keep Alive Time: n/a\\
\>Configured Hold Time: 120 seconds,  Configured Keep Alive Time: 40 seconds\\
\>Minimum AS Origination Interval: 0 seconds\\
\>Minimum Route Advertisement Interval: 0 seconds\\
\\
Peer 2: local 192.150.187.112/179 remote 192.150.187.2/179\\
\>Peer ID: 192.150.187.2\\
\>Peer State: ESTABLISHED\\
\>Admin State: START\\
\>Negotiated BGP Version: 4\\
\>Peer AS Number: 64999\\
\>Updates Received: 52786,  Updates Sent: 28\\
\>Messages Received: 52949,  Messages Sent: 189\\
\>Time since last received update: 2 seconds\\
\>Number of transitions to ESTABLISHED: 17\\
\>Time since last entering ESTABLISHED state: 6478 seconds\\
\>Retry Interval: 120 seconds\\
\>Hold Time: 120 seconds,  Keep Alive Time: 40 seconds\\
\>Configured Hold Time: 120 seconds,  Configured Keep Alive Time: 40 seconds\\
\>Minimum AS Origination Interval: 0 seconds\\
\>Minimum Route Advertisement Interval: 0 seconds\\
\\
\end{tabbing}
\end{alltt}
\end{minipage}
} 

\vspace{0.1in}
The most important piece of information is typically whether or not
the peering is in ESTABLISHED state, indicating that the peering is up
and capable of exchanging routes.  ACTIVE state means that the peering
is configured to be up on this router, but for some reason the peering
is not currently up.  Typically this is because the remote peer is
unreachable, or because no BGP instance is running on the remote peer.

\newpage
The {\stt show bgp routes} command displays the routes received by BGP
from its peers.  On a router with a full BGP routing table (140000
routes as of July 2004) this command will produce a large amount of
output:

\vspace{0.1in}
\noindent\framebox[\textwidth][l]{\scriptsize
\begin{minipage}{6in}
\begin{alltt}
\begin{tabbing}
xxx\=xxxxxxxxxxxxxxxx\=xxxxxxxxxxxxxxxx\=xxxxxxxxxxxxxxx\=xx\=\kill
user@hostname> \textbf{show bgp routes}\\
Status Codes: * valid route, > best route\\
Origin Codes: i IGP, e EGP, ? incomplete\\
\\
\>Prefix\>Nexthop\>Peer\>AS Path\\
\>------\>-------\>----\>-------\\
*>\>3.0.0.0/8\>192.150.187.2\>192.150.187.2\>16694 25 2152 3356 7018 80 i\\
*>\>4.17.225.0/24\>192.150.187.2\>192.150.187.2\>16694 25 2152 11423 209 701 11853 6496 i\\
*>\>4.17.226.0/23\>192.150.187.2\>192.150.187.2\>16694 25 2152 11423 209 701 11853 6496 i\\
*>\>4.17.251.0/24\>192.150.187.2\>192.150.187.2\>16694 25 2152 11423 209 701 11853 6496 i\\
*>\>4.17.252.0/23\>192.150.187.2\>192.150.187.2\>16694 25 2152 11423 209 701 11853 6496 i\\
*>\>4.21.252.0/23\>192.150.187.2\>192.150.187.2\>16694 25 2152 11423 209 701 6389 8063 19198 i\\
*>\>4.23.180.0/24\>192.150.187.2\>192.150.187.2\>16694 25 2152 11423 209 3561 6128 30576 i\\
*>\>4.36.200.0/21\>192.150.187.2\>192.150.187.2\>16694 25 2152 174 3561 14742 11854 14135 i\\
*>\>4.78.0.0/21\>192.150.187.2\>192.150.187.2\>16694 25 2152 11423 209 3561 6347 23071 22938 i\\
*>\>4.78.32.0/21\>192.150.187.2\>192.150.187.2\>16694 25 2152 174 3491 29748 i\\
*>\>4.0.0.0/8\>192.150.187.2\>192.150.187.2\>16694 25 2152 3356 i\\
...
\end{tabbing}
\end{alltt}
\end{minipage}
}
\vspace{0.1in}

The format of the output is one route per line.  On each line:
\begin{itemize}
\item A status code is displayed, showing whether the route is valid,
  and whether it was the best BGP route this router has received.  A
  route is valid if the nexthop is reachable and it isn't filtered by
  the inbound BGP filters.
\item The network prefix for which the route applies is listed in the
  form {\stt 4.17.226.0/23}.  This indicates the base address for the
  network (address {\stt 4.17.226.0}), and the prefix length ({\stt 23} bits).
  Thus this route applies for addresses {\stt 4.17.226.0} to {\stt
  4.17.227.255} inclusive.
\item The nexthop is the IP address of the intermediate router towards
  which packet destined for the network prefix should be sent.  In
  this example all the displayed routes have the same nexthop.
\item The peer is the IP address of the BGP router which sent us this
  route.  The nexthop and the peer need not the the same (they often
  aren't with IBGP peerings for example) but in all the routes in this
  example they are the same.
\item The AS path is listed next.  This lists the AS numbers of the
  autonomous systems that the route has traversed to reach our
  router.  The AS at the left end of the path is the one nearest to
  our router and the one at the right end of the path is usually the
  AS number of the route's originator.  
\item Finally, whether the route's origin is from an IGP ({\stt i}),
  from EGP ({\stt e}, mostly obsolete), or incomplete ({\stt ?}) is
  listed.
\end{itemize}

\subsection{BGP MIB}

XORP includes SNMP support for BGP, though the BGP-4 MIB defined in
RFC 1657.
