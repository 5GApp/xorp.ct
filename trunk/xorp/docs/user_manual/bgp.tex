\chapter{BGP}

\section{Terminology and Concepts}

BGP: Border Gateway Protocol

concept of a path vector protocol.

ref to RFC 1771 and new draft

BGP4 vs earlier BGPs
BGP4+ (multiprotocol extensions)

\section{Configuring BGP}

\subsection{Configuration Syntax}

\vspace{0.1in}
\noindent\framebox[\textwidth][l]{\scriptsize
\begin{minipage}{6in}
\begin{alltt}
\begin{tabbing}
xx\=xx\=xx\=xx\=xx\=\kill
protocols \{\\
\>bgp \{\\
\>\>targetname: {\it text}\\
\>\>bgp-id: {\it IPv4}\\
\>\>local-as: {\it int(1..65535)}\\
\\
\>\>peer {\it text} \{\\
\>\>\>local-ip: {\it IPv4}\\
\>\>\>as: {\it int(1..65535)}\\
\>\>\>next-hop: {\it IPv4}\\
\>\>\>local-port: {\it int(1..65535)}\\
\>\>\>peer-port: {\it int(1..65535)}\\
\>\>\>holdtime: {\it uint}\\
\>\>\>enabled: {\it bool}\\
\>\>\>enable-ipv4-multicast\\
\>\>\>enable-ipv6-unicast\\
\>\>\>enable-ipv6-multicast\\
\>\>\}\\
\\
\>\>network4 {\it IPv4}/{\it int(1..32)} \{\\
\>\>\>next-hop: {\it IPv4}\\
\>\>\>unicast: {\it bool}\\
\>\>\>multicast: {\it bool}\\
\>\>\}\\
\\
\>\>network6 {\it IPv6}/{\it int(1..128)} \{\\
\>\>\>next-hop: {\it IPv6}\\
\>\>\>unicast: {\it bool}\\
\>\>\>multicast: {\it bool}\\
\>\>\}\\
\}
\end{tabbing}
\end{alltt}
\end{minipage}
}
\vspace{0.1in}

\noindent
The configuration parameters are used as follows:
\begin{description}
\item{\tt protocols}: this delimits the configuation for all routing
  protocols in the XORP router configuration.  It is mandatory that
  BGP configuration is under the {\stt protocols} node in the
  configuration.
\item{\tt bgp}: this delimits the BGP configuration part of the XORP
  router configuration.
\item{\tt targetname} this is the name for this instance of BGP.  It
  defaults to ``{\stt bgp}'', and it is not recommended that this
  default is overridden under normal usage scenarios.
\item{\tt bgp-id}: this is the BGP identifier for the BGP instance on
  this router.  It is typically set to one of the router's IP
  addresses, and it is normally required that this is globally unique.
  The required format of the BGP ID is a dotted-decimal IPv4 address,
  as mandated by the BGP specification.  This is required even if the
  router only supports IPv6 forwarding.
\item{\tt local-as}: this is the autonomous system number for the AS
  in which this router resides.  Any peers of this router must be
  configured to know this AS number - if there is a mismatch, a
  peering will not be established.  It is a 16-bit integer.
\item{\tt peer}: this delimits the configuration for a BGP peering
  association with another router.  Most BGP routers will have
  multiple peerings configured.  The peer ID is normally the IPv4
  unicast address of the router we are peering with (the syntax allows
  domain names for convenience, but this is {\it not} recommended in
  production settings).  For IBGP peerings this will normally be an IP
  address bound to the router's loopback address, so it is not
  associated with a specific interface, meaning that the peering will
  not go down if a single internal interface fails.  For EBGP
  peerings, this will normally be the IP address of the peer router on
  the interface over which we wish to exchange traffic, so that if the
  interface goes down, the peering will drop.

  For each configured {\stt peer}, the following
  configuration options can be specified:
\begin{description}
\item{\tt local-ip}: This is the IP address on this router that we
  will use for BGP connections to this peer.  It is mandatory to specify.
\item{\tt as}: this gives the AS number of this peer.  This must match
  the AS number that the peer itself advertises to us, or the BGP
  peering will not be established.  It is a 16-bit integer, and is
  mandatory to specify.
\item{\tt next-hop}: this is the IPv4 address that we will as the
  nexthop router address in routes that we send to this peer.
  Typically this is only specified for EBGP peerings.
\item{\tt local-port}: by default, BGP establishes its BGP connections
  over a TCP connection between port 179 on the local router and port
  179 on the remote router.  The local port for this peering can be
  changed by modifying this attribute.
\item{\tt peer-port}: The remote port for this peering can be
  changed by modifying this attribute. See also: {\stt local-port}.
\item{\tt holdtime}: This is the holdtime BGP should use when
  negotiating the connection with this peer.  If no message is
  received from a BGP peer during the negotiated holdtime, the
  peering will be shut down.
\item{\tt enabled}: This takes the value {\stt true} or {\stt false},
  and indicates whether the peering is currently enabled.  This allows
  a peering to be taken down temporarily without removing the
  configuration.
\item{\tt enable-ipv4-multicast}: this specifies that BGP should
  negotiation multiprotocol support to allow separate routes to be
  used for IPv4 unicast and IPv4 multicast.  Normally this would only
  be enabled if PIM-SM multicast routing is running on the router.
\item{\tt enable-ipv6-unicast}: this specifies that BGP should
  negotiation multiprotocol support to allow IPv6 unicast routes to be
  exchanged.
\item{\tt enable-ipv6-multicast}: this specifies that BGP should
  negotiate multiprotocol support to allow IPv6 multicast routes to be
  exchanged separately from IPv6 unicast routes.  It is possible to
  enable IPv6 multicast support without enabling IPv6 unicast support.
\end{description}
\item{\tt network4}: this specifies an IPv4 route to be advertised
  into the BGP routing, originating from this router.  The parameter
  is an IPv4 subnet in the format IPv4-address/prefix-length.  For
  example {\stt 10.0.0.0/24}.
\begin{description}
\item{\tt next-hop}: this is the next-hop router IPv4 address to be used
  in BGP advertisements for this route.  Typically it will be one of
  the router's IP addresses.
\item{\tt unicast}:
\item{\tt multicast}:
\end{description}

\item{\tt network6}: this specifies an IPv6 route to be advertised
  into the BGP routing, originating from this router.  The parameter
  is an IPv6 subnet in the format IPv4-address/prefix-length.  For
  example {\stt 10:0:0:10::/64}.
\end{description}

\subsection{Example Configurations}
\vspace{0.1in}
\noindent\framebox[\textwidth][l]{\scriptsize
\begin{minipage}{6in}
\begin{alltt}
\begin{tabbing}
xx\=xx\=xx\=xx\=xx\=\kill
protocols \{\\
\>bgp \{\\
\>\>bgp-id: 10.10.10.10\\
\>\>local-as: 65002\\
\\
\>\>peer 10.30.30.30 \{\\
\>\>\>local-ip: 10.10.10.10\\
\>\>\>as: 65000\\
\>\>\>next-hop: 10.10.10.20\\
\>\>\>/*\\
\>\>\>local-port: 179\\
\>\>\>peer-port: 179\\
\>\>\>*/\\
\>\>\>/* holdtime: 120 */\\
\>\>\>/* enabled: true */\\
\\
\>\>\>/* Optionally enable other AFI/SAFI combinations */\\
\>\>\>/* enable-ipv4-multicast */\\
\\
\>\>\>/* enable-ipv6-unicast */\\
\>\>\>/* enable-ipv6-multicast */\\
\>\>\}\\
\\
\>\>/* Originate IPv4 Routes */\\
\>\>/*\\
\>\>network4 10.10.10.0/24 \{\\
\>\>\>next-hop: 10.10.10.10\\
\>\>\>unicast: true\\
\>\>\>multicast: true\\
\>\>\}\\
\>\>*/\\
\\
\>\>/* Originate IPv6 Routes */\\
\>\>/*\\
\>\>network6 10:10:10:10::/64 \{\\
\>\>\>next-hop: 10:10:10:10:10:10:10:10\\
\>\>\>unicast: true\\
\>\>\>multicast: true\\
\>\>\}\\
\>\>*/\\
\>\}\\
\}
\end{tabbing}
\end{alltt}
\end{minipage}
}
\vspace{0.1in}

\section{Monitoring BGP}

On a router running BGP, the BGP routing state can be displayed using
the {\stt show bgp} operational-mode command.  Information is available
about the status of BGP peerings and about the routes received and
used.  In the 1.0 release, the set of commands is fairly crude, and
will be increased in future releases to provide better ways to display
subsets of this information.

As always, command completion using $<$TAB$>$ or ? will display the
available sub-commands and parameters:

\vspace{0.1in}
\noindent\framebox[\textwidth][l]{\scriptsize
\begin{minipage}{6in}
\begin{alltt}
\begin{tabbing}
xx\=xxxxxxxxxxxxxxxx\=\kill
Xorp> \textbf{show bgp ?}\\
Possible completions:\\
\><[Enter]>\>Execute this command\\
\>peers\>Show BGP peers info\\
\>routes\>Print BGP routes\\
\>|\>Pipe through a command\\
\end{tabbing}
\end{alltt}
\end{minipage}
}

The {\stt show bgp peers} command will display information about the
BGP peerings that have been configured.  It supports the optional
paramater {\stt detail} to give a lot more information:

\vspace{0.1in}
\noindent\framebox[\textwidth][l]{\scriptsize
\begin{minipage}{6in}
\begin{alltt}
\begin{tabbing}
xx\=xxxxxxxxxxxxxxxx\=\kill
Xorp> \textbf{show bgp peers ?}\\
Possible completions:\\
\><[Enter]>\>Execute this command\\
\>detail\>Show detailed BGP peers info\\
\>|\>Pipe through a command\\
\end{tabbing}
\end{alltt}
\end{minipage}
}

By itself, {\stt show bgp peers} provides a short list of the peerings
that are configured, irrespective of whether the peering is
in established state or not:

\vspace{0.1in}
\noindent\framebox[\textwidth][l]{\scriptsize
\begin{minipage}{6in}
\begin{alltt}
\begin{tabbing}
xx\=xxxxxxxxxxxxxxxx\=\kill
Xorp> \textbf{show bgp peers} \\
Peer 1: local 192.150.187.112/179 remote 69.110.224.158/179\\
Peer 2: local 192.150.187.112/179 remote 192.150.187.2/179\\
Peer 3: local 192.150.187.112/179 remote 192.150.187.78/179\\
Peer 4: local 192.150.187.112/179 remote 192.150.187.79/179\\
Peer 5: local 192.150.187.112/179 remote 192.150.187.109/179\\
\end{tabbing}
\end{alltt}
\end{minipage}
}

The command {\stt show bgp peers detail} will give a large amount of
information about all the peerings:  

\vspace{0.1in}
\noindent\framebox[\textwidth][l]{\scriptsize
\begin{minipage}{6in}
\begin{alltt}
\begin{tabbing}
xx\=xx\=xx\=xx\=xx\=\kill
Xorp> \textbf{show bgp peers detail}\\
Peer 1: local 192.150.187.112/179 remote 69.110.224.158/179\\
\>Peer ID: none\\
\>Peer State: ACTIVE\\
\>Admin State: START\\
\>Negotiated BGP Version: n/a\\
\>Peer AS Number: 65014\\
\>Updates Received: 0,  Updates Sent: 0\\
\>Messages Received: 0,  Messages Sent: 0\\
\>Time since last received update: n/a\\
\>Number of transitions to ESTABLISHED: 0\\
\>Time since last in ESTABLISHED state: n/a\\
\>Retry Interval: 120 seconds\\
\>Hold Time: n/a,  Keep Alive Time: n/a\\
\>Configured Hold Time: 120 seconds,  Configured Keep Alive Time: 40 seconds\\
\>Minimum AS Origination Interval: 0 seconds\\
\>Minimum Route Advertisement Interval: 0 seconds\\
\\
Peer 2: local 192.150.187.112/179 remote 192.150.187.2/179\\
\>Peer ID: 192.150.187.2\\
\>Peer State: ESTABLISHED\\
\>Admin State: START\\
\>Negotiated BGP Version: 4\\
\>Peer AS Number: 64999\\
\>Updates Received: 52786,  Updates Sent: 28\\
\>Messages Received: 52949,  Messages Sent: 189\\
\>Time since last received update: 2 seconds\\
\>Number of transitions to ESTABLISHED: 17\\
\>Time since last entering ESTABLISHED state: 6478 seconds\\
\>Retry Interval: 120 seconds\\
\>Hold Time: 120 seconds,  Keep Alive Time: 40 seconds\\
\>Configured Hold Time: 120 seconds,  Configured Keep Alive Time: 40 seconds\\
\>Minimum AS Origination Interval: 0 seconds\\
\>Minimum Route Advertisement Interval: 0 seconds\\
\\
\end{tabbing}
\end{alltt}
\end{minipage}
} 

The most important piece of information is typically whether or not
the peering is in ESTABLISHED state, indicating that the peering is up
and capable of exchanging routes.  ACTIVE state means that the peering
is configured to be up on this router, but for some reason the peering
is not currently up.  Typically this is because the remote peer is
unreachable, or because no BGP instance is running on the remote peer.

The {\stt show bgp routes} command displays the routes received by BGP
from its peers.  On a router with a full BGP routing table (140000
routes as of July 2003) this command will produce a large amount of
output:

\vspace{0.1in}
\noindent\framebox[\textwidth][l]{\scriptsize
\begin{minipage}{6in}
\begin{alltt}
\begin{tabbing}
xxx\=xxxxxxxxxxxxxxxx\=xxxxxxxxxxxxxxxx\=xxxxxxxxxxxxxxx\=xx\=\kill
Xorp> \textbf{show bgp routes}\\
Status Codes: * valid route, > best route\\
Origin Codes: i IGP, e EGP, ? incomplete\\
\\
\>Prefix\>Nexthop\>Peer\>AS Path\\
\>------\>-------\>----\>-------\\
*>\>3.0.0.0/8\>192.150.187.2\>192.150.187.2\>16694 25 2152 3356 7018 80 i\\
*>\>4.17.225.0/24\>192.150.187.2\>192.150.187.2\>16694 25 2152 11423 209 701 11853 6496 i\\
*>\>4.17.226.0/23\>192.150.187.2\>192.150.187.2\>16694 25 2152 11423 209 701 11853 6496 i\\
*>\>4.17.251.0/24\>192.150.187.2\>192.150.187.2\>16694 25 2152 11423 209 701 11853 6496 i\\
*>\>4.17.252.0/23\>192.150.187.2\>192.150.187.2\>16694 25 2152 11423 209 701 11853 6496 i\\
*>\>4.21.252.0/23\>192.150.187.2\>192.150.187.2\>16694 25 2152 11423 209 701 6389 8063 19198 i\\
*>\>4.23.180.0/24\>192.150.187.2\>192.150.187.2\>16694 25 2152 11423 209 3561 6128 30576 i\\
*>\>4.36.200.0/21\>192.150.187.2\>192.150.187.2\>16694 25 2152 174 3561 14742 11854 14135 i\\
*>\>4.78.0.0/21\>192.150.187.2\>192.150.187.2\>16694 25 2152 11423 209 3561 6347 23071 22938 i\\
*>\>4.78.32.0/21\>192.150.187.2\>192.150.187.2\>16694 25 2152 174 3491 29748 i\\
*>\>4.0.0.0/8\>192.150.187.2\>192.150.187.2\>16694 25 2152 3356 i\\
...
\end{tabbing}
\end{alltt}
\end{minipage}
}
\vspace{0.1in}

The format of the output is one route per line.  On each line:
\begin{itemize}
\item A status code is displayed, showing whether the route is valid
(available for us to use because it isn't filtered by the inbound BGP
filters and the nexthop is reachable), and whether it was the best BGP
route this router has received.
\item The network prefix for which the route applies is listed in the
  form {\stt 4.17.226.0/23}.  This indicates the base address for the
  network (address {\stt 4.17.226.0}), and the prefix length ({\stt 23} bits).
  Thus this route applies for addresses {\stt 4.17.226.0} to {\stt
  4.17.227.255} inclusive.
\item The nexthop is the IP address of the intermediate router towards
  which packet destined for the network prefix should be sent.  In
  this example all the displayed routes have the same nexthop.
\item The peer is the IP address of the BGP router which sent us this
  route.  The nexthop and the peer need not the the same (they often
  aren't with IBGP peerings for example) but in all the routes in this
  example they are the same.
\item The AS path is listed next.  This lists the AS numbers of the
  autonomous systems that the route has traversed to reach our
  router.  The AS at the left end of the path is the one nearest to
  our router and the one at the right end of the path is usually the
  AS number of the route's originator.  
\item Finally, whether the route's origin is from an IGP ({\stt i}),
  from EGP ({\stt e}, mostly obsolete), or incomplete ({\stt ?}) is
  listed.
\end{itemize}

\subsection{BGP MIB}

say what we support