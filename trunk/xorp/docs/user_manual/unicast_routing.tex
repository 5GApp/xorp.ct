\chapter{Unicast Routing}

\section{An Overview of Unicast Routing}

To forward packets, a router maintains a forwarding table which holds
routes indicating which neighboring router a packet for a particular
destination should be forwarded to.  At the minimum, a route then
consists of a destination {\it subnet} and a {\it nexthop}.  The destination
subnet is usually represented as a base IP address and a prefix-length
in bits.  For example, the subnet {\stt 128.16.64.0/24} has a prefix
length of 24 bits, indicating that the first 24 bits of this address
identify the network in question, and the last 8 bits identify hosts
on this subnet.  Thus a route for this subnet would be used to forward
packets for addresses 128.16.64.0 to 128.16.64.255 inclusive.  The
nexthop can be the IP address of a neighboring router, or it might
indicate that the route is for a subnet that is directly connected to
this router.

IP routers perform {\it longest prefix match} forwarding.  This means
that a router might have more than one route that matches a
destination address, and under such circumstances, it will use the
route that has the longest prefix.   For example, if a router has two routes:
\begin{itemize}
\item Subnet: {\tt 128.16.0.0/16} Nexthop: {\tt 10.0.0.1}
\item Subnet: {\tt 128.16.64.0/24} Nexthop: {\tt 10.0.0.2}
\end{itemize}
\noindent A packet destined for {\tt 128.16.0.1} would match the first
route only, and so would be forwarded to {\tt 10.0.0.1}.  However a
packet destined for {\tt 128.16.64.1} would match both routes, and so
would be forwarded to {\tt 10.0.0.2} because the second route has a
longer prefix (24 is longer than 16).

To be useful, a router needs to populate its forwarding table.  It can
do this in three ways:
\begin{itemize}
\item Routes for directly connected subnets are automatically entered into
the forwarding table.
\item Routes may be configured via the router's configuration file or
command line interface.  Such routes are known as {\it static
routes}. Static routes will be discussed in chapter \ref{staticroutes}.
\item Routes may be learned from another router via a routing
protocol.  Such routes are known as {\it dynamic routes}.
\end{itemize}

\subsection{Dyanmic Routing}

Many different routing protocols can supply dynamic routes.  The dynamic
routing protocols that are in most common use are:
\begin{itemize}
\item {\bf Border Gateway Protocol (BGP)}.  This is used for
inter-domain routing.
\item {\bf Routing Information Protocol (RIP)}.  This is probably the
simplest intra-domain routing protocol, and tends to be used on small
networks.
\item {\bf Open Shortest Path First (OSPF)}.  Used for intra-domain routing,
often on large ISP networks.
\item {\bf Integrated IS-IS}.  Used for intra-domain routing, often
on large ISP networks.  Similar to OSPF.
\item {\bf IGRP}: Used for intra-domain routing, typically in small to
medium sized networks.  Cisco-proprietary.
\end{itemize}
As of the 1.0 release, XORP supports BGP and RIP.  We plan to
implement OSPF and IS-IS shortly.  Our BGP implementation is discussed
in chapter \ref{bgp} and our RIP implementation is discussed in
chapter \ref{rip}.  In addition, there are also multicast routing
protocols, which we will discuss in chapter \ref{multicast}.

\subsection{Administrative Distance}

A router can run multiple routing protocols simultaneously.  For
example, we may use RIP to distribute routes within our network, and
BGP to learn external routes.  In some situations this can lead to a
router learning the same route from more than one routing protocol.
For example, we might learn the following routes:
\begin{itemize}
\item Subnet: {\tt 128.16.64.0/24} nexthop: {\tt 192.150.187.1}, Learned from BGP via an external peering. AS Path: {\tt 123 567 987}.
\item Subnet: {\tt 128.16.64.0/24} Nexthop: {\tt 10.0.0.2}, Learned from RIP with metric 13
\end{itemize}
The longest prefix match rule doesn't help us because the prefix
lengths are the same, and the metric used for RIP is not directly
comparable against the AS path length or any other attribute attached
to a BGP route.  How then do we decide which route to take?

A XORP router uses the concept of {\it administrative distance} to
determine which route wins.  This concept is the same as that used by
Cisco routers.  Basically each routing protocol has a configured
``distance'', and if a route is heard from two protocols, then the
version with the smallest distance wins.

The built-in table of administative distances XORP uses is:
\begin{tabbing}
xxx\=xxxxxxxxxxxxxxxxxxxxxxxxxxxxxxxxxxxxx\=\kill
\>Directly connected subnets: \>0\\
\>Static routes:\>1\\
\>BGP, heard from external peer:\>20\\
\>OSPF (when implemented):\>110\\
\>IS-IS (when implemented):\>115\\
\>RIP:\>120\\
\>BGP, heard from internal peer\>200\\
\end{tabbing}
As of the XORP 1.0 release, the is no way to modify these defaults,
but this capability will be added soon.

\subsection{Route Redistribution}

A common requirement is to redistribute routes between routing
protocols.  Some examples might be:
\begin{itemize}
\item When interconnecting some subnets that are statically routed
  with some subnets that are dynamically routed using RIP.  Rather
  that configure the static routes and additionally tell RIP to
  originate route advertisements for the same subnets, it is simpler
  and less error prone to configure the router to simply redistribute
  all the static routes into RIP.
\item When a network is using RIP internally, and also uses BGP to
  peer with the rest of the Internet.  One solution would be to
  configure BGP at the border routes to originate route advertisements
  for the internal subnets, but if a new subnet is added internally,
  then the border routers also need to be correctly modified.  Instead
  we can simply configure the border routers to redistribute RIP
  routes into BGP.  
\end{itemize}
XORP is capable of performing such route redistribution.  This is
generally configured using the {\stt import} and {\stt export}
configuration statements.  These terms are relative to the router's
routing table, so if the directive {\stt export static} is added to
the RIP configuration, then this indicates that RIP should export all
the static routes to its neighbors via the RIP protocol.

