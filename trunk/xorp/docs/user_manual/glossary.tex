%
% $XORP$
%

\chapter*{Glossary}
\begin{description}

  \item{\bf AS}: see Autonomous System.

  \item{\bf Autonomous System}: a routing domain that is under one
  administrative authority, and which implements its own routing
  policies.  Key concept in BGP.

  \item{\bf BGP}: Border Gateway Protocol.  See Chapter \ref{bgp}.

  \item{\bf Bootstrap Router}: A PIM-SM router that chooses the RPs for
  a domain from amongst a set of candidate RPs.

  \item{\bf BSR}: See Bootstrap Router.

  \item{\bf Candidate RP}: A PIM-SM router that is configured to be a
  candidate to be an RP.  The Bootstrap Router will then choose the
  RPs from the set of candidates.

  \item{\bf Dynamic Route}: A route learned from another router via a
  routing protocol such as RIP or BGP.

  \item{\bf EGP}: see Exterior Gateway Protocol.

  \item{\bf Exterior Gateway Protocol}: a routing protocol used to route
  between Autonomous Systems.  The main example is BGP.

  \item{\bf IGMP}: Internet Group Management Protocol.  See Chapter
  \ref{igmp}.

  \item{\bf IGP}: see Interior Gateway Protocol.

  \item{\bf Interior Gateway Protocol}: a routing protocol used to route
  within an Autonomous System.  Examples include RIP, OSPF and IS-IS.

  \item{\bf Live CD}: A CD-ROM that is bootable.  In the context of
  \xorp, the Live CD can be used to produce a low-cost router without
  needing to install any software.

  \item{\bf MLD}: Multicast Listener Discovery protocols.  See Chapter
  \ref{igmp}.

  \item{\bf MRIB}: See Multicast RIB.

  \item{\bf Multicast RIB}: the part of the RIB that holds multicast routes.
  These are not directly used for forwarding, but instead are used by
  multicast routing protocols such as PIM-SM to perform RPF checks
  when building the multicast distibution tree.

  \item{\bf OSPF}: See Open Shortest Path First.

  \item{\bf Open Shortest Path First:} an IGP routing protocol based on
  a link-state algorithm.  Used to route within medium to large networks.

  \item{\bf PIM-SM}: Protocol Independent Multicast, Sparse Mode. See
  Chapter \ref{pimsm}.

  \item{\bf Rendezvous Point}: A router used in PIM-SM as part of the
  rendezvous process by which new senders are grafted on to the
  multicast tree.

  \item{\bf Reverse Path Forwarding}: many multicast routing protocols
  such as PIM-SM build a multicast distribution tree based on the best
  route back from each receiver to the source, hence multicast packets
  will be forwarded along the reverse of the path to the source.

  \item{\bf RIB}: See Routing Information Base.

  \item{\bf RIP}: Routing Information Protocol.  See Chapter \ref{rip}.

  \item{\bf Routing Information Base}: the collection of routes learned
  from all the dynamic routing protocols running on the router.
  Subdivided into a Unicast RIB for unicast routes and a Multicast RIB.

  \item{\bf RP}: See Rendezvous Point.

  \item{\bf RPF}: See Reverse Path Forwarding.

  \item{\bf Static Route}: A route that has been manually configured on
  the router.

  \item{\bf xorpsh}: \xorp command shell.  See Chapter \ref{xorpsh}.

  \item{\bf xorp\_rtrmgr}: \xorp router manager process.  See Chapter
  \ref{xorpsh}.

\end{description}
