%
% $XORP: xorp/docs/user_manual/config_overview.tex,v 1.6 2005/04/07 03:09:12 pavlin Exp $
%

\chapter{Configuration Overview}
\section{Introduction}

A XORP router must be configured to perform the desired operations.
The configuration information can be provided in one of the two ways:

\begin{itemize}
\item
  Use a configuration file when the rtrmgr is started.
  By default, the rtrmgr will load the configuration from file
  ``config.boot'' in the XORP installation directory.
  This file can be specified by the ``-b~~$<$filename$>$'' command line
  option:
\begin{verbatim}
    xorp_rtrmgr -b my_config.boot
\end{verbatim}

    See ``rtrmgr/config.boot.sample'' for an example of a
    configuration file (note that this file MUST be modified
    before using it).

\item
  Use the xorpsh command line interface after the rtrmgr is started.
  It should be noted that command line completion in the xorpsh
  does greatly simplify configuration.
\end{itemize}

\noindent
A mixture of both methods is permissible. For example,
a configuration file can also be loaded from within the xorpsh.

At very minimum, a router's interfaces must be configured (see
Section~\ref{sec:network_interfaces}). Typically, the FEA needs to be
configured (\eg to enable unicast forwarding); the FEA configuration is
described in Section~\ref{sec:fea}. All protocol configuration is
described in Section~\ref{sec:protocols}.

\section{Network Interfaces}
\label{sec:network_interfaces}

A XORP router will only use interfaces that it has been explicitly
configured to use. Even for protocols such as BGP that are agnostic to
interfaces, if the next-hop router for a routing entry is not through
a configured interface the route will not be installed. For protocols
that are explicitly aware of interfaces only configured interfaces
will be used.

Every physical network device in the system is considered to be an
``interface''. Every interface can contain a number of virtual
interfaces (``vif''s). In the majority of cases the interface name and
vif name will be identical and will map to the name given to the
interface by the operating system. A virtual interface is configured
with the address or addresses that should be used. At each level in
the configuration hierarchy ({\tt interface}, {\tt vif} and
{\tt address}) it is necessary to enable this part of the
configuration.

\vspace{0.1in}
\noindent\framebox[\textwidth][l]{\scriptsize
\begin{minipage}{6in}
\begin{alltt}
\begin{tabbing}
xx\=xx\=xx\=xx\=xx\=\kill
interfaces \{\\
\>interface dc0 \{\\
\>\>description: "data interface"\\
\>\>disable: false\\
\>\>/* default-system-config */\\
\>\>vif dc0 \{\\
\>\>\>disable: false\\
\>\>\>address 10.10.10.10 \{\\
\>\>\>\>prefix-length: 24\\
\>\>\>\>broadcast: 10.10.10.255\\
\>\>\>\>disable: false\\
\>\>\>\}\\
\>\>\>/*\\
\>\>\>address 10:10:10:10:10:10:10:10 \{\\
\>\>\>\>prefix-length: 64\\
\>\>\>\>disable: false\\
\>\>\>\}\\
\>\>\>*/\\
\>\>\}\\
\>\}\\
\}
\end{tabbing}
\end{alltt}
\end{minipage}
}
\vspace{0.1in}

We recommend that you select the interfaces that you want to use on
your system and configure them as above. If you are configuring an
interface that is currently being used by the the system make sure
that there is no mismatch in the {\tt address}, {\tt prefix-length} and
{\tt broadcast} arguments.
If the \linebreak
{\tt default-system-config} statement is used, it
instructs the FEA that the interface should be configured by using
the existing interface information from the underlying system.
In that case, the {\tt vif} and {\tt address} sections must not be
configured.

\section{Forwarding Engine Abstraction}
\label{sec:fea}

It is a requirement to explicitly enable forwarding for each
protocol family.

\vspace{0.1in}
\noindent\framebox[\textwidth][l]{\scriptsize
\begin{minipage}{6in}
\begin{alltt}
\begin{tabbing}
xx\=xx\=xx\=xx\=xx\=\kill
fea \{\\
\>unicast-forwarding4 \{\\
\>\>disable: false\\
\>\}\\
/*\\
\>unicast-forwarding6 \{\\
\>\>disable: false\\
\>\}\\
*/\\
\}
\end{tabbing}
\end{alltt}
\end{minipage}
}
\vspace{0.1in}

If IPv4 forwarding is required you will require the configuration
above. If the system supports IPv6 and IPv6 forwarding is required,
then the {\tt unicast-forwarding6} statement must be used to enable
it~\footnote{Note that prior to XORP Release-1.1, the
{\tt enable-unicast-forwarding4} and {\tt enable-unicast-forwarding6}
flags were used instead to enable or disable the IPv4 and IPv6 forwarding.}.

\newpage
\section{Protocols}
\label{sec:protocols}

A unicast router typically will be configured with one or more
of the following protocols:
StaticRoutes (Section~\ref{sec:protocols:static_routes}),
RIP (Section~\ref{sec:protocols:rip})
or BGP (Section~\ref{sec:protocols:bgp}).

A multicast router must have the MFEA configured
(Section~\ref{sec:protocols:mfea}). Typically, a multicast router should
have IGMP/MLD configured (Section~\ref{sec:protocols:mld6igmp}).
Currently, PIM-SM is the only multicast routing protocol implemented
(Section~\ref{sec:protocols:pim}). If some multicast-specific static
routes need to be installed in the MRIB (for computing the reverse-path
forwarding information), those can be specified in the StaticRoutes
configuration (Section~\ref{sec:protocols:static_routes}).
If there are no
unicast routing protocols configured, the FIB2MRIB module may
need to be configured as well (Section~\ref{sec:protocols:fib2mrib}).

\subsection{Static Routes}
\label{sec:protocols:static_routes}

This is the simplest routing protocol in XORP. It allows the
installation of unicast or multicast static routes (either IPv4 or
IPv6).  Note that in case of multicast the routes are installed only
in the user-level Multicast Routing Information Base and are used for
multicast-specific reverse-path forwarding information by multicast routing
protocols such as PIM-SM.

\vspace{0.1in}
\noindent\framebox[\textwidth][l]{\scriptsize
\begin{minipage}{6in}
\begin{alltt}
\begin{tabbing}
xx\=xx\=xx\=xx\=xx\=\kill
protocols \{\\
\>static \{\\
\>\>route4 10.20.0.0/16 \{\\
\>\>\>nexthop: 10.10.10.20\\
\>\>\>metric: 1\\
\>\>\}\\
\>\>mrib-route4 10.20.0.0/16 \{\\
\>\>\>nexthop: 10.10.10.30\\
\>\>\>metric: 1\\
\>\>\}\\
\>\>/*\\
\>\>route6 20:20:20:20::/64 \{\\
\>\>\>nexthop: 10:10:10:10:10:10:10:20\\
\>\>\>metric: 1\\
\>\>\}\\
\>\>mrib-route6 20:20:20:20::/64 \{\\
\>\>\>nexthop: 10:10:10:10:10:10:10:30\\
\>\>\>metric: 1\\
\>\>\}\\
\>\>*/\\
\>\}\\
\}
\end{tabbing}
\end{alltt}
\end{minipage}
}
\vspace{0.1in}

\newpage
\subsection{Routing Information Protocol}
\label{sec:protocols:rip}

In order to run RIP it is sufficient to specify the set of interfaces,
vifs and addresses ({\tt interface}, {\tt vif} and {\tt address}) on
which RIP is enabled~\footnote{Note that prior to XORP Release-1.1,
the {\tt enable} flag was used instead of {\tt disable} to enable
or disable each part of the configuration.}.

If you wish to announce routes then it is necessary to
{\tt export}~\footnote{Note that in release candidate 1.0-RC, the {\tt import}
 keyword is used in place of {\tt export}.}
the routes that are to be announced. For example, {\tt connected},
{\tt static} and {\tt rip}.

{\bf Note that this mechanism for originating
routes is temporary. In the next release it will be replaced by
a more generic policy framework. Unfortunately, the same keyword
{\stt export} will be used but with different syntax.}

\vspace{0.1in}
\noindent\framebox[\textwidth][l]{\scriptsize
\begin{minipage}{6in}
\begin{alltt}
\begin{tabbing}
xx\=xx\=xx\=xx\=xx\=\kill
protocols \{\\
\>rip \{\\
\>/* Redistribute routes for connected interfaces */\\
\>\>/*\\
\>\>export connected \{\\
\>\>\>metric: 0\\
\>\>\>tag: 0\\
\>\>\}\\
\>\>*/\\
\>/* Redistribute static routes */\\
\>\>/*\\
\>\>export static \{\\
\>\>\>metric: 1\\
\>\>\>tag: 0\\
\>\>\}\\
\>\>*/\\
\>/* Run on specified network interface addresses */\\
\>/*\\
\>\>interface dc0 \{\\
\>\>\>vif dc0 \{\\
\>\>\>\>address 10.10.10.10 \{\\
\>\>\>\>\>disable: false\\
\>\>\>\>\}\\
\>\>\>\}\\
\>\>\}\\
\>*/\\
\>\}\\
\}
\end{tabbing}
\end{alltt}
\end{minipage}
}

\newpage
\subsection{Border Gateway Protocol}
\label{sec:protocols:bgp}

In order to run BGP the {\tt bgp-id} (BGP Identifier) and {\tt local-as}
(Autonomous System number) must be specified.

The {\tt peer} statement specifies a peering. The argument to the peer
statement is the IP address of the peer. The {\tt local-ip} is the IP
address that TCP should use. The {\tt as} is the Autonomous System Number
of the peer.

\vspace{0.1in}
\noindent\framebox[\textwidth][l]{\scriptsize
\begin{minipage}{6in}
\begin{alltt}
\begin{tabbing}
xx\=xx\=xx\=xx\=xx\=\kill
protocols \{\\
\>bgp \{\\
\>\>bgp-id: 10.10.10.10\\
\>\>local-as: 65002\\
\\
\>\>peer 10.30.30.30 \{\\
\>\>\>local-ip: 10.10.10.10\\
\>\>\>as: 65000\\
\>\>\>next-hop: 10.10.10.20\\
\>\>\>/*\\
\>\>\>local-port: 179\\
\>\>\>peer-port: 179\\
\>\>\>*/\\
\>\>\>/* holdtime: 120 */\\
\>\>\>/* disable: false */\\
\\
\>\>\>/* IPv4 unicast is enabled by default */\\
\>\>\>/* ipv4-unicast: true */\\
\\
\>\>\>/* Optionally enable other AFI/SAFI combinations */\\
\>\>\>/* ipv4-multicast: true */\\
\>\>\>/* ipv6-unicast: true */\\
\>\>\>/* ipv6-multicast: true */\\
\>\>\}\\
\\
\>\>/* Originate IPv4 Routes */\\
\>\>/*\\
\>\>network4 10.10.10.0/24 \{\\
\>\>\>next-hop: 10.10.10.10\\
\>\>\>unicast: true\\
\>\>\>multicast: true\\
\>\>\}\\
\>\>*/\\
\\
\>\>/* Originate IPv6 Routes */\\
\>\>/*\\
\>\>network6 10:10:10:10::/64 \{\\
\>\>\>next-hop: 10:10:10:10:10:10:10:10\\
\>\>\>unicast: true\\
\>\>\>multicast: true\\
\>\>\}\\
\>\>*/\\
\>\}\\
\}
\end{tabbing}
\end{alltt}
\end{minipage}
}
\vspace{0.1in}

Currently BGP is not able to {\tt import} routes from other routing
protocols such as {\tt static}. It is however possible to originate routes
using {\tt network4} and {\tt network6} statements such as in the above
example.

\newpage
\subsection{Multicast Forwarding Engine Abstraction}
\label{sec:protocols:mfea}

The MFEA must be configured if the XORP router is to be used for multicast
routing. The MFEA for IPv4 and IPv6 are configured separately.

In the configuration we must explicitly configure the entity itself, and
each {\tt vif}. The {\tt traceoptions} section is used to explicitly enable
log information that can be used for debugging purpose~\footnote{Note that
prior to XORP Release-1.1, the {\tt enable} flag was used instead of
{\tt disable} to enable or disable each part of the configuration.}.

\vspace{0.1in}
\noindent\framebox[\textwidth][l]{\scriptsize
\begin{minipage}{6in}
\begin{alltt}
\begin{tabbing}
xx\=xx\=xx\=xx\=xx\=\kill
plumbing \{\\
\>mfea4 \{\\
\>\>disable: false\\
\>\>interface dc0 \{\\
\>\>\>vif dc0 \{\\
\>\>\>\>disable: false\\
\>\>\>\}\\
\>\>\}\\
\>\>interface register\_vif \{\\
\>\>\>vif register\_vif \{\\
\>\>\>\>/* Note: this vif should be always enabled */\\
\>\>\>\>disable: true\\
\>\>\>\}\\
\>\>\}\\
\>\>traceoptions \{\\
\>\>\>flag all \{\\
\>\>\>\>disable: true\\
\>\>\>\}\\
\>\>\}\\
\>\}\\
\}\\
\\
plumbing \{\\
\>mfea6 \{\\
\>\>disable: true\\
\>\>interface dc0 \{\\
\>\>\>vif dc0 \{\\
\>\>\>\>disable: true\\
\>\>\>\}\\
\>\>\}\\
\>\>interface register\_vif \{\\
\>\>\>vif register\_vif \{\\
\>\>\>\>/* Note: this vif should be always enabled */\\
\>\>\>\>disable: true\\
\>\>\>\}\\
\>\>\}\\
\>\>traceoptions \{\\
\>\>\>flag all \{\\
\>\>\>\>disable: true\\
\>\>\>\}\\
\>\>\}\\
\>\}\\
\}
\end{tabbing}
\end{alltt}
\end{minipage}
}
\vspace{0.1in}

Note that the interface/vif named {\tt register\_vif} is special.
If PIM-SM is configured, then 
\linebreak
{\tt register\_vif} must be enabled
in the MFEA.

\newpage
\subsection{Internet Group Management Protocol/Multicast Listener Discovery}
\label{sec:protocols:mld6igmp}

IGMP/MLD should be configured if the XORP router is to be used for multicast
routing and if we want to track multicast group membership for directly
connected subnets. Typically this is the case for a multicast router,
therefore it should be enabled. IGMP and MLD are configured separately:
IGMP is used for tracking IPv4 multicast members; MLD is used for tracking
IPv6 multicast members.

In the configuration we must explicitly configure each entity and
each {\tt vif}. The {\tt traceoptions} section is used to explicitly
enable log information that can be used for debugging purpose~\footnote{Note
that prior to XORP Release-1.1, the {\tt enable} flag was used instead of
{\tt disable} to enable or disable each part of the configuration.}.

\vspace{0.1in}
\noindent\framebox[\textwidth][l]{\scriptsize
\begin{minipage}{6in}
\begin{alltt}
\begin{tabbing}
xx\=xx\=xx\=xx\=xx\=\kill
protocols \{\\
\>igmp \{\\
\>\>disable: false\\
\>\>interface dc0 \{\\
\>\>\>vif dc0 \{\\
\>\>\>\>disable: false\\
\>\>\>\}\\
\>\>\}\\
\>\>traceoptions \{\\
\>\>\>flag all \{\\
\>\>\>\>disable: false\\
\>\>\>\}\\
\>\>\}\\
\>\}\\
\}\\
\\
protocols \{\\
\>mld \{\\
\>\>disable: false\\
\>\>interface dc0 \{\\
\>\>\>vif dc0 \{\\
\>\>\>\>disable: false\\
\>\>\>\}\\
\>\>\}\\
\>\>traceoptions \{\\
\>\>\>flag all \{\\
\>\>\>\>disable: false\\
\>\>\>\}\\
\>\>\}\\
\>\}\\
\}
\end{tabbing}
\end{alltt}
\end{minipage}
}
\vspace{0.1in}

\newpage
\subsection{Protocol Independent Multicast - Sparse Mode}
\label{sec:protocols:pim}

PIM-SM should be configured if the XORP router is to be used for
multicast routing in PIM-SM domain. PIM-SM for IPv4 and IPv6 are
configured separately. At minimum, the entity itself and the
virtual interfaces should be enabled, and the mechanism for obtaining
the Candidate-RP set (either the Bootstrap mechanism, or a static-RP
set)~\footnote{Note that prior to XORP Release-1.1, the {\tt enable} flag
was used instead of {\tt disable} to enable or disable each part of the
configuration.}.

\vspace{0.1in}
\noindent\framebox[\textwidth][l]{\scriptsize
\begin{minipage}{4.5in}
\begin{alltt}
\begin{tabbing}
xx\=xx\=xx\=xx\=xx\=\kill
protocols \{\\
\>pimsm4 \{\\
\>\>disable: false\\
\>\>interface dc0 \{\\
\>\>\>vif dc0 \{\\
\>\>\>\>disable: false\\
\>\>\>\>/* dr-priority: 1 */\\
\>\>\>\>/* alternative-subnet 10.40.0.0/16 */\\
\>\>\>\}\\
\>\>\}\\
\>\>interface register\_vif \{\\
\>\>\>vif register\_vif \{\\
\>\>\>\>/* Note: this vif should be always enabled */\\
\>\>\>\>disable: false\\
\>\>\>\}\\
\>\>\}\\
\\
\>\>static-rps \{\\
\>\>\>rp 10.60.0.1 \{\\
\>\>\>\>group-prefix 224.0.0.0/4 \{\\
\>\>\>\>\>/* rp-priority: 192 */\\
\>\>\>\>\>/* hash-mask-len: 30 */\\
\>\>\>\>\}\\
\>\>\>\}\\
\>\>\}\\
\\
\>\>bootstrap \{\\
\>\>\>disable: false\\
\>\>\>cand-bsr \{\\
\>\>\>\>scope-zone 224.0.0.0/4 \{\\
\>\>\>\>\>/* is-scope-zone: false */\\
\>\>\>\>\>cand-bsr-by-vif-name: "dc0"\\
\>\>\>\>\>/* bsr-priority: 1 */\\
\>\>\>\>\>/* hash-mask-len: 30 */\\
\>\>\>\>\}\\
\>\>\>\}\\
\\
\>\>\>cand-rp \{\\
\>\>\>\>group-prefix 224.0.0.0/4 \{\\
\>\>\>\>\>/* is-scope-zone: false */\\
\>\>\>\>\>cand-rp-by-vif-name: "dc0"\\
\>\>\>\>\>/* rp-priority: 192 */\\
\>\>\>\>\>/* rp-holdtime: 150 */\\
\>\>\>\>\}\\
\>\>\>\}\\
\>\>\}\\
\\
{\rm continued overleaf....}
\end{tabbing}
\end{alltt}
\end{minipage}
}
\newpage
\vspace{0.1in}
\noindent\framebox[\textwidth][l]{\scriptsize
\begin{minipage}{4.5in}
\begin{alltt}
\begin{tabbing}
xx\=xx\=xx\=xx\=xx\=\kill
\>\>switch-to-spt-threshold \{\\
\>\>\>/* approx. 1K bytes/s (10Kbps) threshold */\\
\>\>\>disable: false\\
\>\>\>interval-sec: 100\\
\>\>\>bytes: 102400\\
\>\>\}\\
\\
\>\>traceoptions \{\\
\>\>\>flag all \{\\
\>\>\>\>disable: false\\
\>\>\>\}\\
\>\>\}\\
\>\}\\
\}
\\
\\
protocols \{\\
\>pimsm6 \{\\
\>\>disable: false\\
\>\>interface dc0 \{\\
\>\>\>vif dc0 \{\\
\>\>\>\>disable: false\\
\>\>\>\>/* dr-priority: 1 */\\
\>\>\>\>/* alternative-subnet 40:40:40:40::/64 */\\
\>\>\>\}\\
\>\>\}\\
\>\>interface register\_vif \{\\
\>\>\>vif register\_vif \{\\
\>\>\>\>/* Note: this vif should be always enabled */\\
\>\>\>\>disable: false\\
\>\>\>\}\\
\>\>\}\\
\\
\>\>static-rps \{\\
\>\>\>rp 50:50:50:50:50:50:50:50 \{\\
\>\>\>\>group-prefix ff00::/8 \{\\
\>\>\>\>\>/* rp-priority: 192 */\\
\>\>\>\>\>/* hash-mask-len: 126 */\\
\>\>\>\>\}\\
\>\>\>\}\\
\>\>\}\\
\\
\>\>bootstrap \{\\
\>\>\>disable: false\\
\>\>\>cand-bsr \{\\
\>\>\>\>scope-zone ff00::/8 \{\\
\>\>\>\>\>/* is-scope-zone: false */\\
\>\>\>\>\>cand-bsr-by-vif-name: "dc0"\\
\>\>\>\>\>/* bsr-priority: 1 */\\
\>\>\>\>\>/* hash-mask-len: 30 */\\
\>\>\>\>\}\\
\>\>\>\}\\
\\
\>\>\>cand-rp \{\\
\>\>\>\>group-prefix ff00::/8 \{\\
\>\>\>\>\>/* is-scope-zone: false */\\
\>\>\>\>\>cand-rp-by-vif-name: "dc0"\\
\>\>\>\>\>/* rp-priority: 192 */\\
\>\>\>\>\>/* rp-holdtime: 150 */\\
\>\>\>\>\}\\
\>\>\>\}\\
\>\>\}\\
\\
{\rm continued overleaf....}
\end{tabbing}
\end{alltt}
\end{minipage}
}
\newpage
\vspace{0.1in}
\noindent\framebox[\textwidth][l]{\scriptsize
\begin{minipage}{4.5in}
\begin{alltt}
\begin{tabbing}
xx\=xx\=xx\=xx\=xx\=\kill
\>\>switch-to-spt-threshold \{\\
\>\>\>/* approx. 1K bytes/s (10Kbps) threshold */\\
\>\>\>disable: false\\
\>\>\>interval-sec: 100\\
\>\>\>bytes: 102400\\
\>\>\}\\
\\
\>\>traceoptions \{\\
\>\>\>flag all \{\\
\>\>\>\>disable: false\\
\>\>\>\}\\
\>\>\}\\
\>\}\\
\}
\end{tabbing}
\end{alltt}
\end{minipage}
}
\vspace{0.1in}

A number of parameters have default values, therefore
they don't have to be configured (those parameters are commented-out
in the above sample configuration).

Note that the interface/vif named {\tt register\_vif} is special.
If PIM-SM is configured, then {\tt register\_vif} must be enabled.

The {\tt dr-priority} parameter is used to configure the Designated Router
priority per virtual interface (note that in case of {\tt register\_vif}
it is not used).

The {\tt alternative-subnet} statement is used to associate additional
subnets with a network interface. For example, if you want to make
incoming traffic with a non-local source address appear as it is
coming from a local subnet, then {\tt alternative-subnet} can be
used. Typically, this is needed as a work-around solution when we use
uni-directional interfaces for receiving traffic (e.g., satellite
links).  Note: use {\tt alternative-subnet} with extreme care, only if
you know what you are really doing!

If PIM-SM uses static RPs, those can be configured within the
{\tt static-rps} section. For each RP, an {\tt rp} section is needed, and each
section should contain the multicast prefix address the static RP is
configured with. The RP priority can be modified with the {\tt rp-priority}
parameter.

If PIM-SM uses the Bootstrap mechanism to obtain the Candidate-RP set,
this can be configured in the {\tt bootstrap} section.
If the XORP router is to be used as a Candidate-BSR, this should be
specified in the {\tt cand-bsr} section.
For a router to be a Candidate-BSR it must advertise for
each zone (scoped or non-scoped) the associated multicast prefix address.
The {\tt cand-bsr} section should contain {\tt scope-zone} statements for each
multicast prefix address.
The vif name with the address that is to be used as the Candidate-BSR
is specified by the {\tt cand-bsr-by-vif-name} statement.
The Candidate-BSR priority can be modified with the {\tt bsr-priority}
parameter.

If the XORP router is to be a Candidate-RP, this should be specified
in the {\tt cand-rp} section.
For a router to be a Candidate-RP it must advertise for
each zone (scoped or non-scoped) the associated multicast prefix address.
The {\tt cand-rp} section should contain {\tt group-prefix} statements
for each multicast prefix address.
The vif name with the address that is to be used as the Candidate-RP
is specified by the {\tt cand-rp-by-vif-name} statement.
The Candidate-RP priority can be modified with the
{\tt rp-priority} parameter; the Candidate-RP holdtime can be modified
with the {\tt rp-holdtime} parameter.

The {\tt is-scope-zone} parameter is used to specify whether a
Candidate-BSR {\tt scope-zone} or a Candidate-RP {\tt group-prefix} is
scoped. Currently, 
scoped zones are not well tested, hence it is recommended {\tt scope-zone}
is always set to {\tt false}.
Note that typically the {\tt hash-mask-len} should not be modified; if you
don't know what {\tt hash-mask-len} is used for, don't modify it!

The {\tt switch-to-spt-threshold} section can be used to specify the
multicast data bandwidth threshold used by the last-hop PIM-SM routers
and the RPs to initiate shortest-path switch toward the multicast source.
Parameter {\tt interval-sec} is used to specify the periodic measurement
interval;
parameter {\tt bytes} is used to specify the threshold in number of bytes
within the measurement interval. It is recommended that the measurement
interval is not too small, and should be on the order of tens of seconds.

The {\tt traceoptions} section is used to explicitly enable log information
that can be used for debugging purpose.

\newpage
\subsection{FIB2MRIB}
\label{sec:protocols:fib2mrib}

The FIB2MRIB module is used to obtain the Forwarding Information Base
information from the underlying system (via the FEA), and to propagate
it to the MRIB, so it can be used by multicast routing protocols
such as PIM-SM. Typically, it is needed only if the unicast routing
protocols (if any) on that router do not inject routes into the
MRIB~\footnote{Note that prior to XORP Release-1.1, the {\tt enable} flag
was used instead of {\tt disable} to enable or disable each part of the
configuration.}.

\vspace{0.1in}
\noindent\framebox[\textwidth][l]{\scriptsize
\begin{minipage}{6in}
\begin{alltt}
\begin{tabbing}
xx\=xx\=xx\=xx\=xx\=\kill
protocols \{\\
\>fib2mrib \{\\
\>\>disable: false\\
\>\}\\
\}
\end{tabbing}
\end{alltt}
\end{minipage}
}
\vspace{0.1in}


