%
% $XORP: xorp/docs/user_manual/multicast_routing.tex,v 1.4 2005/04/07 03:09:12 pavlin Exp $
%

\chapter{Multicast Routing}
\label{multicast}
\section{An Overview of Multicast Routing}

IP Multicast is a technology that allows one-to-many and many-to-many
distribution of data on the Internet.  Senders send their data to a
multicast IP destination address, and receives express an interest in
receiving traffic destined for such an address.  The network then
figures out how to get the data from senders to receivers.  

If both the sender and receiver for a multicast group are on the same
local broadcast subnet, then the routers do not need to be involved in
the process, and communication can take place directly.  If, however,
the sender and receiver are on different subnets, then a multicast
routing protocol needs to be involved in setting up multicast
forwarding state on the tree between the sender and the receivers.

\subsection{Multicast Routing}

Broadly speaking, there are two different types of multicast routing
protocols:
\begin{itemize}
\item Dense-mode protocols, where traffic from a new multicast source
  is delivered to all possible receivers, and then subnets where there
  are no members request to be pruned from the distribution tree.
\item Sparse-mode protocols, where explicit control messages are used
  to ensure that traffic is only delivered to the subnets where there
  are receivers that requested to receive it.
\end{itemize}
Examples of dense-mode protocols are {\it DVMRP} and {\it PIM Dense
Mode}.  Examples of sparse-mode protocols are PIM Sparse Mode, CBT,
and MOSPF.  Most of these protocols are largely historic at this time,
with the exception of PIM Sparse Mode (PIM-SM) and PIM Dense Mode
(PIM-DM), and even PIM-DM is not very widely used.

In addition to the routing protocols used to set up forwarding state
between subnets, a way is needed for the routers to discover that
there are local receivers on a directly attached subnet.  For IPv4
this role is served by the Internet Group Management Protocol (IGMP)
and for IPv6 this role is served by the Multicast Listener Discovery
protocol (MLD).

\subsection{Service Models: ASM vs SSM}

There are two different models for IP multicast:
\begin{itemize}
\item Any Source Multicast (ASM), in which a receiver joins a
  multicast group, and receives traffic from any senders that send to
  that group.
\item Source-Specific Multicast (SSM), in which a receiver explicitly
  joins to a (source, group) pairing.
\end{itemize}
Traditionally IP multicast used the ASM model, but problems deploying
inter-domain IP multicast resulted in the much simpler SSM model being
proposed.  In the future it is likely that ASM will continue to be
used within intranets and enterprises, but SSM will be used when
multicast is used inter-domain.  The two models are compatible, and
PIM-SM can be used as a multicast routing protocol for both.  The
principal difference is that ASM only requires IGMPv2 or MLDv1,
whereas SSM requires IGMPv3 or MLDv2 to permit the receivers to
specify the address of the sending host.

\subsection{Multicast Addresses}

For IPv4, multicast addresses are in the range 224.0.0.0 to
239.255.255.255 inclusive.  Addresses within 224.0.0.0/24 are
considered link-local and should not be forwarded between subnets.
Addresses within 232.0.0.0/8 are reserved for SSM usage.  Addresses in
239.0.0.0/8 are ASM addresses defined for varying sizes of limited
scope.

IPv6 multicast addresses are a little more complex.  IPv6 multicast
addresses start with the prefix {\stt ff}, and have the following
format:
\begin{verbatim}
   |   8    |  4 |  4 |                  112 bits                   |
   +------ -+----+----+---------------------------------------------+
   |11111111|flgs|scop|                  group ID                   |
   +--------+----+----+---------------------------------------------+
\end{verbatim}
\begin{itemize}
\item {\stt 11111111} ({\stt ff} in hexadecimal) at the start of the address
identifies the address as being a multicast address.

\item {\it flgs} is a set of 4 flags: 
\begin{verbatim}
    +-+-+-+-+
    |0|0|0|T|
    +-+-+-+-+
\end{verbatim}

The high-order 3 flags are reserved, and must be initialized to 0.

{\it T} = 0 indicates a permanently-assigned (``well-known'') multicast
address, assigned by the global internet numbering authority.

{\it T} = 1 indicates a non-permanently-assigned (``transient'')
multicast address.

\item {\it scop} is a 4-bit multicast scope value used to limit the scope of
the multicast group.  The values in hex are:
\begin{description}
\item{\stt 1}  node-local scope
\item{\stt 2}  link-local scope
\item{\stt 5}  site-local scope
\item{\stt 8}  organization-local scope
\item{\stt E}  global scope
\end{description}

\item {\it group ID} identifies the multicast group, either permanent or
transient, within the given scope.  
\end{itemize}

RFC 2373 gives more details about IPv6 multicast addresses.

\section{Supported Protocols}

XORP supports the following multicast protocols:
\begin{itemize}
\item PIM Sparse Mode for both ASM and SSM multicast routing for IPv4.
\item PIM Sparse Mode for both ASM and SSM multicast routing for IPv6.
\item IGMPv1 and IGMPv2 for IPv4 local multicast membership. 
\item MLDv1 for IPv6 local multicast membership. 
\end{itemize}
XORP does not yet support IGMPv3 and MLDv2, but plans to do so soon.
