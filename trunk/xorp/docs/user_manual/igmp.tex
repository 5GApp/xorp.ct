\chapter{IGMP and MLD}
\label{igmp}

\section{Terminology and Concepts}

When a receiver joins a multicast group, the multicast routers serving
that receiver's subnet need to know that the receiver has joined so
that they can arrange for multicast traffic destined for that group to
reach this subnet.  The Internet Group Management Protocol (IGMP) is
a link-local protocol for IPv4 that communicates this information
between receivers and routers.  The same role for IPv6 is performed by
the Multicast Listener Discovery protocol (MLD).

The basic IGMP mechanism works as follows.  When a multicast receiver
joins a multicast group it multicasts an IGMP Join message onto the
subnet on which it is joining.  The local routers receive this join,
and cause multicast traffic destined for the group to reach this
subnet.  Periodically one of the local routers sends a IGMP Query
message onto the subnet.  If there are multiple multicast routers on
the subnet, then one of them is elected as the sole querier for that
subnet.  In response to an IGMP query, receivers respond by refreshing
their IGMP Join.  If the join is not refreshed in response to queries,
then the state is removed, and multicast traffic for this group ceases
to reach this subnet.

There are three different versions of IGMP:
\begin{itemize}
\item IGMP version 1 functions as described above.
\item IGMP version 2 adds support for IGMP Leave messages to allow
  fast leave from a multicast group.
\item IGMP version 3 adds support for source include and exclude
  lists, to allow a receiver in indicate that it only wants to hear
  traffic from certain sources, or not receive traffic from certain
  sources.
\end{itemize}

Currently XORP supports IGMPv1 and IGMPv2.

MLD for IPv6 functions in basically the same way as IGMP.  The
functionality of MLDv1 corresponds with that of IGMPv2, and the
functionality of MLDv2 corresponds with that of IGMPv3.  

Currently XORP supports MLDv2.

\vfill\eject
\section{Standards}
XORP complies with the following standards for multicast group membership:
\begin{description}
\item{\bf RFC 2236}: Internet Group Management Protocol, Version 2
\item{\bf RFC 1112}: Host extensions for IP multicasting.
\item{\bf RFC 2710}: Multicast Listener Discovery (MLD) for IPv6.
\end{description}

\section{Configuring IGMP and MLD}

IGMP and MLD only require the interfaces/vifs to be configured that
are intended to have multicast listeners.

\subsection{Configuration Syntax}
\vspace{0.1in}
\noindent\framebox[\textwidth][l]{\scriptsize
\begin{minipage}{6in}
\begin{alltt}
\begin{tabbing}
xx\=xx\=xx\=xx\=xx\=\kill
protocols \{\\
\>igmp \{\\
\>\>targetname: {\it text}\\
\>\>enabled: {\it bool}\\
\>\>interface {\it text} \{\\
\>\>\>vif {\it text}\{\\
\>\>\>\>enabled: {\it bool}\\
\>\>\>\}\\
\>\>\}\\
\>\>traceoptions \{\\
\>\>\>flag all \{\\
\>\>\>\>enabled: {\it bool}\\
\>\>\>\}\\
\>\>\}\\
\>\}\\
\}\\
\\
protocols \{\\
\>mld \{\\
\>\>targetname: {\it text}\\
\>\>enabled: {\it bool}\\
\>\>interface {\it text} \{\\
\>\>\>vif {\it text} \{\\
\>\>\>\>enabled: {\it bool}\\
\>\>\>\}\\
\>\>\}\\
\>\>traceoptions \{\\
\>\>\>flag all \{\\
\>\>\>\>enabled: {\it bool}\\
\>\>\>\}\\
\>\>\}\\
\>\}\\
\}
\end{tabbing}
\end{alltt}
\end{minipage}
}
\vspace{0.1in}

\begin{description}
\item{\tt protocols}: this delimits the configuration for all routing
  protocols in the XORP router configuration.  It is mandatory that
  IGMP configuration is under the {\stt protocols} node in the
  configuration.
\item{\tt igmp}: this delimits the IGMP configuration part of the XORP
  router configuration.
\item{\tt targetname} this is the name for this instance of IGMP.  It
  defaults to ``{\stt IGMP}'', and it is not recommended that this
  default is overridden under normal usage scenarios.
\item{\stt enabled}: this takes the value {\stt true} or {\stt false},
  and determines whether IGMP as a whole is enabled on this router.
  The default value is {\stt true}.
\item{\stt interface}: this specifies an interface to be monitored by
  IGMP for the presence of multicast receivers.  Each interface to be
  monitored by IGMP needs to be explicitly listed. The value is the
  name of an interface that has been configured in the {\stt
  interfaces} section of the router configuration (see Chapter
  \ref{interfaces}).

  For each interface, one or more VIFs must be specified:
\begin{description}
\item{\stt vif}: this specifies a vif to be monitored by IGMP for the
  presence of multicast receivers.  Each vif to be monitored by IGMP
  needs to be explicitly listed.  The value is the name of a vif that
  has been configured in the {\stt interfaces} section of the router
  configuration (see Chapter \ref{interfaces}).
 
  Each vif takes the following optional parameter:
\begin{description}
\item{\stt enabled}: this takes the value {\stt true} or {\stt false},
  and determines whether IGMP is enabled on this vif.
  The default value is {\stt true}.
\end{description}
\end{description}
\item{\stt traceoptions}: this directive delimits the configuration of
  debugging and tracing options for IGMP.
\begin{description}
\item{\stt flag}: this directive is used to specify which tracing
  options are enabled.  Possible parameters are:
\begin{description}
\item{\stt all}: this directive specifies that all tracing
  options should be enabled.  Possible parameters are:
\begin{description}
\item{\stt enabled}: this takes the value {\stt true} or {\stt false},
  and enables or disables tracing. The default is {\stt false}.
\end{description}
\end{description}
\end{description}
\end{description}

The configuration parameters for MLD are identical to those for IGMP,
except that they are delimited by an {\stt mld} directive rather than
an {\stt igmp} directive.

\vfill\eject
\subsection{Example Configurations}
\vspace{0.1in}
\noindent\framebox[\textwidth][l]{\scriptsize
\begin{minipage}{6in}
\begin{alltt}
\begin{tabbing}
xx\=xx\=xx\=xx\=xx\=\kill
protocols \{\\
\>igmp \{\\
\>\>interface dc0 \{\\
\>\>\>vif dc0 \\
\>\>\}\\
\>\>interface dc1 \{\\
\>\>\>vif dc1 \\
\>\>\}\\
\>\}\\
\}\\
\\
protocols \{\\
\>mld \{\\
\>\>enabled: true\\
\>\>interface dc0 \{\\
\>\>\>vif dc0 \{\\
\>\>\>\>enabled: true\\
\>\>\>\}\\
\>\>\}\\
\>\>traceoptions \{\\
\>\>\>flag all \{\\
\>\>\>\>enabled: true\\
\>\>\>\}\\
\>\>\}\\
\>\}\\
\}
\end{tabbing}
\end{alltt}
\end{minipage}
}

\vspace{0.1in} In the example configuration above, IGMP is enabled on
two vifs on two different interfaces ({\stt dc0/dc0} and {\stt
dc1/dc1}).  In addition, MLD is enabled on interface/vif {\stt
dc0/dc0}, and all MLD tracing functionality is enabled for diagnostic
purposes.

\newpage
\section{Monitoring IGMP}

The {\stt show igmp group} command can be used to display
information about IGMP group membership:

\vspace{0.1in}
\noindent\framebox[\textwidth][l]{\scriptsize
\begin{minipage}{6in}
\begin{alltt}
\begin{tabbing}
xxxxxxxxxxxxx\=xxxxxxxxxxxxxxxx\=xxxxxxxxxxxxxxxx\=xxxxxxxxxxxxx\=xxxx\=xxxx\=\kill
Xorp> \textbf{show igmp group}\\
Interface  \>Group         \>Source        \>LastReported\>Timeout\\
dc0        \>224.0.0.2     \>0.0.0.0       \>10.4.0.1     \>\>161\\
dc0        \>224.0.0.13    \>0.0.0.0       \>10.4.0.1     \>\>159\\
dc0        \>224.0.1.20    \>0.0.0.0       \>10.4.0.2     \>\>197\\
dc2        \>224.0.0.2     \>0.0.0.0       \>10.3.0.2     \>\>155\\
dc2        \>224.0.0.13    \>0.0.0.0       \>10.3.0.1     \>\>157
\end{tabbing}
\end{alltt}
\end{minipage}
}
\vspace{0.1in}

In the above example, {\stt Source} refers to the multicast source
address in case of source-specific IGMP join entries or is set to
{\stt 0.0.0.0} in case of any-source IGMP join entries.
Field {\stt LastReported} contains the address of the last receiver that
responded by an IGMP Join message.
The {\stt Timeout} field shows the number of seconds until it is time
to query for host members (\ie to send an IGMP Query message for this
particular entry).

The {\stt show igmp interface} command can be used to display
information about IGMP interfaces:

\vspace{0.1in}
\noindent\framebox[\textwidth][l]{\scriptsize
\begin{minipage}{6in}
\begin{alltt}
\begin{tabbing}
xxxxxxxxxxxxx\=xxxxxxxxx\=xxxxxxxxxxxxxxxx\=xxx\=x\=xxxx\=xxxxxx\=xx\=xxxxx\=x\=\kill
Xorp> \textbf{show igmp interface}\\
Interface  \>State  \>Querier       \>Timeout\>\>\>Version\>\>Groups\\
dc0        \>UP     \>10.4.0.1       \>\>None \>\>\>2          \>\>3\\
dc2        \>UP     \>10.3.0.1      \>\>\>136   \>\>2          \>\>2\\
register\_vif\>DISABLED\>0.0.0.0      \>\>None \>\>\>2          \>\>0
\end{tabbing}
\end{alltt}
\end{minipage}
}
\vspace{0.1in}

Note that in the above example it is normal for the interface named
{\stt register\_vif} to be {\stt DISABLED}. This interface has special
purpose and is used only by PIM-SM.

The {\stt show igmp interface address} command can be used to display
information about addresses of IGMP interfaces:

\vspace{0.1in}
\noindent\framebox[\textwidth][l]{\scriptsize
\begin{minipage}{6in}
\begin{alltt}
\begin{tabbing}
xxxxxxxxxxxxx\=xxxxxxxxxxxxxxxx\=xxxxxxxxxxxxx\=\kill
Xorp> \textbf{show igmp interface address}\\
Interface  \>PrimaryAddr   \>SecondaryAddr\\
dc0        \>10.4.0.1\\
dc2        \>10.3.0.2\\
register\_vif\>0.0.0.0       \>10.4.0.1
\end{tabbing}
\end{alltt}
\end{minipage}
}
\vspace{0.1in}

In the above example, the {\stt PrimaryAddr} per interface is the address
used to originate IGMP messages, and all other alias addresses are listed
as {\stt SecondaryAddr} with one address per line.

The equivalent commands for MLD are:
\begin{itemize}
\item {\stt show mld group}
\item {\stt show mld interface}
\item {\stt show mld interface address}
\end{itemize}
