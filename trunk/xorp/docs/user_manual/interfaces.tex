\chapter{Network Interfaces}

\section{Network Interfaces Terminology and Concepts}

A router receives packets on its network interfaces from its
neighboring routers.  Some of those packets will be destined for the
router itself, but most of then will normally be forwarded on to other
routers via another network interface, or on to locally connected
hosts via a network interface.

There are many different types of network interface, such as Ethernet,
ATM, DS3, or ISDN.  Sometimes the underlying physical interface will
need explicit configuration before it can establish a link, and
sometimes the link requires no configuration.  In addition, some
network interfaces behave from a routing point of view as if they were
really multiple interfaces, in that the router may have to forward
packets between different ``channels'' on the same interface.  

We choose to distinguish in a XORP router between physical interfaces
(which we call {\it interfaces}, and logical interfaces, which we can
virtual interfaces or {\it vifs}.  An example of a {\it interface}
might be an Ethernet card.  An example of a {\it vif} might be one of
many VLANs configured on that ethernet\footnote{As of the 1.0 release,
VLANs are not yet supported in XORP.}.

Conceptually, XORP routes packets between vifs.  Thus it is also vifs
that are given IP addresses.  Each interface may contain many vifs.
Conversely every vif is always part of an interface, although some
interfaces such as the loopback interface do not have a physical
realisation.

The XORP naming convention for vifs is that they are named as they
would be in the underlying forwarding path.  For example, if the
forwarding path is implemented in the FreeBSD kernel, then {\stt fxp0}
might denote a 100-base-T Ethernet vif (with no VLAN).  On a router
using Linux as the forwarding path, the same vif might be called {\stt
eth0}.  

If a physical interface cannot support multiple vifs, or if there's a
default vif on a physical interface, then the interface name and the
vif name will normally be the same.  Again, this is determined by the
underlying forwarding path.  A common example would be Ethernet
without VLANs, where the interface and vif might both be named {\stt
fxp0} on Freebsd or both called {\stt eth0} on Linux.

\section{Configuring Network Interfaces}

A XORP router will only use interfaces that it has been explicitly
configured to use. Even for protocols such as BGP that are agnostic to
interfaces, if the next-hop router for a routing entry is not through
a configured interface the route will not be installed. For protocols
such as RIP that are explicitly aware of interfaces only, configured
interfaces will be used.  

\subsection{Configuration Syntax}

The available configuration syntax for network interfaces and
addresses is as follows:

\vspace{0.1in}
\noindent\framebox[\textwidth][l]{\scriptsize
\begin{minipage}{6in}
\begin{alltt}
\begin{tabbing}
xx\=xx\=xx\=xx\=xx\=\kill
interfaces \{\\
\>interface {\it text} \{\\
\>\>description: {\it text}\\
\>\>mac: {\it macaddr}\\
\>\>mtu: {\it uint}\\
\>\>default-system-config\\
\>\>enabled: {\it bool}\\
\>\>vif {\it text} \{\\
\>\>\>enabled: {\it bool}\\
\>\>\>address {\it IPv4-addr} \{\\
\>\>\>\>prefix-length: {\it int(1..32}\\
\>\>\>\>broadcast: {\it IPv4-addr}\\
\>\>\>\>destination: {\it IPv4-addr}\\
\>\>\>\>enabled: {\it bool}\\
\>\>\>\}\\
\>\>\>address {\it IPv6-addr} \{\\
\>\>\>\>prefix-length: {\it int(1..128)}\\
\>\>\>\>destination: {\it IPv6-addr}\\
\>\>\>\>enabled: {\it bool}\\
\>\>\>\}\\
\>\>\}\\
\>\}\\
\}
\end{tabbing}
\end{alltt}
\end{minipage}
}

\begin{description}
\item{\tt interfaces}: this delimits all the interface configuration information within the XORP configuration file.
\item{\tt interface}: this delimits the configuration for a particular
  interface.  The parameter is the name of the interface, which must
  correspond to an interface known to the router forwarding path.

  For each interface, the following configuration is possible:
\begin{description}
\item{\tt description}: this is a human-readable description for the
  interface.  It is primarily used to help the router operator
  remember which interface serves which purpose.  It is optional.
\item{\tt mac}: This allows the MAC address for the interface to be
  set.  MAC addresses on devices such as Ethernets are usually fixed,
  but in some cases it is possible to override the built-in default
  MAC address.  The format should be in a form appropriate for the
  interface type.  For an ethernet interfacek, this would be six
  colon-separated 16-bit numbers in hexadecimal, such as
  {\stt 00:0a:59:9a:f2:ba}.
\item{\tt mtu}: This allows the maximum transfer unit (MTU) to be set
  for the interface as a whole (applying to all VIFs).  The value is
  an integer number of bytes, and should be less that or equal to the
  largest MTU supported by the physical device. When forwarding,
  packets larger than the MTU will be fragmented.
\item{\tt default-system-config}: Normally all the interfaces, vifs
  and addresses on a XORP router will be configured through the XORP
  configuration file and command line interface.  However, under
  certain circumstances it is useful to be able to run XORP as a
  routing daemon without changing the current configuration of
  interfaces and addresses.  This primitive tells XORP to obtain it's
  interface configuration by reading the existing configuration back
  from the forwarding engine rather than by configuring the forwarding
  engine.
\item{\tt enabled}: this flag enables or disables the interface for
  routing and forwarding.  It takes the value {\stt true} or {\stt
  false}.  Configuring an interface to be disabled has the same effect
  as removing its configuration, but without losing what the
  configuration used to be.
\item{\tt vif}: this configures a vif on the corresponding interface.
  In some cases this may cause the vif to be created; and example
  might be an Ethernet VLAN.  In other cases this merely denotes the
  start of the configuration for the vif.  The parameter is the name
  of the vif, as understood by the router forwarding engine.

  For each vif, the following configuration is possible:
\begin{description}
\item{\tt enabled}: this flag enables or disables the vif for
  routing and forwarding.  It takes the value {\stt true} or {\stt
  false}.  Configuring a vif to be disabled has the same effect
  as removing its configuration, but without losing what the
  configuration used to be.
\item{\tt address}: this specifies a new IP address for this vif.  A
  vif might have multiple IP addresses, and might have both IPv4
  address and IPv6 addresses.  The parameter is either an IPv4 or IPv6
  address.

  For each address, the following configuration is possible:
\begin{description}
\item{\tt prefix-length}: this gives the prefix length of the subnet
  connected to this interface.  For an IPv4 address, prefix-length
  must be between 4 and 32.  For an IPv6 address, prefix-length must
  be between 8 and 128.  This field is mandatory for each address.
\item{\tt broadcast}: this gives the subnet broadcast address for the
  subnet corresponding to the vif address.  It is only needed for IPv4
  addresses, and is only needed for historical reasons.  It takes the
  form of an IPv4 address. 

  Normally the broadcast address will have the local hosts part of the
  subnet address set to all ones.  For example, with address 10.0.0.0
  and prefix-length 20, the broadcast address will have the last 12
  bits set to one, and hence will be 10.0.15.255.
\item{\tt destination}: this specified the destination IP address.  It
  is only relevant for point-to-point interfaces, where the IP address
  at each end of the link need not share an IP subnet.
\item{\tt enabled}: this flag enables or disables this IP address on
  this vif.  It takes the value {\stt true} or {\stt false}.
  Configuring an IP address to be disabled has the same effect as removing its
  configuration, but without losing what the configuration used to be.
\end{description}
\end{description}
\end{description}
\end{description}

\subsection{Example Configurations}

We recommend that you select the interfaces that you want to use on
your system and configure them as below.  Interfaces that you do not
wish XORP to use for forwarding should be omitted from the
configuration.

\vspace{0.1in}
\noindent\framebox[\textwidth][l]{\scriptsize
\begin{minipage}{6in}
\begin{alltt}
\begin{tabbing}
xx\=xx\=xx\=xx\=xx\=\kill
interfaces \{\\
\>interface dc0 \{\\
\>\>description: "ethernet interface"\\
\>\>enabled: true\\
\>\>vif dc0 \{\\
\>\>\>enabled: true\\
\>\>\>address 10.10.10.10 \{\\
\>\>\>\>prefix-length: 24\\
\>\>\>\>broadcast: 10.10.10.255\\
\>\>\>\>enabled: true\\
\>\>\>\}\\
\>\>\>\\
\>\>\>address 10:10:10:10:10:10:10:10 \{\\
\>\>\>\>prefix-length: 64\\
\>\>\>\>enabled: true\\
\>\>\>\}\\
\>\>\>\\
\>\>\}\\
\>\}\\
\}
\end{tabbing}
\end{alltt}
\end{minipage}
}

In the example above, the router has only one interface configured.
This interface is called {\stt dc0}, and the vif is also called {\stt
dc0}.  In this case, this is because this interface is an ethernet
inferface, and VLANs are not being used, so the vif is simply the
default vif for this interface.  

The vif has both an IPv4 and an IPv6 address configured.  The IPv4
address is {\stt 10.10.10.10}, and connects to the subnet {\stt
10.10.10.0/24} as determined by the prefix-length.  Consistent with
this, the subnet broadcast address is {\stt 10.10.10.255}.

The IPv6 address has a prefix-length of 64 bits, and does not need (or
allow) the broadcast address to be explicitly specified.  

In this case, an ethernet is not a point-to-point interface, so no
destination address is specified.



If the {\tt default-system-config} statement is used, as shown in the
example belore, it instructs the FEA that the interface should be
configured by using the existing interface information from the
underlying system.  In that case, the {\tt vif} and {\tt address}
sections must not be configured.

\vspace{0.1in}
\noindent\framebox[\textwidth][l]{\scriptsize
\begin{minipage}{6in}
\begin{alltt}
\begin{tabbing}
xx\=xx\=xx\=xx\=xx\=\kill
interfaces \{\\
\>interface dc0 \{\\
\>\>description: "data interface"\\
\>\>enabled: true\\
\>\>default-system-config\\
\>\}\\
\}
\end{tabbing}
\end{alltt}
\end{minipage}
}
\vspace{0.1in}



\section{Monitoring Network Interfaces}

The state of a XORP router's interfaces can be displayed from
operational mode using the {\stt show interfaces} command.  By itself,
{\stt show interfaces} will list information about all the interfaces
in the router:

\vspace{0.1in}
\noindent\framebox[\textwidth][l]{\scriptsize
\begin{minipage}{6in}
\begin{alltt}
\begin{tabbing}
xx\=xxxxxxxxxxxxxxxx\=\kill
Xorp> \textbf{show interfaces}\\

TBD 

\end{tabbing}
\end{alltt}
\end{minipage}
}

To display information about a specific interface, use the {\stt show
interfaces $<$interface$>$} command:

\vspace{0.1in}
\noindent\framebox[\textwidth][l]{\scriptsize
\begin{minipage}{6in}
\begin{alltt}
\begin{tabbing}
xx\=xxxxxxxxxxxxxxxx\=\kill
Xorp> \textbf{show interfaces}\\

TBD 

\end{tabbing}
\end{alltt}
\end{minipage}
}
