\chapter{Network Interfaces}

\section{Terminology and Concepts}

vif vs interface, v4, v6

\section{Configuring Network Interfaces}

A XORP router will only use interfaces that it has been explicitly
configured to use. Even for protocols such as BGP that are agnostic to
interfaces, if the next-hop router for a routing entry is not through
a configured interface the route will not be installed. For protocols
such as RIP that are explicitly aware of interfaces only, configured
interfaces will be used.

Every physical network device in the system is considered to be an
``interface''. Thus an ethernet interface or an ATM adaptor might be
considered to be an interface.  

Every interface can contain a number of virtual interfaces (``VIF''s).
In many cases there will be a one-to-one correspondence between
interfaces and VIFs, but this does not have to be the case.  In the
case of ethernet, the same physical interface might have multiple
logical devices in the form of VLANs.  Such VLANs would be multiple
vifs on a single interface.\footnote{As of the 1.0 release, VLANs are
not yet supported in XORP.}

In cases where the interface only supports one VIF, the VIF name will
be the same as the interface name, and will map to the name given to
the interface by the operating system.   A virtual interface is
configured with the address or addresses that should be used. At each
level in the configuration hierarchy ({\tt interface}, {\tt vif} and
{\tt address}) it is necessary to enable this part of the
configuration.

\subsection{Configuration Syntax}

The available configuration syntax for network interfaces and
addresses is as follows:

\vspace{0.1in}
\noindent\framebox[\textwidth][l]{\scriptsize
\begin{minipage}{6in}
\begin{alltt}
\begin{tabbing}
xx\=xx\=xx\=xx\=xx\=\kill
interfaces \{\\
\>interface {\it text} \{\\
\>\>description: {\it text}\\
\>\>mac: {\it macaddr}\\
\>\>mtu: {\it uint}\\
\>\>default-system-config\\
\>\>enabled: {\it bool}\\
\>\>vif {\it text} \{\\
\>\>\>enabled: {\it bool}\\
\>\>\>address {\it IPv4-addr} \{\\
\>\>\>\>prefix-length: {\it int(1..32}\\
\>\>\>\>broadcast: {\it IPv4-addr}\\
\>\>\>\>destination: {\it IPv4-addr}\\
\>\>\>\>enabled: {\it bool}\\
\>\>\>\}\\
\>\>\>address {\it IPv6-addr} \{\\
\>\>\>\>prefix-length: {\it int(1..128)}\\
\>\>\>\>destination: {\it IPv6-addr}\\
\>\>\>\>enabled: {\it bool}\\
\>\>\>\}\\
\>\>\}\\
\>\}\\
\}
\end{tabbing}
\end{alltt}
\end{minipage}
}

\subsection{Example Configurations}

We recommend that you select the interfaces that you want to use on
your system and configure them as below. If you are configuring an
interface that is currently being used by the the system make sure
that there is no mismatch in the {\tt address}, {\tt prefix-length} and
{\tt broadcast} arguments.

\vspace{0.1in}
\noindent\framebox[\textwidth][l]{\scriptsize
\begin{minipage}{6in}
\begin{alltt}
\begin{tabbing}
xx\=xx\=xx\=xx\=xx\=\kill
interfaces \{\\
\>interface dc0 \{\\
\>\>description: "data interface"\\
\>\>enabled: true\\
\>\>vif dc0 \{\\
\>\>\>enabled: true\\
\>\>\>address 10.10.10.10 \{\\
\>\>\>\>prefix-length: 24\\
\>\>\>\>broadcast: 10.10.10.255\\
\>\>\>\>enabled: true\\
\>\>\>\}\\
\>\>\>\\
\>\>\>address 10:10:10:10:10:10:10:10 \{\\
\>\>\>\>prefix-length: 64\\
\>\>\>\>enabled: true\\
\>\>\>\}\\
\>\>\>\\
\>\>\}\\
\>\}\\
\}
\end{tabbing}
\end{alltt}
\end{minipage}
}

If the {\tt default-system-config} statement is used, it instructs the
FEA that the interface should be configured by using the existing
interface information from the underlying system.  In that case, the
{\tt vif} and {\tt address} sections must not be configured.

\vspace{0.1in}
\noindent\framebox[\textwidth][l]{\scriptsize
\begin{minipage}{6in}
\begin{alltt}
\begin{tabbing}
xx\=xx\=xx\=xx\=xx\=\kill
interfaces \{\\
\>interface dc0 \{\\
\>\>description: "data interface"\\
\>\>enabled: true\\
\>\>default-system-config\\
\>\}\\
\}
\end{tabbing}
\end{alltt}
\end{minipage}
}
\vspace{0.1in}



\section{Monitoring Network Interfaces}

description of op-mode commands