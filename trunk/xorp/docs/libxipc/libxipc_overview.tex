\documentclass[11pt]{article}
\usepackage{times}
\usepackage{amsmath}
\usepackage{graphicx}
\usepackage{rotating}

\newcommand{\eg}{\emph{e.g.\/}} % e.g.

\title{XORP Inter-Process Communication Library Overview \\
\vspace{1ex}
Version 0.1}
\author{ XORP Project					\\
	 International Computer Science Institute	\\
	 Berkeley, CA 94704, USA			\\
	 {\it feedback@xorp.org}
}
\date{December 11, 2002}

\begin{document}
\maketitle
\begin{abstract}Extensibility and robustness are key goals of the eXtensible
Open Router Project (XORP).  A step towards these goals is separating
the tasks of router in multiple userland programs.  The programs may
need to communicate with the kernel and may need to communicate across
multiple hosts in the case of a distributed router.  We have developed
an asynchronous remote procedure call mechanism that is capable of
working with multiple transport protocols between remote hosts and can
leverage existing IPC mechanisms within a single host. This note
documents the initial design and directions it may take.
\end{abstract}

\section{Introduction}

Robustness and extensibility are two of the goals of the XORP project.
One way a router can achieve robustness is to run routing protocols in
protected environments, such as separate userland processes on a
modern operating system.  And one way a router can achieve
extensibility is to be independent of the details about where those
routing processes are running and what the composition of the
forwarding plane is.  The routing processes and network interfaces
could be running on one machine or distributed across a cluster of
machines that appear as single router.  A necessary feature once
routing protocols are running in distinct processes and potentially on
distinct machines is an inter-process communication mechanism.  In
contrast to traditional inter-process communication schemes, the
scheme employed in the XORP project can utilize multiple transport
protocols and potentially communicate with unmodified components
through these protocols, for instance SNMP or HTTP.

The lofty goals of XORP's Inter-Process Communication (XIPC) scheme are:

\begin{itemize}

\item to provide all of the IPC communication mechanisms that a router is
likely to need, \eg sockets, ioctl's, sysV messages and shared memory.

\item to provide a consistent and transparent interface irrespective
of the underlying transport mechanism used.

\item to transparently select the appropriate IPC mechanism when
alternatives exist.

\item to provide an asynchronous interface.

\item to be efficient.

\item to potentially wrapper communication with non-Xorp processes,
\eg HTTP and SNMP servers.

\item to be renderable in human readable form so XORP processes can
read and write commands from configuration files.

\end{itemize}

The XIPC goals are realized through XORP Resource Locators (XRLs).  An
XRL describes a procedure call.  A resolver process, the
\emph{Finder}, translates XRLs into methods for inter-process
communication.  Each XRL has a textual description and each XORP
process supports multiple XRL call interfaces, each comprised of a
group of related XRLs that implement some aspect of the processes
exported functionality.  When a XORP process starts it registers its
supported XRLs and transport protocols with the Finder.  The Finder is
then able to inform processes with sufficient privileges how to
resolve to resolve those XRLs.

The XIPC library consists of a set of routines for building and
manipulating XRLs, for dispatching and handling the responses from
XRLs, and converting XRLs to and from human readable text.  This
document describes XRLs, the XIPC library, and the issues involved in
it's design.  The internals of the library are subject to change and
some portions, such as access control, are not yet implemented in the
library.

In addition to the XIPC library, an interface definition language
exists, and tools to translate these into callable C++ interfaces and
into a set of C++ handler routines for handling the receipt of XRL
calls.  These tools are described in document \cite{xorp:xrl_interfaces}.
The tools reduce 
the amount of familiarity the working programmer need have with the
internals of the XIPC library.  This document provides an overview of
the XIPC library and is the recommended starting point before using
the library.

\section{XORP Resource Locators (XRL's)}

The mechanism we've settled for IPC within XORP processes is mediated
through \emph{Xorp Resource Locators} (XRL's).  An XRL describes a
procedure call.  It comprises of the protocol family to be used for
transport, the arguments for the protocol family, the interface of the
target being called and it's version, the method, and an argument
list.  Examples of XRLs in their human readable forms are shown in
figure \ref{fig:human-readable}.  The existence of a human readable
form for XRLs is chiefly a convenience for humans who need to work
with XRLs and not indicative of how they work internally.

Resolved and unresolved forms of the same are XRL are depicted in
figure \ref{fig:human-readable}.  The unresolved form is the starting
point for the majority of inter-process communication.  In the
unresolved form the protocol family is set to ``finder'' and the
protocol parameters set to the target name that the XRL call is
intended for.  A process wishing to dispatch an XRL for the first time
passes the unresolved XRL to the Finder, which returns the resolved
form with the appropriate protocol family and protocol family
arguments.  The finder may also modify the other components of the
XRL, but doesn't usually do so.  This functionality may be useful when
supporting backwards compatibility of interfaces, ie the Finder could
modify the interface number and method name.

The resolved forms of XRLs are typically maintained in a client side
cache so the Finder need not be consulted for each XRL dispatch.

\clearpage
\begin{sidewaysfigure}
(a) Unresolved form:

\begin{centering}
\small\begin{verbatim}
finder://fea/fti/0.1/add_route?net:ipv4net=10.0.0.1/8&gateway:ipv4=192.150.187.1
+-----   +-- +-- +-- +-------- +------------------------------------------------
|        |   |   |   |         |
|        |   |   |   Method    Arguments
|        |   |   |
|        |   |   Interface version 
|        |   |
|        |   Interface Name  
|        | 
|        Protocol Parameters
|
Protocol Family
\end{verbatim}
\end{centering}
\normalsize

(b) Resolved form:

\small
\begin{verbatim}
xudp://192.150.1.5:1992/fti/0.1/add_route?net:ipv4net=10.0.0.1&gateway:ipv4=192.150.1.1
+---   +--------------  +-- +-- +-------- +--------------------------------------------
|      |                |   |   |         |
|      |                |   |   Method    Arguments
|      |                |   |
|      |                |   Interface version 
|      |                |
|      |                Interface Name  
|      | 
|      Protocol Parameters
|
Protocol Family
\end{verbatim}
\normalsize
\caption{\label{fig:human-readable}Human readable XRL forms.}
\end{sidewaysfigure}
\clearpage

\section{Components of XRL Framework}

\begin{description}
\item [XRL] an inter-process call that is transparent to the
underlying transport method.

\item [Finder] the process that co-ordinates the resolution of target
names into a parseable form to find the Xrl Protocol Family Listener.

\item [XRL Router] an object responsible for dispatching and receiving
XRL requests.  They manage all the underlying interactions and are the
interface that users are expected to use for XRL interactions.

\item [Finder Client] an object associated with an XRL Router that
manages the communication with the Finder.

\item [XRL Protocol Family] a supported transport mechanism for the
invoked XRL.

\item [XRL Protocol Family Sender] an entity that dispatches XRL
requests and handles responses.  Senders are created based on Finder
lookup's of the appropriate communication mechanism.

\item [XRL Protocol Family Listener] an entity that listens for
incoming requests, dispatches the necessary hook, and sends the
responses.  When Listeners are created they register the appropriate
mapping with the Finder so that corresponding Senders can be
instantiated to talk with them.

\end{description}

The kdoc documentation provides details of the particular classes.

\end{document}
