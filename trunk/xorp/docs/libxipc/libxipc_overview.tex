\documentclass[11pt]{article}
\usepackage{times}
\usepackage{amsmath}
\usepackage{graphicx}
\usepackage{rotating}
\usepackage{fullpage}

\newcommand{\eg}{\emph{e.g.\/}} % e.g.
\newcommand{\ie}{\emph{i.e.\/}} % i.e.

\title{
XORP Inter-Process Communication Library Overview \\
\vspace{1ex} Version 0.4
}
\author{ XORP Project					\\
	 International Computer Science Institute	\\
	 Berkeley, CA 94704, USA			\\
	 {\it feedback@xorp.org}
}
\date{August 28, 2003}

\begin{document}
\maketitle
\begin{abstract}

Extensibility and robustness are key goals of the eXtensible Open
Router Project (XORP).  A step towards satisfying both of these goals
is to separate the functional components of a router into independent
tasks and to execute each of these tasks as separate processes.  From
an extensibility perspective this allows components of the router to
be started, stopped, and exchanged dynamically, and to be distributed
across a number of hosts.  From a robusness perspective, the processes
are afforded the protection mechanisms afforded by modern operating
systems so a failure of one routing task does not necessarily bring
the router to a halt. The routing tasks do need to communicate and we
have developed an asynchronous remote procedure call mechanism that is
capable of working with multiple transport protocols between remote
hosts and can leverage existing IPC mechanisms within a single
host. This document discusses aspects of the design and the directions
it may take in future.
\end{abstract}

\section{Introduction}

Robustness and extensibility are two of the goals of the XORP project.
One way a router can achieve robustness is to run routing protocols in
protected environments, such as separate userland processes on a
modern operating system.  And one way a router can achieve
extensibility is to be independent of the details about where those
routing processes are running and what the composition of the
forwarding plane is.  The routing processes and network interfaces
could be running on one machine or distributed across a cluster of
machines that appear as single router.  A necessary feature once
routing protocols are running in distinct processes and potentially on
distinct machines is an inter-process communication mechanism.  In
contrast to traditional inter-process communication schemes, the
scheme employed in the XORP project can utilize multiple transport
protocols and potentially communicate with unmodified components
through these protocols, for instance SNMP or HTTP.

The lofty goals of XORP's Inter-Process Communication (XIPC) scheme are:

\begin{itemize}

\item to provide all of the IPC communication mechanisms that a router is
likely to need, \eg sockets, ioctl's, System V messages, shared memory.

\item to provide a consistent and transparent interface irrespective
of the underlying transport mechanism used.

\item to transparently select the appropriate IPC mechanism when
alternatives exist.

\item to provide an asynchronous interface.

\item to be efficient.

\item to potentially wrapper communication with non-XORP processes,
\eg HTTP and SNMP servers.

\item to be renderable in human readable form so XORP processes can
read and write commands from configuration files.

\end{itemize}

The XIPC goals are realized through XORP Resource Locators (XRLs) and
the XORP IPC (XIPC) library.  XRLs are responsible for describing an
inter-process calls and their arguments.  An XRL may be represented in
human readable form that allows for easy manipulation with editing
tools and invocation from the command line during development.

XORP processes export XRL interfaces to a process known as the {\em
Finder} and inform it of which IPC schemes are available to invoke
each XRL.  The Finder is then able to provide a resolution service for
XORP processes.  When a process needs to dispatch an XRL it first
contacts the Finder, obtains details on which IPC mechanisms are
available to contact the process, and then instantiates
a suitable transport.

The XIPC library provides the framework for handling and manipulating
XRLs, communicating with the Finder, and common protocol families for
conducting IPC.  In addition to the XIPC library, an interface
definition language exists, together with tools to translate these
into callable C++ interfaces and into a set of C++ handler routines
for handling the receipt of XRL calls.  These tools are described in
document \cite{xorp:xrl_interfaces}.  The tools reduce the amount of
familiarity the working programmer needs to have with the internals of
the XIPC library.  This document provides an overview of the XIPC
library and is the recommended starting point before using the
library.

\section{XORP Resource Locators (XRL's)}

The mechanism we've settled on for IPC within XORP processes is mediated
through \emph{Xorp Resource Locators} (XRL's).  An XRL describes a
procedure call.  It comprises the protocol family to be used for
transport, the arguments for the protocol family, the interface of the
target being called and its version, the method, and an argument
list.  Examples of XRLs in their human readable forms are shown in
figure \ref{fig:human-readable}.  The existence of a human readable
form for XRLs is chiefly a convenience for humans who need to work
with XRLs and not indicative of how they work internally.

Resolved and unresolved forms of the same are XRL are depicted in
figure \ref{fig:human-readable}.  The unresolved form is the starting
point for the majority of inter-process communication.  In the
unresolved form the protocol family is set to ``finder'' and the
protocol parameters set to the target name that the XRL call is
intended for.  A process wishing to dispatch an XRL for the first time
passes the unresolved XRL to the Finder, which returns the resolved
form with the appropriate protocol family and protocol family
arguments.  The finder may also modify the other components of the
XRL, but doesn't usually do so.  This functionality may be useful when
supporting backwards compatibility of interfaces, \ie the Finder could
modify the interface number and method name.

The resolved forms of XRLs are typically maintained in a client side
cache so the Finder need not be consulted for each XRL dispatch.

\clearpage
\begin{sidewaysfigure}
(a) Unresolved form:

\begin{centering}
\small\begin{verbatim}
finder://fea/fti/0.1/add_route?net:ipv4net=10.0.0.1/8&gateway:ipv4=192.150.187.1
+-----   +-- +-- +-- +-------- +------------------------------------------------
|        |   |   |   |         |
|        |   |   |   Method    Arguments
|        |   |   |
|        |   |   Interface version 
|        |   |
|        |   Interface Name  
|        | 
|        Protocol Parameters
|
Protocol Family
\end{verbatim}
\end{centering}
\normalsize

(b) Resolved form:

\small
\begin{verbatim}
stcp://192.150.1.5:1992/fti/0.1/add_route?net:ipv4net=10.0.0.1&gateway:ipv4=192.150.1.1
+---   +--------------  +-- +-- +-------- +--------------------------------------------
|      |                |   |   |         |
|      |                |   |   Method    Arguments
|      |                |   |
|      |                |   Interface version 
|      |                |
|      |                Interface Name  
|      | 
|      Protocol Parameters
|
Protocol Family
\end{verbatim}
\normalsize
\caption{\label{fig:human-readable}Human readable XRL forms.}
\end{sidewaysfigure}
\clearpage

\section{XRL Targets and the Finder}

An {\em XRL Target} is an entity that is capable of dispatching XRLs.
The {\em Finder} is the entity in XIPC system that knows about XRL
Targets and is responsible for directing IPC communication between XRL
Targets.  The IPC communication schemes that XRL Targets use to
communicate XRLs between each other once they have been resolved by
the Finder are known as {\em Protocol Families}.

Each XRL Target has an associated class name and an instance name.
Class names indicate the functionality that the target implements and
there may be multiple targets in the XIPC system with the same class
name.  Instance names are unique identifiers for each target in the
XIPC system.  XRLs to be resolved by the Finder may address a target
by class name or by instance name.  The Finder treats the first
instance in a class as the primary instance of that class and XRLs
directed to a class resolve equivalently to the primary instance.

The class of an XRL Target dictates which methods the target supports.
Interfaces are collections of XRLs that relate to some aspect of
functionality and objects in a given class implement handlers for one
or more interfaces.  Once an XRL Target has registered its presence
with the Finder, it registers the XRLs associated with the interfaces
it supports. XRLs are registered one at a time and only when all of
the XRLs have been registered does the XRL Target become visible to
the outside world through the Finder.

The Finder provides a one-to-many mapping service.  Each XRL Target
may specify multiple protocol families for each XRL they export. When
a target requests the resolution of an XRL, the answer will contain
the complete list of available resolutions and the resolver can decide
which it would prefer to use.  When registering the target indicates
the XRL method name and an appropriate protocol family.  so targets
that support multiple protocol families perform a separate
registration each XRL and protocol family pair.  This provides a
flexible system for implementing optimizations on the level of
individual XRLs.

% When a process starts up, it registers it's XRL targets with the
% Finder, and registers the XRLs supported by each target individually.
% During registration of an XRL, both the method name and the available
% protocol families for executing the XRL are communicated to the
% Finder.  The first XRL Target instantiated in a particular class is
% considered by the Finder to be the primary instance of that class.
% When the Finder receives XRLs to resolve that refer to a class name as
% a target, it resolves them as if the XRL contained the instance name
% of the primary instance in the class.  Should the XRL Target
% corresponding to the primary instance of a class leave the XRL system,
% because of a process exiting or explicit deregistration, then the next
% instance of the class that registered with the Finder is used as the
% primary instance.

There is a slight subtlety with registration for the sake of security.
This arises because some of the protocol families that can be used to
communicate XRLs and responses to XRLs allow for remote access,
examples being UDP and TCP protocol families.  The source code to the
XORP project is publicly available and the XRL interfaces and targets
are included the package.  With knowledge of the XRL interfaces and
targets it would be relatively straightforward for external parties to
opportunistically dispatch XRLs on a XORP router.  However, if XRL
communication can be coerced to use the Finder, access control can be
enforced centrally.  To achieve this, the Finder performs XRL method
name transformation.  When an XRL is registered with the Finder, the
Finder transforms the resolved form of the XRLs method name, and
instructs the registering target to dispatch the transformed name as
if it were the original name.  The method name transformation is
designed to be hard to counter and each XRL method is transformed
uniquely.  The registering process therefore has to maintain a mapping
table from the interface.

In addition to handle XRL registrations and resolutions, the Finder is
capable of notifying XRL Targets about events it is aware of, like the
birth and death of other XRL Targets.  The Finder exposes an XRL
interface for this purpose and is able to invoke XRLs on XRL Targets
to perform the notification.  A special tunneling mechanism exists in
the communication protocol used to communicate with the Finder for
this purpose.  The details of this communication will be expanded upon
later on [XXX in a later edit to this document].

\section{Components of XRL Framework}

\begin{description}
\item [XRL] an inter-process call that is transparent to the
underlying transport method.

\item [Finder] the process that co-ordinates the resolution of target
names into a parseable form to find the XRL Protocol Family Listener.

\item [XRL Router] an object responsible for sending and receiving
XRLs.  They manage all the underlying interactions and are the
interface that users are expected to use for XRL interactions.
% TODO: the above description of [XRL Router] is unclear.

\item [Finder Client] an object associated with an XRL Router that
manages the communication with the Finder.

\item [XRL Protocol Family] a supported transport mechanism for the
invoked XRL.

\item [XRL Protocol Family Sender] an entity that dispatches XRL
requests and handles responses.  Senders are created based on Finder
lookup's of the appropriate communication mechanism.

\item [XRL Protocol Family Listener] an entity that listens for
incoming requests, dispatches the necessary hook, and sends the
responses.  When Listeners are created they register the appropriate
mapping with the Finder so that corresponding Senders can be
instantiated to talk with them.

\end{description}

The kdoc documentation provides details of the particular classes.

%%%%%%%%%%%%%%%%%%%%%%%%%%%%%%%%%%%%%%%%%%%%%%%%%%%%%%%%%%%%%%%%%%%%%%%
%     BIBLIOGRAPHY
%%%%%%%%%%%%%%%%%%%%%%%%%%%%%%%%%%%%%%%%%%%%%%%%%%%%%%%%%%%%%%%%%%%%%%%
\bibliography{../tex/xorp}
\bibliographystyle{plain}

\end{document}
