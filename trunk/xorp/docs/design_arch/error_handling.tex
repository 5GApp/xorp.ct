\documentclass[11pt]{article}
\usepackage{xspace}
\usepackage{times}
\usepackage{psfig}
\usepackage{amsmath}
\newcommand{\module} {{\em module}\@\xspace}
\newcommand{\modules} {{\em modules}\@\xspace}
\newcommand{\finder} {{\em Finder}\@\xspace}
\newcommand{\xorpsh} {{\em Xorpsh}\@\xspace}
\newcommand{\cm} {{\em CM}\@\xspace}
\newcommand{\xrl} {{\em XRL}\@\xspace}
\textwidth 6.5in
\topmargin 0.0in
\textheight 8.5in
\headheight 0in
\headsep 0in
\oddsidemargin 0in
%\date{}
\title{Error Handling in XORP}
%\author{Atanu Ghosh}
%\twocolumn
\begin{document}
\parsep 0ex
\parskip 1.0ex
\parindent 0em
\noindent
\maketitle                            
\section{Introduction}

A xorp router is made up of a number processes that communicate via
XRLS [REF] (a messaging system developed for XORP). In this document
we will focus on how to deal with errors that are generated directly
or indirectly by \xrl calls. As well as dealing with process failure
and restarts. Most XORP processes share routing state that must remain
synchronised. The BGP process sends the result of its decision process
to the RIB. If the RIB fails then BGP has lost the ability to
manipulate the FIB. BGP should either withdraw all routes or more
catastrophically drop all peerings until the RIB restarts.

A critical component of the \xrl infrastructure is the \finder. The
\finder is the rendezous point for the \xrl system.  As the \finder is
the rendezous point for \xrl communication it is able to detect
process failures and restarts.

In any complex system such as a xorp router errors can occur. The
errors can range from a xorp process failing to an attempt to install
a route that already exists. Errors will occur and need to be dealt
with in a consistent manor. The types of error that may occur are
categorised below.

The first type of error a {\em Communication Error}. At the most basic
level an attempt to send an \xrl has failed. The process that was the
recipient of the \xrl may have failed or be slow to respond. The
message that was being sent may have been lost in transit.

The second type of error {\em Execution Error} is when an an XRL call
returns an error due to some underlying interaction failure. The most
fundamental example of this type of error is a route add failing. Lets
consider what interactions occur when a route is passed from BGP to
RIB to FEA. BGP receives an update packet, the update packet will
contain one or both of withdraws and nlris. These components of the
update packet may cause the BGP decision process to be rerun. A
consequence of the running of the decision process may be the addition
or deletion of a route.

The third type of error {\em Type Error} is when an XRL call fails
because the arguments passed to an XRL are invalid. This error will
most likely be due to a version mismatch between xorp processes. If
all the processes in a XORP router have been built from the source
same tree this error should not occur. As we are building an
extensbile router it may be the case that a process built from a
different source tree may encounter compatibility problems.

\subsection{XRL Error Handling}
Interprocess communication in XORP is achieved using XRLs. In this
section we will consider what should be done when an XRL call fails
due to a communication error.

All XRL calls will ultimately get a response. In the normal case the
response returns the status of the call (good or bad). In addition to
error responses produced by the application, the XRL library can also
return the following error responses:
\begin{itemize}
\item NO\_FINDER
\item RESOLVE\_FAILED
\item SEND\_FAILED
\item REPLY\_TIMED\_OUT
\end{itemize}
From an application point or view, the first three errors are
equivalent: the XRL was not communicated to the destination.  We will
discuss these below using the generic term {\em XRL send failure}.

However, its not clear what can be inferred from a timeout
response. The reasons for a timeout can be: the peer has died, peer is
slow to respond, the network cable has been removed. As in all network
communications when a timeout occurs we don't know if the last
unacknowledged XRL request was received and processed by the peer.

If the timeout has occurred because the peer has died we will receive
notification of this explicitly and will deal with it as specified in
section \ref{pfailure}.  Thus an XRL transport error SHOULD NOT be
taken as an indication that the peer is dead.  If an application cares
that the peer has died or restarted, it SHOULD register with the
finder to receive notifications of process restarts.  This a process
SHOULD assume that an XRL transport problem will be transient until it
receives an explicit confirmation that the destination has failed. 

XRLs can be sent over unreliable transports such as UDP or reliable
transports such as TCP. The type of transport that should be used will
be specified when defining the interface. In the case of reliable
transport, the errors above should generally not occur, but in any
event we need general rules about how to handle them should something
fail in an unexpectedly way.

In addition, the way the application uses an XRL interface can by
pipelined or non-pipelined.  In the pipelined case, multiple requests
can be outstanding simultaneously; in the non-pipelined case at most
one request can be outstanding at a time.

It is useful for us to categorise XRL interfaces along these two axes:
reliable/unreliable and pipelined/non-pilelined.

\subsubsection*{Unreliable, Non-pipelined}

If an XRL send failure occurs, the sending application MAY choose to
retransmit the XRL, or ignore the failure as it sees fit.  

In an XRL timeout occurs, the sending application MAY also choose to
retransmit the XRL, or ignore the failure as it sees fit.  However, if
the application chooses to re-send the XRL, the interface MUST be
written in such a way that if this XRL had previously been received,
this will not cause a further failure.

\subsubsection*{Reliable, Non-pipelined}

If an XRL send failure occurs, the sending application SHOULD
retransmit the XRL.  In an XRL timeout occurs the sending application
SHOULD also retransmit the XRL.  Further requests using this interface
MUST be queued until the XRL has successfully been received.

The interface should be written in such a way that if this XRL had
previously been received, this will not cause a further failure.

An alternative strategy is possible.  If the XRL in question changes
state at the receiving application, the interface may also support a
query mechanism.  If the XRL fails with a timeout, the sending
application may opt not to blindly re-transmit the XRL, but instead
send a query (retransmitted as necessary) to determine whether the
state at the remote system is as it would be if the XRL had been
received.  Only if it the query indicates that the state was not
received would the original XRL be retransmitted.

This alternative is more complicated, and so it should only be
prefered when the consequences of receiving the same XRL twice
outweight the additional complexity.

\subsubsection*{Unreliable, Pipelined}

The same issues apply as with unreliable, non-pipelined, but the
situation is more complicated.  An interface that uses unreliable
transport and pipelining is one that explicitly permits loss and {\em
re-ordering} of requests.  It is up to the application to choose
whether to retransmit XRLs that return XRL send failed or timeout, but
the application must only do so if it is certain that the re-ordering
caused by retransmission will not be a problem.

\subsubsection*{Reliable, Pipelined}

Reliable, pipelined interfaces are the most difficult in which to
handle XRL errors.  Three issues need to be considered:
\begin{itemize}
\item The XRL that failed due to a transport error may be followed by
pipelined XRLs that succeeded.
\item The XRL that failed due to a transport error may be followed by
pipelined XRLs that failed at the application level due to the state
caused by the first failed XRL not being instantiated.
\item If a failed XRL was followed by pipelined XRLs that succeeded,
retransmitting that XRL will cause a re-ordering that might leave the
destination in a different state than it would be if the XRLs had
arrived in order.
\end{itemize}
To avoid all these problems we require that a reliable XRL transport
fail to deliver all subsequent XRLs in the pipeline if a single XRL
fails.  Thus the reliable pipelined interface falls back to a
reliable, non-pipelined interface after a failure.  A subsequent XRL
that succeeds then permits pipelined operation to resume.

\subsection{BGP}

\section{\label{pfailure}Process Failure}

A XORP router is made up of a number of distinct processes. There are
dependencies between these processes. We define the critical
dependencies and what action to take on detecting failure.

The most critical component of a XORP router is the Rtrmgr/\finder
process. One of the functions of this component is to start/re-start
processes. If process A is dependent on the status (alive, dead,
restarted) of process B, then process A registers this interest with
the \finder. This dependency on the Rtrmgr/\finder for managing and
monitoring process state mean that a XORP router cannot survive the
failure of this process. A XORP process on detecting the loss of the
\finder must exit. There is one exception to this rule the \xorpsh
process, this will be discussed later [REF].



Each process in a XORP router is described and how it should behave
when another process in the system fails. Processes can explicitly
register interest in the status of other processes through the
\finder. If process A is dependent on the state of process B then
process A must register interest in process B.

The only mechanism that a process should use to determine if 

It is conceivable that the \finder could restart and in the same time
window another monitored process could restart. In which case the
restarting of the monitored process could be missed. To guard against
this possible race if the \finder fails or restarts all processes
should exit. The assumption is that the \finder failing is a very low
probability event.

\begin{table}[ht]
\begin{center}
\begin{tabular}{|c|c|c|c|c|c|c|c|c|c|c|}
\hline
Process fails &                &          &      &      &      &         &      &      &      &        \\\hline
              & Rtrmgr/\finder & FEA      & MFEA & RIB  & IGMP & PIM     & BGP  & RIP  & OSPF & XORPSH \\\hline
Rtrmgr/finder & /              & Withdraw & Exit & Exit & Exit & Exit    & Exit & Exit & Exit & Report \\
              &                & All      &      &      &      &         &      &      &      & Problem\\
              &                & Unicast  &      &      &      &         &      &      &      & Wait   \\
              &                & Routes   &      &      &      &         &      &      &      &        \\
              &                & Exit     &      &      &      &         &      &      &      &        \\\hline
FEA           &  Restart       & /        & Exit & Exit & Exit & Exit    & Exit & Exit & Exit & Exit   \\\hline
MFEA          &  Restart       & -        & /    & -    & Exit & Exit    & -    & -    & -    & -      \\\hline
RIB           &  Restart       & Withdraw & /    & -    & Exit & Exit    & Exit & Exit & Exit & Exit   \\
              &                & All      &      &      & (G)  & (G)     & (G)  & (G)  & (G)  & (G)    \\
              &                & Unicast  &      &      &      &         &      &      &      &        \\
              &                & Routes   &      &      &      &         &      &      &      &        \\\hline
IGMP          &  Restart       & -        & -    & -    & /    & Delete  & -    & -    & -    & -      \\
              &                &          &      &      &      & Local   &      &      &      &        \\
              &                &          &      &      &      & Members &      &      &      &        \\
              &                &          &      &      &      & After   &      &      &      &        \\
              &                &          &      &      &      & Timeout &      &      &      &        \\\hline
PIM           &  Restart       & -        & -    & -    & -    & /       & -    & -    & -    & -      \\\hline
BGP           &  Restart       & -        & -    & -    & -    & -       & /    & -    & -    & -      \\\hline
BGP           &  Restart       & -        & -    & -    & -    & -       & /    & -    & -    & -      \\\hline
RIP           &  Restart       & -        & -    & -    & -    & -       & -    & /    & -    & -      \\\hline
OSPF          &  Restart       & -        & -    & -    & -    & -       & -    & -    & /    & -      \\\hline
XORPSH        &  Restart       & -        & -    & -    & -    & -       & -    & -    & -    & /      \\\hline
\end{tabular}
\end{center}
\caption{\label{stt}Action to take on detecting process failure}
\end{table}


\subsection{RTRMGR/\finder - Router manager}


\subsection{FEA - Forwarding Engine Abstraction}
The FEA primarily excepts routes from the RIB and places them in the
kernel. The FEA should tag all routes that it has installed in the
kernel. The FEA on restart should remove all routes that a previous
incarnation of the FEA has placed in the kernel. When an FEA is
exiting it should attempt remove all routes that it has installed in
the kernel.

The FEA process should register interest in the RIB. If the RIB fails
the FEA should withdraw all routes that the RIB has sent it.

\subsection{RIB - Routing Information Base}
Routes from the routing processes are sent to the RIB the winners are
sent to the FEA.

The RIB should register interest in the FEA. If the FEA fails the RIB
needs to notify all the routing processs that are feeding it routes
that it can no longer accept routes. The simplest way to do this may
be for the RIB to exit. Another solution may be an upcall to the
routing processes saying control plane gone.


\subsection{RIP}

\subsection{IS-IS}

\subsection{OSPF}

\subsection{PIM}

\subsection{BGP}
The major interaction that the BGP process has is with the RIB. At the
time of writing the BGP process creates and manages its own TCP
connections (In the future it may be the case that the TCP connections
are mediated through the FEA). 

The BGP process should register interest in status changes of the RIB.
If the BGP process detects that the RIB has failed then it should
withdraw all routes that have been sent to its peers and wait for the
RIB to restart. A more drastic solution would be to tear out all
peerings and wait for the RIB to restart.

\bibliographystyle{plain}
\bibliography{xorp}
\end{document}
