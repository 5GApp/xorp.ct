\documentclass[11pt]{article}
\usepackage{graphicx}
\usepackage{times}
\usepackage{xspace}
\usepackage{alltt}
\usepackage{stmaryrd}
\textwidth 6.5in
\topmargin 0.0in
\textheight 8.5in
\headheight 0in
\headsep 0in
\oddsidemargin 0in
\parskip 0in

%\newcommand{\xorpsh}{\textsc{xorpsh}}
\newcommand{\xorpsh}{{\sf\small xorpsh}\xspace}

\title{XORPSH User Guide\\
Part 1: Command Structure\\
\vspace{1ex}
Version 0.3}
\author{ XORP Project					\\
	 International Computer Science Institute	\\
	 Berkeley, CA 94704, USA			\\
	 {\it feedback@xorp.org}
}
\date{June 4, 2003}
%\twocolumn
\begin{document}
\maketitle                            
\section{Introduction}
To interact with a XORP router using the command line interface (CLI),
a user runs \xorpsh.  This allows configuration of the router and
monitoring of the router state.  

In this document we describe how to interact with \xorpsh.  This
document does not provide specifics of how to configure BGP, PIM, SNMP
and other processes - this will eventually be described in additional
documents.

The user interface style is loosely modelled on that of a Juniper
router.

\section{Running xorpsh}
\xorpsh provides an interactive command shell to a XORP user, similar
in many ways to the role played by a Unix shell.  In a production
router, it is envisaged that \xorpsh might be set up as a user's login
shell - they would login to the router via ssh and be directly in the
\xorpsh environment.  However, for research and development purposes,
it makes more sense to login normally to the machine running the
rtrmgr process, and to run \xorpsh directly from the Unix command line.

\xorpsh should normally be run as a regular user; it is neither
necessary or desirable to run it as root.  If the user is to be
permitted to make changes to the running router configuration, they
need to be in the Unix group {\tt xorp}.  The choice of GID of group
xorp is not important.

\xorpsh needs to be able to communicate with the rtrmgr using the local
file system.  If the rtrmgr cannot write files in /tmp that the \xorpsh
can read, then \xorpsh will not be able to authenticate the user to the
rtrmgr.

Multiple users can run \xorpsh simultaneously.  There is some degree of
configuration locking to prevent simultaneous changes to the router
configuration, but currently this is fairly primitive.

\section{Basic Commands}

On starting \xorpsh, you will be presented with a command line prompt:
\vspace{0.1in}

\noindent\fbox{
\begin{minipage}{4in}
\begin{alltt}
Xorp>
\end{alltt}
\end{minipage}
}
\vspace{0.1in}

\noindent
You can exit \xorpsh at any time by trying Control-d.

\noindent
Typing ``?'' at the prompt will list the current command available to
you:
\vspace{0.1in}

\noindent\fbox{
\begin{minipage}{6in}
\begin{alltt}
\begin{tabbing}
Xorp> \textbf{?}\\
Po\=ssible compl\=etions:\\
\>configure       \>Switch to configuration mode\\
\>help            \>Provide help with commands\\
\>quit            \>Quit this command session\\
\>show            \>help
\end{tabbing}
\end{alltt}
\end{minipage}
}
\vspace{0.1in}

\noindent
If you type the first letter or letters of a command, and hit
{\tt <Tab>}, then command completion will occur.

\noindent
At any time you can type ``?'' again to see further 
command completions.  For
example:
\vspace{0.1in}

\noindent\fbox{
\begin{minipage}{4in}
\begin{alltt}
\begin{tabbing}
Xorp> \textbf{config?}\\
Po\=ssible compl\=etions:\\
\>configure\>Switch to configuration mode\\
Xorp> \textbf{config}
\end{tabbing}
\end{alltt}
\end{minipage}
}
\vspace{0.1in}

\noindent
If the cursor is after the command, typing ``?'' will list the possible
parameters for the command:
\vspace{0.1in}

\noindent\fbox{
\begin{minipage}{4in}
\begin{alltt}
\begin{tabbing}
Xorp> \textbf{configure ?}\\
Po\=ssible compl\=etions:\\
\><[Enter]>       \>Execute this command\\
\>exclusive       \>Switch to configuration mode, locking out other users\\
\>|               \>Pipe through a command\\
Xorp> \textbf{configure}
\end{tabbing}
\end{alltt}
\end{minipage}
}

\subsection{Command History and Command Line Editing}

\xorpsh supports emacs-style command history and editing of the text
on the command line.  The most important commands are:
\begin{itemize}
\item The {\bf up-arrow} or {\bf control-p} moves to the previous
command in the history.
\item The {\bf down-arrow} or {\bf control-n} moves to the next
command in the history.
\item The {\bf left-arrow} or {\bf control-b} moves back along the
command line.
\item The {\bf right-arrow} or {\bf control-f} move forward along the
command line.
\item {\bf control-a} moves to the beginning of the command line.
\item {\bf control-e} moves to the beginning of the command line.
\item {\bf control-d} deletes the character directly under the cursor.
\item {\bf control-t} toggles (swaps) the character under the cursor with
the character immediately preceding it.
\item {\bf control-space} marks the current cursor position.
\item {\bf control-w} deletes the text between the mark and the current
cursor position, copying the deleted text to the cut buffer.
\item {\bf control-k} kills (deletes) from the cursor to the end of the
command line, copying the deleted text to the cut buffer.
\item {\bf control-y} yanks (pastes) the text from the cut buffer,
inserting it at the
current cursor location.
\end{itemize}

\section{Command Modes}

\xorpsh has two command modes:
\begin{description}
\item{\bf Operational Mode,}  which allows interaction with the router
to monitor it's operation and status.
\item{\bf Configuration Mode,} which allows the user to view the
configuration of the router, to change that configuration, and to
load and save configurations to file.
\end{description}
Generally speaking, operational mode is considered to give
non-priviledged access; there should be nothing a user can type that
would seriously impact the operation of the router.  In contrast,
configuration mode allows all aspects of router operation to be
modified.

In the long run, \xorpsh and the rtrmgr will probably come to support
fine-grained access control, so that some users can be given
permission to change only subsets of the router configuration.  At the
present time though, there is no fine-grained access control.

A user can only enter configuration mode if they are in the xorp Unix
group.

\subsection{Operational Mode}
\noindent\fbox{
\begin{minipage}{6in}
\begin{alltt}
\begin{tabbing}
Xorp> \textbf{?}\\
Po\=ssible compl\=etions:\\
\>configure       \>Switch to configuration mode\\
\>help            \>Provide help with commands\\
\>quit            \>Quit this command session\\
\>show            \>help
\end{tabbing}
\end{alltt}
\end{minipage}
}
\vspace{0.1in}

The main commands in operational mode are:
\begin{description}
\item{\bf configure}: switches from operational mode to configuration
mode.
\item{\bf help}: provides online help (not yet implemented)
\item{\bf quit}: quit from xorpsh.  (not yet implemented - use
control-d instead).
\item{\bf show}: displays many aspects of the running state of the
router.
\end{description}
\subsubsection{Show Command}
\noindent\fbox{
\begin{minipage}{6in}
\begin{alltt}
\begin{tabbing}
Xorp> \textbf{show ?}\\
Po\=ssible compl\=etions:\\
\>  bgp             \>help\\
\>  interface       \>help\\
\>  vif             \>help\\
Xorp> \textbf{show}
\end{tabbing}
\end{alltt}
\end{minipage}
}
\vspace{0.1in}

\noindent
The show command is used to display many aspects of the running state
of the router.  We don't describe the sub-commands here, because they
depend on the running state of the router.  For example, only a router
that is running BGP should support {\tt show bgp}.  

As an example, we show the peers of a BGP router:
\vspace{0.1in}

\noindent\fbox{
\begin{minipage}{6in}
\begin{alltt}
\begin{tabbing}
Xorp> \textbf{show bgp peers detail}\\
OK\\
\\
Pe\=er 1: local 192.150.187.108/179 remote 192.150.187.109/179\\
\>  Peer ID: 192.150.187.109\\
\>  Peer State: ESTABLISHED\\
\>  Admin State: START\\
\>  Negotiated BGP Version: 4\\
\>  Peer AS Number: 65000\\
\>  Updates Received: 5157,  Updates Sent: 0\\
\>  Messages Received: 5159,  Messages Sent: 1\\
\>  Time since last received update: 4 seconds\\
\>  Number of transitions to ESTABLISHED: 1\\
\>  Time since last entering ESTABLISHED state: 47 seconds\\
\>  Retry Interval: 120 seconds\\
\>  Hold Time: 90 seconds,  Keep Alive Time: 30 seconds\\
\>  Configured Hold Time: 90 seconds,  Configured Keep Alive Time: 30 seconds\\
\>  Minimum AS Origination Interval: 0 seconds\\
\>  Minimum Route Advertisement Interval: 0 seconds\\
\end{tabbing}
\end{alltt}
\end{minipage}
}
\vspace{0.1in}

\newpage
\subsection{Configuration Mode}
\noindent\fbox{
\begin{minipage}{6in}
\begin{alltt}
\begin{tabbing}
Xorp> \textbf{configure}\\
Entering configuration mode.\\
There are no other users in configuration mode.\\
\end{tabbing}
\end{alltt}
\end{minipage}
}
\vspace{0.1in}

\noindent
When in configuration mode, the command prompt changes to be all
capitals.
The command prompt is also usually preceded by a line indicating which
part of the configuration tree is currently being edited.
\vspace{0.1in}

\noindent\fbox{
\begin{minipage}{6in}
\begin{alltt}
\begin{tabbing}
[edit]\\
XORP> \textbf{?}\\
Po\=ssible completi\=ons:\\
\>delete  \>    Delete a configuration element\\
\>edit    \>    Edit a sub-element\\
\>exit    \>    Exit from this configuration level\\
\>help    \>    Provide help with commands\\
\>interfaces\>help\\
\>load    \>    Load configuration from a file\\
\>protocols\> help\\
\>quit    \>    Quit from this level\\
\>run     \>    Run an operational-mode command\\
\>save    \>    Save configuration to a file\\
\>set     \>    Set the value of a parameter\\
\>show    \>    Show the value of a parameter\\
\>top     \>    Exit to top level of configuration\\
\>up      \>    Exit one level of configuration
\end{tabbing}
\end{alltt}
\end{minipage}
}
\vspace{0.1in}

\noindent
The router configuration has a tree form similar to the directory
structure on a Unix filesystem.  The current configuration or parts of
the configuration can be
shown with the show command:
\vspace{0.1in}

\noindent\fbox{
\begin{minipage}{6in}
\begin{alltt}
\begin{tabbing}
xx\=xx\=xx\=xx\=xx\=\kill
[edit]\\
XORP> \textbf{show interfaces}\\
\>interface rl0 \{\\
\>\>description: "control interface"\\
\>\>vif rl0 \{\\
\>\>\>address 192.150.187.108 \{\\
\>\>\>\>prefix-length: 25\\
\>\>\>\>broadcast: 192.150.187.255\\
\>\>\>\>disable: false\\
\>\>\>\}\\
\>\>\>disable: false\\
\>\>\}\\
\>\>disable: false\\
\>\}\\
\end{tabbing}
\end{alltt}
\end{minipage}
}
\vspace{0.1in}

\noindent
You can change the current location in the configuration tree using
the edit, up, top and exit commands.
\begin{itemize}
\item \textbf{edit $<$\textit{element name}$>$}:       Edit a sub-element
\item \textbf{exit}:       Exit from this configuration level
\item \textbf{quit}:       Quit from this level
\item \textbf{top}:        Exit to top level of configuration
\item \textbf{up}:         Exit one level of configuration
\end{itemize}

\noindent
For example:
\vspace{0.1in}

\noindent\fbox{
\begin{minipage}{6in}
\begin{alltt}
\begin{tabbing}
xx\=xx\=xx\=xx\=xx\=\kill
[edit]\\
XORP> \textbf{edit interfaces interface rl0 vif rl0}\\
\\
\noindent[edit interfaces interface rl0 vif rl0]\\
XORP> \textbf{show}\\
\>address 192.150.187.108 \{\\
\>\>prefix-length: 25\\
\>\>broadcast: 192.150.187.255\\
\>\>disable: false\\
\>\}\\
\>disable: false\\
\\
\noindent[edit interfaces interface rl0 vif rl0]\\
XORP> \textbf{up}\\
\\
\noindent[edit interfaces interface rl0]\\
XORP> \textbf{top}\\
\\
\noindent[edit]\\
XORP>
\end{tabbing}
\end{alltt}
\end{minipage}
}
\vspace{0.1in}


\end{document}


