\section{The XORP Vision}

XORP stands for eXtensible Open Router Platform.  It is called an {\it
extensible} router platform for good reason.  We believe that only by
designing for extensibility from the outset can we simultaneously
satisfy several different user groups:
\begin{itemize}
\vspace{-0.07in}
\item Network researchers needing a platform for experimentation.
\vspace{-0.07in}
\item Network operators needing a low-cost stable routing platform on
commodity hardware.
\vspace{-0.07in}
\item Network equipment vendors with special purpose hardware.
\vspace{-0.07in}
\item Network application writers looking for an open platform to
support their applications.
\end{itemize}
\vspace{-0.05in}
Only time will tell if we're right, but in this paper we'll try and
explain where we hope to go from here.  Of course we'll only get there
with good luck and a lot of help from all of these communities.  First
we'll discuss the XORP archiecture, then return to how it helps these
diverse groups.

\subsection{XORP Overview}

XORP divides into two subsystems. The user-level subsystem consists of the routing protocols themselves, along with routing
information bases and support processes. The kernel-level subsystem manages the forwarding path and provides APIs for the user-level to access. The goal is for almost all of the user-level code to be agnostic as to the details of the forwarding path.

For user-level XORP, we've developed a multi-process architecture with
one process per routing protocol, plus extra processes for management,
configuration, and coordination.  To enable extensibility we designed
a novel inter-process communication mechanism, XORP Resource
Locators (XRLs), for communication between these modules.  

%The lower level uses the Click modular router~\cite{click}, a
%modular, extensible toolkit for packet processing on conventional PCs.
%Further work will help this architecture span a
%large range of hardware forwarding platforms, from commodity
%PC hardware, through mid-range PC-based platforms
%enhanced with intelligent network interfaces (such as Intel's
%IXP1200~\cite{ixp1200,scoutixp}), to high-end hardware-based forwarding engines.
%We may also support alternative forwarding paths, such as the
%FreeBSD forwarding path with AltQ queuing extensions~\cite{altq}
%or an alternative extensible forwarding path such as
%Scout~\cite{pathrouter}.
%Some forwarding path choices may influence the functionality available
%for end users to extend or change.  But for many aspects of research, such
%as routing protocol experiments that don't
%require access to the forwarding path, a conventional FreeBSD
%lower level would suffice.

\begin{figure}[h]
\centerline{\psfig{figure=processes3.ps,width=3.0in}}
\caption{XORP High-level Processes}
\label{fig:processes}
\vspace{-0.1in}
\end{figure}

Figure~\ref{fig:processes} shows how a XORP router's user-level processes
and Click forwarding path fit together.
The shared user-level processes are the XORP architecture's most innovative
feature. Four core processes are particularly worthy of comment: the
\emph{router manager},
the \emph{ipc finder}, the \emph{routing information base}, and the 
\emph{forwarding engine abstraction}.  

The \emph{router manager}
process manages the router as a whole.  It maintains
configuration information; starts other processes, such as routing
protocols, as required by the configuration; and restarts failed
processes as necessary.

The \emph{ipc finder} process stores mappings between abstracted application
requests, such as ``How many interfaces does this router have?'', and the
particular IPC calls necessary to answer those requests.
Think of it as an IPC redirector:
when an application wishes to make an IPC
call, it consults the finder to discover how to do it.  The
application typically caches this information so future calls
circumvent the finder.  Furthermore, the finder can instruct
applications to update contact information.  Thus it is easy to change
how application requests are handled at run-time.  We
can trace XORP's communication pattern by asking the finder to map
abstract requests to sequences of IPCs, some for tracing and some for
doing the work.  XORP processes can communicate without
bootstrapping using the finder, but since XRLs are relatively low cost we
have not found this necessary to date.

%% XXX This is duptext from Eddie and Orion editing at the same time.
%% Either this or the next paragraph should go.

%The \emph{forwarding engine abstraction} process, or FEA, abstracts the
%details of how the router's forwarding path is implemented.
%Routing protocols talk to the FEA to install routes and discover properties
%of interfaces, for example. 
%Interfaces to the FEA remain the same regardless of whether the forwarding
%path is implemented in Click, in a conventional kernel, or even in external
%hardware.
%Again, XORP processes can bypass the FEA when required.

The \emph{routing information base} process (RIB) receives routes from
the routing processes, and arbitrates which routes should be
propagated into the forwarding path, or redistributed to other routing
processes.  The forwarding path is managed
by the \emph{forwarding engine abstraction} process (FEA).  The FEA
abstracts the details of how the forwarding path of the router is
implemented and as a result, the routing processes are agnostic to
whether the forwarding plane is Click \cite{click} based, a conventional UNIX
kernel, or an alternative method.  The FEA manages the networking
interfaces and forwarding table in the router, and provides information to routing processes
about the interface properties and events occurring on interfaces, such as an
interface being taken down.  As with the finder, XORP processes can bypass the
FEA when required.


\subsection{Extensibility}

XORP's design encourages the construction of useful
interfaces through multi-process design.
A routing protocol process, for example, must communicate with other
processes to install routes and discover information about the router
itself.
Open inter-process interfaces, built in to the system from the beginning,
form the basic source of user-level XORP's extensibility.

\subsubsection{XRLs}
\def\xrl#1{\textsf{\small #1}}

Most inter-process communication within XORP takes place via XRLs.  XRLs resemble the Web's URLs. They specify in
human-readable form the type of IPC transport desired (the
``protocol''), the abstract name for the entity being communicated
with, the method being invoked, and a list of named arguments.  Unlike
URLs, they can also specify the nature of the response expected.

As an example, the general form of one of the XRLs for the forwarding
engine abstraction (FEA) process might be rendered in human readable
form as:
%
\begin{figure}[h]
\vspace{-0.05in}
\centerline{\psfig{figure=xrl.ps,width=3.5in}}
\vspace{-0.1in}
\end{figure}

\noindent The initial `\xrl{finder:}' portion specifies the protocol;
in this case the actual protocol has not yet been resolved.
The first time this XRL is called, the client XRL library
contacts the finder, which responds with a redirection to a new XRL
containing the actual protocol to be used, together with all the
parameters needed to contact the current FEA.
Subsequent communication then goes
directly between the client and the FEA process.
If the FEA restarts, the client's XRL calls to the old FEA will fail, and
it can consult the finder to update the redirection.

%The XRL library, which all of our processes link against, takes an XRL
%and performs argument marshaling, then it invokes the specified transport
%mechanism, and handles the response or any
%failure that might occur.  Unlike many RPC or remote method invocation
%mechanisms, XRLs don't try and hide from the programmer that
%off-process communication is taking place.  While this makes the
%programmer's job harder at first, it is essential for robustness that
%the programmer is made aware that the failure modes of IPC are different from those of
%a local procedure call.  
%To help the programmer and improve performance, an IDL and a stub
%generator exist, so most XRL clients never need to parse the human
%readable form.

The original motivation for XRLs was to encapsulate existing
protocols within our consistent IPC framework.   For example, we
might wish to run third-party software that uses SNMPv3 for
configuration. 
To integrate this software into our XRL-based management framework, we
might write an SNMP `protocol family' for the XRL client library.
Then XORP processes could transparently interoperate with the third-party
software via XRLs that start with `\xrl{snmp:}'.
%XRLs are general enough to encompass simple
%communication with the kernel via \texttt{ioctl}s, and even signaling
%via \texttt{kill()}.  At the present time, we have not had a need to
%add this functionality, but should the need arise, our architecture
%would support it.  The current XRL library supports
%XORP-specific protocols for remote procedure call, one layered over
%TCP and the other over UDP, and a local procedure call mechanism for
%intra-process communication.

\subsubsection{XRL Example: Command-line Interface}

One of the biggest issues faced by an {extensible} router is 
the integration of separately maintained components into a coherent
system.  
%Consider the interaction between management mechanisms such
%as a command-line interface (CLI) and a new routing protocol.  The
%author of each of the management processes has no knowledge of future
%routing protocols.  At the same time, the author of each routing
%protocol has no knowledge of future management mechanisms. 
Our solution is for all management, including initial configuration, to
take place using XRLs.  To add support for a specific management
mechanism, such as SNMP or a command-line interface, the protocol
implementor writes simple text files that map management requests to
XRL calls.  These thin mapping files are easy enough to write that
third parties might add them as new management interfaces become
available.

To get more concrete, our configuration manager has a strict hierarchy
of configuration parameters, which is directly reflected in our
default CLI.  A fragment of a router configuration file might look
like:

\begin{quotation}
\noindent \sf \small protocols ospf \{\\
\parindent 10pt 
\indent     router-id: 128.16.64.1\\
\indent     area 128.16.0.1 \{\\\
\parindent 20pt 
\indent       interface xl0 \{\\
\parindent 30pt 
\indent         hello-interval: 30\\
\parindent 20pt 
\indent       \}\\
\parindent 10pt 
\indent     \}\\
\}
\end{quotation}

The configuration manager takes a directory of template files, which
define the possible configurable parameters for each XORP routing
protocol, and generates mappings of configuration parameters to XRL
dispatches.  The designer of a new routing protocol can simply add a
template file specifying the new functionality provided.  Thus, the
template entry for OSPF might contain the fragment:

\begin{quotation}
\noindent \sf \small hello-interval: uint \{ \\
\parindent 10pt 
\indent \%set: xrl ``ospf/ospf/1.0/set\_hello\_interval?\\
\parindent 60pt 
\indent if:txt=\$(IFNAME)\&interval:i32=\$(VALUE)'';\\
\parindent 10pt 
\indent \%get: xrl ``ospf/ospf/1.0/hello\_interval?if:txt\\
\parindent 60pt 
\indent -$>$ interval:i32'';\\
\}
\end{quotation}

The configuration manager can read the template file, discover the new
functionality, and know how to communicate with the process to use it.
The new functionality is then immediately available through the CLI.

\subsection{Robustness}

The routing and coordination processes in XORP run in user space on a
traditional UNIX operating system.  Routing processes are protected
from each other and can have their resources constrained according to
administrative preference.  Furthermore, routing processes can crash
without affecting the kernel, forwarding plane, or each other.  And if
a routing protocol does crash, the RIB will remove its routes
from the forwarding engine, and optionally inform the re-starting
routing process of the routes it previously held.

%Multiple processes are used in Zebra~\cite{zebra} and Cisco's proposed ENA
%router operating system, but not by some of the larger
%commercial vendors today.

A significant aspect of robustness is security.  One benefit of
being forwarding-path agnostic is that we can abstract
privileged operations, such as sending on a raw socket, into
the FEA via XRLs.  This allows us to run many routing protocols
in a sandbox. They have no interaction with the outside world except
through XRLs and packets, and so an exploitable vulnerability in a routing
protocol is far more difficult to escalate into full control of the
router.

%Robustness in the forwarding path is also important, but solutions
%such as memory protection that work well in user-space are not
%acceptable.  In the Click forwarding path robustness comes from the
%granularity and simplicity of Click's components.  Each element is
%small enough to be well understood and tested in isolation.  And since
%many of Click's components can be run and debugged in user-space,
%confidence about their robustness can be attained before being used in
%the kernel.

\section{Satisfying Diverse Users}

We now return to examine how XORP might satisfy the different user
bases we mentioned earlier.

\subsection{Network Research}

There are a lot of very smart people in the network research
community, but it's difficult for many of these people to do experiments
that give meaningful results.  A lot of work happens in simulation,
but often it's hard to know whether the simulation bears much
relationship to reality.  There's really no substitute for trying
something out in the real world.  This is where XORP comes in.

Here are some examples of how XORP can be used:

\begin{itemize}
\item If XORP is used as a production router, it is easy to instrument
it to perform measurements of traffic, routing messages, or
practically anything else that goes on in a router.  

\item XORP can be used as a platform to develop new routing protocols.
XORP has no built-in concept of which routing protocols exists, so you
can easily add your own, together with an ASCII configuration template
file, so the XORP command line interface knows what new
functionality is available.

\item XORP can be used as a network emulator.
XORP's Forwarding Engine Abstraction (FEA) process provides an
abstract interface through which the whole control plane communicates
with the forwarding plane and the outside world.  It is possible to
modify the FEA so that multiple emulated routers all exist on the same
host.  This allows experiments using XORP's real routing code to take
place in a carefully controlled environment.  It also facilitates new
protocol development without needing a large lab of machines.

\item XORP is scriptable.  XORP's internal communication between
processes uses XRLs.
These have a canonical form that is ASCII, such as 
\vspace{-0.1in}
\begin{verbatim}
finder://fea/fti/0.1/add_route?net:ipv4net=10.0.0.1/8&\
                     gateway:ipv4=192.150.187.1
\end{verbatim}
\vspace{-0.1in} There's a command line program \textbf{call\_xrl} that
allows any scripting language to make calls to any XORP process.  We
this scipting ability is unique as an enabler for novel uses of
existing router code.
\end{itemize}

\subsection{XORP as a Low-Cost Router}

There are many organizations that need router functionality, but
cannot justify buying an expensive commercial router.  Although there
are cheap home routers available these days, they are not really
intended to run any non-trivial routing configuration.  A modern \$300
PC has enough forwarding capability to saturate a few 100Mb/s network
links, and ethernet NICs are extremely cheap, so with a little
knowledge, it's very viable to build your own PC-based router.

The goal is that you shouldn't need to know anything about how XORP
works internally to run it as a router.  XORP has a single unified
command line interface which allows all the routing protocols, network
interfaces, and so forth to be configured.  In future releases, this
CLI will also be extended to encompass additional router functionality
such as queue manangement, QoS configuration, firewalls, NATs and DHCP
configuration.

XORP already supports IPv4 and IPv6, together with BGP4+ and RIP for
unicast routing, PIM-SM and IGMPv2 for multicast, and limited SNMP
support.  

We offer XORP as a Live CD - a CD image that you can burn to a
bootable CD which allows you to run XORP without installing any
software or knowing anything about Unix system administration.  As it
runs from a readonly CD, this configuration is more secure than a
normal Unix installation.

On the subject of security, the XORP architecture permits different
routing protocols to run in different security sandboxes.  For
example, BGP does not need access to the router's filesystem or need
privileged access to communicate with its peers so, should something
go wrong, it's much harder to compromise the rest of the router.  The
aim is for XORP to be more robust and secure than alternative router
platforms.

It is very important to us that XORP is both very stable and has
sufficient features for mission-critical production use.  We'll
realise this goal only with extensive feedback from our users with
regards to what works well in the real world, and what doesn't.


\subsection{XORP for Equipment Vendors}

Once XORP has proven itself as a stable software stack for PC routers,
and its functionality and feature set filled out fully, we hope it
will prove an attractive alternative to commercial stacks for network
equipment vendors.  

XORP is licensed under a BSD-style license, which allows it to be used
for any purpose whatsoever.  Why didn't we use the GPL for our
license?  Vendors are extremely reticent to reveal details of fast
forwarding hardware.  But we believe it's also in vendors interest to
contribute back to the XORP {\it core}, even though they're not
compelled to do so.

We believe XORP's architecture is well suited as a software stack to
control an advanced hardware forwarding plane.  Our forwarding engine
abstraction (FEA) process provides a key abstraction layer providing
isolation between all the higher level routing functionality and the
underlying operating system and forwarding engines.  This should make
XORP comparatively easy to port to new platforms.

\subsection{XORP for Network Application Writers}

In the long run, we hope that XORP will enable a class of software
that currently doesn't exist: the \textit{router application}.  Currently,
there is no market for third party software for mainstream commercial
router platforms.  This is clearly because there are no open APIs for
commercial router platforms.  

We believe that XORP's extensible architecture is a possible solution
to this problem.  XORP's novel inter-process communication mechanism,
combined with it's run-time extensible router-manager process and
command line interface should permit a router operator to install a
new binary application process on a XORP router, and for it to appear
as an integrated part of the router from an operational point of view.
We are very interested to see what novel network functionality this
enables in the future.
