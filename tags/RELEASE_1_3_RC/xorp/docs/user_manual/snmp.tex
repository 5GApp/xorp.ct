%
% $XORP: xorp/docs/user_manual/snmp.tex,v 1.5 2006/03/09 06:46:34 pavlin Exp $
%

\chapter{SNMP}
\label{snmp}

\section{Terminology and Concepts}

SNMP (Simple Network Management Protocol) is a mechanism for managing
network and computer devices.
SNMP uses a manager/agent model for managing the devices. The agent
resides in the device, and provides the interface to the physical device being
managed. The manager resides on the management system and provides the
interface between the user and the SNMP agent.
The interface between the SNMP manager and the SNMP agent uses a Management
Information Base (MIB) and a small set of commands to exchange information.

The MIB contains the set of variables/objects that are managed (\eg MTU on a
network interface). Those objects are organized in a tree structure where
each object is a leaf node. Each object has its unique Object IDentifier
(OID). There are two types of objects: {\tt scalar} and {\tt tabular}.
A scalar object defines a single object instance. A tabular object defines
multiple related object instances that are grouped in MIB tables.
For example, the uptime on a device is a scalar object, but the routing table
in a router is a tabular object.

The set of commands used in SNMP are: GET, GET-NEXT, GET-RESPONSE, SET, and
TRAP. GET and GET-NEXT are used by the manager to request information about
an  object. SET is used by the manager to change the value of a specific
object. GET-RESPONSE is used by the SNMP agent to return the requested
information by GET or GET-NEXT, or the the status of the SET operation.
The TRAP command is used by the agent to inform asynchronously the manager
about the occurrence of some events that are important to the manager.

Currently there are three versions of SNMP:

\begin{itemize}

  \item {\tt SNMPv1:} This is the first version of the protocol. It is
  described in RFC 1157.

  \item {\tt SNMPv2:} This is an evolution of the first version, and it adds a
  number of improvements to SNMPv1.

  \item {\tt SNMPv3:} This version improves the security model in SNMPv2, and
  adds support for proxies.

\end{itemize}

\section{Configuring SNMP}

Before configuring SNMP on XORP, you must make sure that SNMP support is
compiled. For example, when running {\stt ./configure} in the top-level XORP
directory, you have to supply the {\stt --with-snmp} flag:

{\tt ./configure --with-snmp}

\subsection{Configuring Net-SNMP}
\label{snmp:configuring_netsnmp}

XORP itself does not implement the SNMP protocol and requires an external
SNMP implementation for that. Currently, XORP supports only Net-SNMP (see
{\stt http://www.net-snmp.org}) as such implementation.
Before configuring SNMP in XORP, you must take the following steps to
configure your Net-SNMP agent to run with XORP:

\begin{itemize}

  \item You need Net-SNMP version 5.0.6 or greater.

  \item You must make {\stt libnetsnmpxorp.so} accessible to your runtime
  loader. Depending on your system, that requires one of the following:

  \begin{itemize}

    \item Copy {\stt libnetsnmpxorp.so} to your library directory (typically
    {\stt /usr/local/lib}).

    \item Set a linker environment variable (typically
    {\stt LD\_LIBRARY\_PATH}) to point to the directory where the library is.

  \end{itemize}

  \item To avoid opening security holes, we recommend allowing only SNMPv3
  authenticated requests.  If you want to create a secure user, execute the
  command {\stt net-snmp-config --create-snmpv3-user}.  These are the settings
  that  match the provided {\stt snmp.conf} file inside the
  {\stt \$\{XORP\}/mibs/snmpdscripts/} directory:

\begin{table*}[h]
\begin{center}
\begin{tabular}{|l|l|l|} \hline
User & Pass phrase & Security level \\
\hline\hline
privuser &  I am priv user & authPriv \\ \hline
\end{tabular}
\end{center}
\end{table*}

  You must create at least one user if you want to be able to access the SNMP 
  agent.

  \item {\stt snmpd} can only respond to XRLs after
  {\stt xorp\_if\_mib\_module} has been loaded.
  Adding the following line to the file {\stt snmpd.conf} (by default located
  in {\stt /usr/local/share/snmp}) will preload this module when {\stt snmpd}
  is started:

  {\tt dlmod xorp\_if\_mib\_module <absolute path full filename>}

  For example:

  {\tt dlmod xorp\_if\_mib\_module /usr/local/xorp/mibs/xorp\_if\_mib\_module.so}
\end{itemize}

\subsection{Configuration Syntax}
\vspace{0.1in}
\noindent\framebox[\textwidth][l]{\scriptsize
\begin{minipage}{6in}
\begin{alltt}
\begin{tabbing}
xx\=xx\=xx\=xx\=xx\=\kill
protocols \{\\
\>snmp \{\\
\>\>targetname: {\it text}\\
\>\>mib-module {\it text} \{\\
\>\>\>abs-path: {\it text}\\
\>\>\>mib-index: {\it uint}\\
\>\>\}\\
\>\}\\
\}\\
\end{tabbing}
\end{alltt}
\end{minipage}
}
\vspace{0.1in}

\begin{description}

  \item {\tt protocols}: this delimits the configuration for all routing
   protocols in the XORP router configuration.  It is mandatory that
   SNMP configuration is under the {\stt protocols} node in the
   configuration.

  \item {\tt snmp}: this delimits the SNMP configuration part of the XORP
   router configuration.

  \item {\tt targetname}: this is the name for this instance of SNMP.
   It defaults to ``{\stt xorp\_if\_mib}'', and it is not recommended
   that this default is overridden under normal usage scenarios.

  \item {\tt mib-module}: this specifies the MIB module to configure. It
  should be set to the MIB module file name (without the file name extension).

  For each MIB, the following parameters can be configured:

  \begin{description}

    \item {\tt abs-path}: this is the absolute path to the module file with
    the MIB to load.

    \item {\tt mib-index}: this is the MIB index. It is set internally by
    XORP when a MIB module is loaded, and should not be set in the XORP
    configuration.

  \end{description}

\end{description}

Below is a sample SNMP configuration that configures a BGP MIB:

\vspace{0.1in}
\noindent\framebox[\textwidth][l]{\scriptsize
\begin{minipage}{6in}
\begin{alltt}
\begin{tabbing}
xx\=xx\=xx\=xx\=xx\=\kill
protocols \{\\
\>snmp \{\\
\>\>mib-module bgp4\_mib\_1657 \{\\
\>\>\>abs-path: "/usr/local/xorp/mibs/bgp4\_mib\_1657.so"\\
\>\>\}\\
\>\}\\
\}\\
\end{tabbing}
\end{alltt}
\end{minipage}
}
\vspace{0.1in}

\section{Using SNMP to Monitor a Router}

Currently (March 2006) XORP does not provide SNMP-related operational
commands.

However, there are few client-side scripts that can be used to experiment
with the SNMP agent:

\begin{itemize}
  \item The scripts are in the
  {\stt \$\{XORP\}/mibs/snmpdscripts/} directory, and they use the client-side
  Net-SNMP tools to communicate with the agent.  They rely on file
  {\stt snmp.conf} in the same directory to provide valid default values for
  the SNMP version to use, the user, community and security level. If your
  agent was configured with the default security user suggested in
  Section~\ref{snmp:configuring_netsnmp}, you should copy the
  {\stt \$\{XORP\}/mibs/snmpdscripts/snmp.conf} file to
  {\stt \$\{HOME\}/.snmp/snmp.conf}. Otherwise, you'll have to create your
  own {\stt snmp.conf} so it matches your settings.

  \item You must make XORP textual MIB files
  ({\stt \$\{XORP\}/mibs/textual/*.txt}) accessible to the Net-SNMP command
  line tools.  Either set the {\stt MIBDIRS} environment variable
  ({\stt man snmpcmd(1)}) to point to that directory or copy those files to
  your MIBS directory (default is {\stt /usr/local/share/snmp/mibs}).  For
  instance (if {\stt sh} is the login shell):

  {\tt export MIBDIRS=+/usr/local/xorp/mibs/textual}

  \item You must tell Net-SNMP about specific MIB modules that you will
  be using.  The {\stt MIBS} environment variable can be used for that
  purpose. For BGP4-MIB you would do (if {\stt sh} is the login shell):

  {\tt export MIBS=+BGP4-MIB}

\end{itemize}
