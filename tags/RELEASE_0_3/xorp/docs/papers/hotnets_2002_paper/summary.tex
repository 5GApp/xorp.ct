% -*- mode: latex; tex-main-file: "pospaper.tex" -*-

\section{Summary}

We believe that much good Internet research is being frustrated by an
inability to deploy experimental router software at points in the
network where it makes most sense.  These problems affect a wide range
of research, including routing protocols themselves, active queue
management schemes, and so-called ``middlebox'' functionality such as
our own traffic normalizer~\cite{norm}.  Many of these problems would
not exist if the router software market more closely resembled the
end-system software market, which has well defined APIs for
application software, and high performance reliable open-source
operating systems that permit kernel protocol experimentation.  Our
vision for XORP is to provide just such an open
platform; one that is stable and fully featured enough for serious
production use (initially on edge-routers), but designed from the very
outset to support extensibility and experimentation without
compromising the stability of the core platform.

There are many unanswered questions that we still have to resolve.
For example, network managers simply wish to say {\em what} a router
should do; they don't typically care about how this is implemented.
Click starts from the point where you tell it precisely how to plumb
together the forwarding path.  Thus we need a forwarding path
configuration manager that, given the network manager's high-level
configuration, determines the combination of click elements and their
plumbing needed to implement the configuration.  The simplest solution
is a set of static profiles, but we believe that an optimizing
configuration manager might be able to do significantly better,
especially when it comes to reasoning about the placement and
interaction of elements such as filters.

Thus, while XORP is primarily intended to {\em enable} research, we
also believe that it will also further knowledge about how to
construct robust extensible networking systems.

Finally, while we do not expect to change the whole way the router
software market functions, it is not impossible that the widespread
use of an open software platform for routers might have this effect.
The ultimate measure of our success would be if commercial router
vendors either adopted XORP directly, or opened up their software
platforms in such a way that a market for router application software
is enabled.
