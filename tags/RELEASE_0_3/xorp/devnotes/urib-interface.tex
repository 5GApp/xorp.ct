\documentclass[11pt]{article}
\usepackage{times}
\usepackage{psfig}
\usepackage{amsmath}
\textwidth 6.5in
\topmargin 0.0in
\textheight 8.5in
\headheight 0in
\headsep 0in
\oddsidemargin 0in
\date{}
\title{URIB XRL Interface}
\author{Mark Handley}
%\twocolumn
\begin{document}
\parsep 0ex
\parskip 1.0ex
\parindent 0em
\noindent
\maketitle                            
\section{Introduction}
This document defines the URIB's external XRL interface.  Typically
this interface will be called by routing protocols, and by the router
manager to do initial configuration.  The URIB itself depends on the
FEA to be running to allow it to instantiate routes in the kernel.

NOTE: THIS DOCUMENT IS NOT GUARANTEED TO BE UP-TO-DATE.  
\newline
The
implementation in xorp/urib/xrl\_interface.cc is the definitive
specification.  
\section{XRL Interface}
The URIB provides the following XRLs which can be called by other modules:
\begin{itemize}
\item 
{\tt urib/add\_igp\_table/protocol[txt]{\it protocol}}
\newline
{\bf Returns:} {\tt result[i32]{\it result}}
\newline
This creates a new IGP routing table called {\it protocol} in the
URIB.  The special routing table ``connected'' is reserved for
directly connected interfaces, and should not be avoided.

The URIB will return an XrlAtomList containing the result.  A result
of zero indicates success.  Non-zero return codes are failures, but
the precise values are not yet standardized.

\item 
{\tt urib/add\_egp\_table/protocol[txt]{\it protocol}}
\newline
{\bf Returns:} {\tt result[i32]{\it result}}
\newline
This creates a new EGP routing table called {\it protocol} in the
URIB. The difference between an EGP and an IGP is that routes in an
IGP table MUST provide a nexthop that is on one of the directly
connected interfaces.  Routes in an EGP table may include nexthops
that are not immediate neighbors; the URIB will use the routes in the
IGP tables to resolve non-neighbor nexthops.

The URIB will return an XrlAtomList containing the result.  A result
of zero indicates success.  Non-zero return codes are failures, but
the precise values are not yet standardized.

\item 
{\tt urib/delete\_igp\_table/protocol[txt]{\it protocol}}
\newline
{\bf Returns:} {\tt result[i32]{\it result}}
\newline
This deletes the pre-existing IGP routing table called {\it protocol}
from the URIB.  Typically this should be called when a routing
protocol instance exits.

The URIB will return an XrlAtomList containing the result.  A result
of zero indicates success.  Non-zero return codes are failures, but
the precise values are not yet standardized.

\item 
{\tt urib/delete\_egp\_table/protocol[txt]{\it protocol}}
\newline
{\bf Returns:} {\tt result[i32]{\it result}}
\newline
This deletes the pre-existing EGP routing table called {\it protocol}
from the URIB.  Typically this should be called when a routing
protocol instance exits.

The URIB will return an XrlAtomList containing the result.  A result
of zero indicates success.  Non-zero return codes are failures, but
the precise values are not yet standardized.

\item 
{\tt urib/add\_route/protocol[txt]{\it protocol}\& netbase[ipv4]{\it net}\&
\newline
...\hspace{0.25in}masklen[i32]{\it masklen}\& nexthop[ipv4]{\it nexthop}}
\newline
{\bf Returns:} {\tt result[i32]{\it result}}
\newline
{\tt add\_route} adds the specified route to the routing table called {\it
protocol} in the URIB.  The routing table must already have been
created, and the route must not be a duplicate of an existing route
(although it may overlap existing routes with a different prefix
length).  {\it netbase} is the base address of the route's subnet.
{\it masklen} is the prefix (mask) length in bits.  {\it nexthop} is
the router via which packets to this subnet should be forwarded.  For
example, protocol=''ospf'', netbase=10.1.2.0, masklen=24 and
nexthop=128.16.64.1 would add a route to subnet 10.1.2.0/24 via
gateway 128.16.64.1 to the routing table associate with OSPF.

The URIB will return an XrlAtomList containing the result.  A result
of zero indicates success.  Non-zero return codes are failures, but
the precise values are not yet standardized.

\item 
{\tt urib/delete\_route/protocol[txt]{\it protocol}\& netbase[ipv4]{\it net}\&
\newline
...\hspace{0.25in}masklen[i32]{\it masklen}}
\newline
{\bf Returns:} {\tt result[i32]{\it result}}
\newline
{\tt delete\_route} deletes the specified route from the routing table
called {\it protocol} in the URIB.  The routing table must already
have been created, and the route must already exist.  {\it netbase} is
the base address of the route's subnet.  {\it masklen} is the prefix
(mask) length in bits.  {\it nexthop} is the router via which packets
to this subnet should be forwarded.  The route will only be deleted
from the table specified by {\it protocol}, irrespective of whether or
not it exists in other routing tables.

The URIB will return an XrlAtomList containing the result.  A result
of zero indicates success.  Non-zero return codes are failures, but
the precise values are not yet standardized.

\item 
{\tt urib/lookup\_route/protocol[txt]{\it protocol}\& addr[ipv4]{\it addr}}
\newline
{\bf Success Returns:} {\tt result[i32]0\& nexthop[ipv4]{\it nexthop}}
\newline
{\bf Failure Returns:} {\tt result[i32]-1}
\newline
{\tt lookup\_route} performs a route lookup on the URIB's internal
routing tables to determine how the URIB believes an IPv4 packet
destined for {\it addr} would be routed.  If the route lookup was
successful, result=0 is returned, along with the IPv4 address of the
immediate neighbor that the packet would be forward to.  If no route
exists, only result=-1 is returned.

\item 
{\tt urib/new\_vif/name[txt]{\it name}\& type[i32]{\it type}\& ipv4addr[ipv4]{\it addr}}
\newline
{\bf Returns:} {\tt result[i32]{\it result}}
\newline
{\tt new\_vif} informs the URIB about a new virtual interface.  The
``connected'' route table will contain routes for immediately
connected interfaces, but the URIB needs to know about the router's
own IP address on these interfaces.  {\tt new\_vif} is used to inform
the URIB about an interface, and must be called before a route for
that interface is added.

The URIB will return an XrlAtomList containing the result.  A result
of zero indicates success.  Non-zero return codes are failures, but
the precise values are not yet standardized.

\item 
{\tt urib/redist\_enable/from[txt]{\it from}\& to[txt]{\it to}}
\newline
{\bf Returns:} {\tt result[i32]{\it result}}
\newline
{\tt redist\_enable} enables restribution of routes from routing table
{\it from} to routing table {\it to}.  Probably doesn't currently work
properly.

The URIB will return an XrlAtomList containing the result.  A result
of zero indicates success.  Non-zero return codes are failures, but
the precise values are not yet standardized.
\item 
{\tt urib/redist\_disable/from[txt]{\it from}\& to[txt]{\it to}}
\newline
{\bf Returns:} {\tt result[i32]{\it result}}
\newline
{\tt redist\_disable} disables restribution of routes from routing
table {\it from} to routing table {\it to}.  Route redistribution must
have previously been started with{\tt redist\_enable}.

The URIB will return an XrlAtomList containing the result.  A result
of zero indicates success.  Non-zero return codes are failures, but
the precise values are not yet standardized.
\end{itemize}
\end{document}


