%
% $XORP: xorp/docs/user_manual/static_routes.tex,v 1.11 2006/01/19 02:28:00 pavlin Exp $
%

\chapter{Static Routes}
\label{static_routes}
\section{Terminology and Concepts}

A static route is a manually configured route.  Static routes will not
automatically change if a link or neighboring router fails.  In
general, static routes should only be used for very simple network
topologies, or to override the behaviour of a dynamic routing protocol
for a small number of routes.

Static routes can be configured for IPv4 and IPv6.  Each route can be
specified as to be used for unicast forwarding, or as part of the
multicast topology used by multicast routing, or both.  

The term {\it RIB} refers to the router's {\it Routing Information
Base}.  This is the collection of all routes the router has learned
from its configuration or from its dynamic routing protocols.  The RIB
maintains separate collections of routes for IPv4 and IPv6.  Within
each of those collections, the router also maintains separate route
tables for unicast routes and for multicast routes.  Unicast routes
will be used to determine the forwarding table used for unicast packet
forwarding.  Multicast routes do not directly determine the multicast
forwarding table, but instead are used by multicast routing protocols
such as PIM.  PIM uses this to determine the RPF (Reverse-Path
Forwarding) information\footnote{The RPF information represents the path back
to a source.} needed to route 
multicast control information that in turn sets up the multicast
forwarding tree.  The part of the {\it RIB} used to contain multicast
topology information is called the {\it Multicast RIB} or {\it MRIB}.

\newpage
\section{Configuration of Static Routes}

When a static route is specified, it is necessary to indicate not only
the {\it destination subnet} and {\it next-hop} router, but also
whether the route should be placed in the unicast RIB or in the MRIB
or both.

\subsection{Configuration Syntax}

The syntax for defining static routes is shown below.

\vspace{0.1in}
\noindent\framebox[\textwidth][l]{\scriptsize
\begin{minipage}{6in}
\begin{alltt}
\begin{tabbing}
xx\=xx\=xx\=xx\=xx\=\kill
protocols \{\\
\>static \{\\
\>\>targetname: {\it text}\\
\>\>disable: {\it bool}\\
\>\>route4 {\it IPv4-addr}/{\it int(0..32)} \{\\
\>\>\>next-hop: {\it IPv4-addr}\\
\>\>\>metric: {\it uint}\\
\>\>\}\\
\>\>mrib-route4 {\it IPv4-addr}/{\it int(0..32)} \{\\
\>\>\>next-hop: {\it IPv4-addr}\\
\>\>\>metric: {\it uint}\\
\>\>\}\\
\\
\>\>route6 {\it IPv6-addr}/{\it int(0..128)} \{\\
\>\>\>next-hop: {\it IPv6-addr}\\
\>\>\>metric: {\it uint}\\
\>\>\}\\
\>\>mrib-route6 {\it IPv6-addr}/{\it int(0..128)} \{\\
\>\>\>next-hop: {\it IPv6-addr}\\
\>\>\>metric: {\it uint}\\
\>\>\}\\
\\
\>\>interface-route4 {\it IPv4-addr}/{\it int(0..32)} \{\\
\>\>\>next-hop-interface: {\it text}\\
\>\>\>next-hop-vif: {\it text}\\
\>\>\>next-hop-router: {\it IPv4-addr}\\
\>\>\>metric: {\it uint}\\
\>\>\}\\
\>\>mrib-interface-route4 {\it IPv4-addr}/{\it int(0..32)} \{\\
\>\>\>next-hop-interface: {\it text}\\
\>\>\>next-hop-vif: {\it text}\\
\>\>\>next-hop-router: {\it IPv4-addr}\\
\>\>\>metric: {\it uint}\\
\>\>\}\\
\\
\>\>interface-route6 {\it IPv6-addr}/{\it int(0..128)} \{\\
\>\>\>next-hop-interface: {\it text}\\
\>\>\>next-hop-vif: {\it text}\\
\>\>\>next-hop-router: {\it IPv6-addr}\\
\>\>\>metric: {\it uint}\\
\>\>\}\\
\>\>mrib-interface-route6 {\it IPv6-addr}/{\it int(0..128)} \{\\
\>\>\>next-hop-interface: {\it text}\\
\>\>\>next-hop-vif: {\it text}\\
\>\>\>next-hop-router: {\it IPv6-addr}\\
\>\>\>metric: {\it uint}\\
\>\>\}\\
\>\}\\
\}
\end{tabbing}
\end{alltt}
\end{minipage}
}
\vspace{0.1in}

The configuration parameters are used as follows:
\begin{description}
\item{\tt protocols}: this delimits the configuration for all routing
  protocols in the XORP router configuration.  It is mandatory that
  BGP configuration is under the {\stt protocols} node in the
  configuration.
\item{\tt static}: the delimits the part of the router configuration
  that is related to configuring static routes.
\item{\tt targetname}: this is the name for this instance of
  static\_routes.  It defaults to ``{\stt static\_routes}'', and it is
  not recommended that this default is overridden under normal usage
  scenarios.
\item{\tt disable}: this takes the value {\stt true} or {\stt false},
  and determines whether any static routes are installed or not~\footnote{Note
  that prior to XORP Release-1.1, the {\tt enable} flag was used instead of
  {\tt disable}.}.
  Setting it to {\stt true} has the same effect as deleting the whole
  static routes configuration, but without losing what the old
  configuration actually was.
\item{\tt route4}: this specifies an IPv4 unicast route to be
  installed in the RIB.  The parameter is an IPv4 destination subnet
  expressed in the form {\it address/prefix-length}.

  Each {\stt route4}: specification takes the following attributes:
\begin{description}
\item{\tt next-hop}: this specifies the IPv4 address of the nexthop
  router towards the destination subnet~\footnote{Note that prior to the XORP
  Release-1.1, the {\tt nexthop} attribute was used instead of {\tt
  next-hop}.}. It is mandatory.
\item{\tt metric}: this specifies the routing metric or cost for this
  route.  It is a non-negative integer.  The metric for a static route
  is not directly used to decide which route to use, but may affect
  the choice of routes for protocols such as BGP and PIM-SM that
  indirectly use this information.  For example, BGP uses the IGP
  metric to the nexthop to decide between alternative routes as part
  of its decision process.  As with all routing metrics, lower values
  indicate better routes.
\end{description}
\item{\tt mrib-route4}: this specifies an IPv4 multicast route to be
  installed in the Multicast RIB.  The parameter is an IPv4
  destination subnet expressed in the form {\it
  address/prefix-length}.  This route will not directly affect
  forwarding, but will be used by multicast routing protocols such as
  PIM-SM to control how multicast trees are formed.

  An {\stt mrib-route4} specification takes the same attributes as a
  {\stt route4} specification.
\item{\tt route6}: this specifies an IPv6 unicast route to be
  installed in the RIB.  The parameter is an IPv6 destination subnet
  expressed in the form {\it address/prefix-length}.

  Each {\stt route6}: specification takes the following attributes:
\begin{description}
\item{\tt next-hop}: this specifies the IPv6 address of the nexthop
  router towards the destination subnet. It is mandatory.
\item{\tt metric}: this specifies the routing metric or cost for this
  route.  See {\stt route4 metric} for details.
\end{description}
\item{\tt mrib-route6}: this specifies an IPv4 multicast route to be
  installed in the Multicast RIB.  The parameter is an IPv6
  destination subnet expressed in the form {\it
  address/prefix-length}.  This route will not directly affect
  forwarding, but will be used by multicast routing protocols such as
  PIM-SM to control how multicast trees are formed.

  An {\stt mrib-route6} specification takes the same attributes as a
  {\stt route6} specification.

\item{\tt interface-route4}: this specifies an IPv4 unicast route to be
  installed in the RIB.  The parameter is an IPv4 destination subnet
  expressed in the form {\it address/prefix-length}. Typically, this
  specification will be used in wireless environment to install
  static routes where this router and next-hop router don't share the same
  subnet address on some (wireless) interface.

  Each {\stt interface-route4}: specification takes the following attributes:
\begin{description}
\item{\tt next-hop-interface}: this specifies the name of the nexthop
  interface towards the destination subnet. It is mandatory.
\item{\tt next-hop-vif}: this specifies the name of the nexthop
  vif towards the destination subnet. It is mandatory.
\item{\tt next-hop-router}: this specifies the IPv4 address of the nexthop
  router towards the destination subnet.
\item{\tt metric}: this specifies the routing metric or cost for this
  route.  See {\stt route4 metric} for details.
\end{description}

\end{description}

The {\tt mrib-interface-route4} specification is same as the {\tt
interface-route4} specification, except that it is used to configure
IPv4 routes that are to be installed in the Multicast RIB.
The {\tt interface-route6} and {\tt mrib-interface-route6} specifications
are also same, except that they are used to configure IPv6 routes
(in the Unicast and the Multicast RIB respectively).

\newpage
\subsection{Example Configurations}
\vspace{0.1in}
\noindent\framebox[\textwidth][l]{\scriptsize
\begin{minipage}{6in}
\begin{alltt}
\begin{tabbing}
xx\=xx\=xx\=xx\=xx\=\kill
protocols \{\\
\>static \{\\
\>\>route4 10.20.0.0/16 \{\\
\>\>\>next-hop: 10.10.10.20\\
\>\>\>metric: 1\\
\>\>\}\\
\>\>mrib-route4 10.20.0.0/16 \{\\
\>\>\>next-hop: 10.10.10.30\\
\>\>\>metric: 1\\
\>\>\}\\
\>\>\\
\>\>route6 20:20:20:20::/64 \{\\
\>\>\>next-hop: 10:10:10:10:10:10:10:20\\
\>\>\>metric: 1\\
\>\>\}\\
\>\>mrib-route6 20:20:20:20::/64 \{\\
\>\>\>next-hop: 10:10:10:10:10:10:10:30\\
\>\>\>metric: 1\\
\>\>\}\\
\>\>interface-route4 10.30.0.0/16 \{\\
\>\>\>next-hop-interface: rl0\\
\>\>\>next-hop-vif: rl0\\
\>\>\>metric: 1\\
\>\>\}\\
\>\>mrib-interface-route4 10.30.0.0/16 \{\\
\>\>\>next-hop-interface: rl1\\
\>\>\>next-hop-vif: rl1\\
\>\>\>metric: 1\\
\>\>\}\\
\>\>\\
\>\>interface-route6 30:30:30:30::/64 \{\\
\>\>\>next-hop-interface: rl0\\
\>\>\>next-hop-vif: rl0\\
\>\>\>metric: 1\\
\>\>\}\\
\>\>mrib-interface-route6 30:30:30:30::/64 \{\\
\>\>\>next-hop-interface: rl1\\
\>\>\>next-hop-vif: rl1\\
\>\>\>metric: 1\\
\>\>\}\\
\>\}\\
\}
\end{tabbing}
\end{alltt}
\end{minipage}
}
\vspace{0.1in}


\section{Monitoring Static Routes}

IPv4 unicast static routes can be displayed using the command {\stt show route
table ipv4 unicast static}:

\vspace{0.1in}
\noindent\framebox[\textwidth][l]{\scriptsize
\begin{minipage}{6in}
\begin{alltt}
\begin{tabbing}
xxxxxxxxxxxxxxxx\=\kill
user@hostname> \textbf{show route table ipv4 unicast static}\\
192.168.0.0/24 \>[static(1)/1]\\
               \>> to 192.150.187.1 via fxp0/fxp0\\
192.168.1.0/24 \>[static(1)/1]\\
               \>> to 192.150.187.2 via fxp0/fxp0
\end{tabbing}
\end{alltt}
\end{minipage}
}

\vspace{0.1in} 
The information shown for each route not only indicates the configured
information (network, nexthop and metric), but also the interface
and vif via which this route will forward packets.

If the nexthop is not actually reachable, the route will not be shown
by this command because there is not current interface or vif.

\vspace{0.2in}
IPv6 unicast static routes can be displayed using the command {\stt show route
table ipv6 unicast static}.

The Multicast RIB static routes can be displayed using the commands
{\stt show route table ipv4 multicast static} and
{\stt show route table ipv6 multicast static} for IPv4 and IPv6 respectively.
