%
% $XORP: xorp/docs/user_manual/diagnostics.tex,v 1.4 2005/11/11 06:24:02 pavlin Exp $
%

\chapter{Diagnostics and Debugging}

\section{Debugging and Diagnostic Commands}

XORP supports several operational commands in \xorpsh
that can be used for debugging or diagnostics purpose.

The {\stt ping <host>} command can be used to test if a network host
responds to ICMP ECHO\_REQUEST packets:

\vspace{0.1in}
\noindent\framebox[\textwidth][l]{\scriptsize
\begin{minipage}{6in}
\begin{alltt}
\begin{tabbing}
xxxxxxxxxxxxxxxxxx\=\kill
user@hostname> \textbf{ping 10.3.0.2}\\
PING 10.3.0.2 (10.3.0.2): 56 data bytes\\
64 bytes from 10.3.0.2: icmp\_seq=0 ttl=64 time=0.281 ms\\
64 bytes from 10.3.0.2: icmp\_seq=1 ttl=64 time=0.244 ms\\
64 bytes from 10.3.0.2: icmp\_seq=2 ttl=64 time=0.302 ms\\
64 bytes from 10.3.0.2: icmp\_seq=3 ttl=64 time=0.275 ms\\
user@hostname> ping 10.3.0.2\\
Command interrupted!
\end{tabbing}
\end{alltt}
\end{minipage}
}
\vspace{0.1in}

The {\stt ping} command can be interrupted by the {\stt Ctrl-C} key
combination.

The {\stt traceroute <host>} command can be used to print the route packets
take to a network host:

\vspace{0.1in}
\noindent\framebox[\textwidth][l]{\scriptsize
\begin{minipage}{6in}
\begin{alltt}
\begin{tabbing}
x\=xxx\=xxxxxxxxxxxxxxxxxxxxx\=xxxxxxxxxx\=xxxxxxxxxx\=\kill
user@hostname> \textbf{traceroute 10.4.0.2}\\
traceroute to 10.4.0.2 (10.4.0.2), 64 hops max, 44 byte packets\\
 \>1  \>xorp3-t2 (10.3.0.2)  \>0.451 ms  \>0.366 ms  \>0.384 ms\\
 \>2  \>xorp7-t0 (10.4.0.2)  \>0.596 ms  \>0.499 ms  \>0.527 ms
\end{tabbing}
\end{alltt}
\end{minipage}
}
\vspace{0.1in}

The {\stt traceroute} command can be interrupted by the {\stt Ctrl-C} key
combination.

The {\stt show host} commands can be used to display various information
about the host itself.

The {\stt show host date} command can be used to show the host current date:

\vspace{0.1in}
\noindent\framebox[\textwidth][l]{\scriptsize
\begin{minipage}{6in}
\begin{alltt}
\begin{tabbing}
xxxxxxxxxxxxxxxxxxxxxx\=\kill
user@hostname> \textbf{show host date}\\
Mon Apr 11 15:01:35 PDT 2005
\end{tabbing}
\end{alltt}
\end{minipage}
}
\vspace{0.1in}

The {\stt show host name} command can be used to show the host name:

\vspace{0.1in}
\noindent\framebox[\textwidth][l]{\scriptsize
\begin{minipage}{6in}
\begin{alltt}
\begin{tabbing}
xxxxxxxxxxxxxxxxxxxxxx\=\kill
user@hostname> \textbf{show host name}\\
xorp2
\end{tabbing}
\end{alltt}
\end{minipage}
}
\vspace{0.1in}

The {\stt show host os} command can be used to show details about the host
operating system:

\vspace{0.1in}
\noindent\framebox[\textwidth][l]{\scriptsize
\begin{minipage}{6in}
\begin{alltt}
\begin{tabbing}
xxx\=\kill
user@hostname> \textbf{show host os}\\
FreeBSD xorp2 4.9-RELEASE FreeBSD 4.9-RELEASE \#0: Wed May 19 18:56:49 PDT 2004\\
   \>atanu@xorpc.icir.org:/scratch/xorpc/u3/obj/home/xorpc/u2/freebsd4.9.usr/src/sys/XORP-4.9 i386
\end{tabbing}
\end{alltt}
\end{minipage}
}
\vspace{0.1in}

%% \section{Logging in to Unix}
%%
%% (regarding LiveCD)
