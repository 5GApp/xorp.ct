% -*- mode: latex; tex-main-file: "pospaper.tex" -*-

\section{Research and Reality}

The Internet has been fabulously successful; previously unenvisaged
new applications appear frequently, and changing usage patterns have
been accommodated with relative ease.  But underneath the successful
veneer, the low-level protocols that support the Internet have largely
ossified, and stresses are beginning to show.  Examples are security
and convergence problems with BGP routing, the inability to deploy
multicast, QoS, or IPv6, and the lack of effective defence mechanisms
against denial-of-service attacks.

At the same time, participation in the IETF by the research
community is at an all time low, and many researchers
have moved into areas such
as sensor-nets, Grid computing, and overlay networks where they
perceive they can still make a difference.  This trend is
understandable, but seems a recipe for disaster in the long term,
because the Internet industry is not doing research that would
solve these problems. Thus it behoves us to ask two questions: {\it
what is the root cause of this disconnect}, and {\it what can we do to
solve it?}

The root cause appears to be that the router software market is
closed, in the sense that if you buy a router from a vendor, then that
router will only run that vendor's software.  This makes it almost
impossible for researchers to experiment in real networks, or to
develop proof-of-concept code that might convince network operators
that there are alternatives to current practice.  Much innovation is
also driven by startup companies, but the lack of open router APIs
excludes this channel for change too.

The solution seems simple in principle: router software needs to have
open APIs for extensibility.  Unfortunately existing router software
was not written with third-party extension in mind, so doesn't
generally include the right hooks, extension mechanisms and security
boundaries.  
How then can we enable a pathway that permits research and
experimentation to be performed in production environments whilst
minimally impacting existing network services?  In part, this
is the same problem that Active Networks attempted to solve, but we
believe that a much more conservative approach is more likely to see
real-world usage.

Our vision is of an integrated open-source software router platform,
running on commodity hardware, that is viable as both a research and as a
production platform.  The software architecture should be designed
with extensibility as a primary goal, permitting experimental
protocol deployment with minimal risk to existing services.  Internet researchers needing access to router software could
then share a common platform for experimentation deployed in places
where real traffic conditions exist.  In these ways, the loop
between research and realistic real-world experimentation can be
closed, and innovation can take place much more freely.

We started work in early 2001, and made our 1.0 release this summer.
We call it XORP.


